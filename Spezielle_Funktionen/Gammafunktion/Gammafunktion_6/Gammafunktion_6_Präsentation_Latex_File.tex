\documentclass[10pt]{beamer}

\title{}
\author{Artur's \( \oint \) Mathematikstübchen}
\date{}


% ===== Packages =========
\usepackage[utf8]{inputenc}

\usepackage[natbibapa]{apacite}
\bibliographystyle{apacite}
\usepackage[ngerman]{babel}
\usepackage{graphicx}
\usepackage{fancyhdr}
\usepackage{amsmath}
\usepackage{amssymb}
\usepackage{graphicx}
\usepackage{MnSymbol}
\usepackage{enumitem}
\usepackage{amsthm}
\usepackage{mleftright}
\usepackage{dsfont}
\usepackage{tikz-cd}
\usepackage{physics}


\def\bC{\mathbb{C}}
\def\bD{\mathbb{D}}               
\def\bE{\mathbb{E}}
\def\bG{\mathbb{G}}
\def\bN{\mathbb{N}}
\def\bP{\mathbb{P}}
\def\bQ{\mathbb{Q}}
\def\bR{\mathbb{R}}
\def\bBarR{\bar{\mathbb{R}}}
\def\bY{\mathbb{Y}}



\def\mA{\mathcal{A}}
\def\mB{\mathcal{B}}
\def\mD{\mathcal{D}}
\def\mE{\mathcal{E}}
\def\mF{\mathcal{F}}
\def\mG{\mathcal{G}}
\def\mH{\mathcal{H}}
\def\mL{\mathcal{L}}
\def\mN{\mathcal{N}}
\def\mP{\mathcal{P}}
\def\mS{\mathcal{S}}
\def\mT{\mathcal{T}}
\def\mX{\mathcal{X}}
\def\mY{\mathcal{Y}}



\usetheme{Madrid}



% ======================== Beginn Document ========================

\begin{document}





% ======================== Begrüßung ==================

\begin{frame}
    \begin{center}
        \textbf{\huge Willkommen in der guten Stube \newline \newline :D}
    \end{center}
\end{frame}
% =====================================================



% ======================== Präsentation ==================

\begin{frame}
    \begin{alertblock}{Aufgabe}
        Man zeige, für alle \( \alpha > 0 \) gilt:
        \begin{align*}
            \int_{0}^{\infty} e^{-t^{\alpha}} \dd{t}
            = \frac{1}{\alpha} \Gamma\left( \frac{1}{\alpha} \right).
        \end{align*}
    \end{alertblock}
\end{frame}



\begin{frame}{Hilfsmittel}
    
\end{frame}



\begin{frame}{Hilfsmittel}
    Die Gammafunktion ist für komplexe Zahlen \( z \in \bC \) mit \( Re(z) > 0 \) definiert als:
\end{frame}



\begin{frame}{Hilfsmittel}
    Die Gammafunktion ist für komplexe Zahlen \( z \in \bC \) mit \( Re(z) > 0 \) definiert als:
    \begin{align*}
        \Gamma( z ) 
        =  \int_{0}^{\infty} t^{z - 1}e^{-t} \dd{t}.
    \end{align*}
\end{frame}




\begin{frame}{Beweis}
    
\end{frame}



\begin{frame}{Beweis}
    Sei \( \alpha > 0 \) eine positive reelle Zahl.
\end{frame}



\begin{frame}{Beweis}
    Sei \( \alpha > 0 \) eine positive reelle Zahl. Wir substituieren \( x = t^{\alpha} \).
\end{frame}



\begin{frame}{Beweis}
    Sei \( \alpha > 0 \) eine positive reelle Zahl. Wir substituieren \( x = t^{\alpha} \). Daraus folgt \( t = x^{\frac{1}{\alpha}} \).
\end{frame}



\begin{frame}{Beweis}
    Sei \( \alpha > 0 \) eine positive reelle Zahl. Wir substituieren \( x = t^{\alpha} \). Daraus folgt \( t = x^{\frac{1}{\alpha}} \) und:
\end{frame}



\begin{frame}{Beweis}
    Sei \( \alpha > 0 \) eine positive reelle Zahl. Wir substituieren \( x = t^{\alpha} \). Daraus folgt \( t = x^{\frac{1}{\alpha}} \) und:
    \begin{align*}
        \frac{\dd{x}}{\dd{t}}
        & = \alpha t^{\alpha - 1}
    \end{align*}
\end{frame}



\begin{frame}{Beweis}
    Sei \( \alpha > 0 \) eine positive reelle Zahl. Wir substituieren \( x = t^{\alpha} \). Daraus folgt \( t = x^{\frac{1}{\alpha}} \) und:
    \begin{align*}
        \frac{\dd{x}}{\dd{t}}
        & = \alpha t^{\alpha - 1}
        \quad\quad \Leftrightarrow \quad\quad dt = \frac{1}{\alpha} t^{1 - \alpha} \dd{x}.
    \end{align*}
\end{frame}



\begin{frame}{Beweis}
    Sei \( \alpha > 0 \) eine positive reelle Zahl. Wir substituieren \( x = t^{\alpha} \). Daraus folgt \( t = x^{\frac{1}{\alpha}} \) und:
    \begin{align*}
        \frac{\dd{x}}{\dd{t}}
        & = \alpha t^{\alpha - 1}
        \quad\quad \Leftrightarrow \quad\quad dt = \frac{1}{\alpha} t^{1 - \alpha} \dd{x}.
    \end{align*}
    Mit dieser Substitution folgt:
\end{frame}


\begin{frame}{Beweis}
    Sei \( \alpha > 0 \) eine positive reelle Zahl. Wir substituieren \( x = t^{\alpha} \). Daraus folgt \( t = x^{\frac{1}{\alpha}} \) und:
    \begin{align*}
        \frac{\dd{x}}{\dd{t}}
        & = \alpha t^{\alpha - 1}
        \quad\quad \Leftrightarrow \quad\quad dt = \frac{1}{\alpha} t^{1 - \alpha} \dd{x}.
    \end{align*}
    Mit dieser Substitution folgt:
    \begin{align*}
        \int_{0}^{\infty} e^{-t^{\alpha}} \dd{t}
    \end{align*}
\end{frame}



\begin{frame}{Beweis}
    Sei \( \alpha > 0 \) eine positive reelle Zahl. Wir substituieren \( x = t^{\alpha} \). Daraus folgt \( t = x^{\frac{1}{\alpha}} \) und:
    \begin{align*}
        \frac{\dd{x}}{\dd{t}}
        & = \alpha t^{\alpha - 1}
        \quad\quad \Leftrightarrow \quad\quad dt = \frac{1}{\alpha} t^{1 - \alpha} \dd{x}.
    \end{align*}
    Mit dieser Substitution folgt:
    \begin{align*}
        \int_{0}^{\infty} e^{-t^{\alpha}} \dd{t}
        & = \int_{0}^{\infty} \frac{1}{\alpha} t^{1 - \alpha} e^{-x} \dd{x}
    \end{align*}
\end{frame}



\begin{frame}{Beweis}
    Sei \( \alpha > 0 \) eine positive reelle Zahl. Wir substituieren \( x = t^{\alpha} \). Daraus folgt \( t = x^{\frac{1}{\alpha}} \) und:
    \begin{align*}
        \frac{\dd{x}}{\dd{t}}
        & = \alpha t^{\alpha - 1}
        \quad\quad \Leftrightarrow \quad\quad dt = \frac{1}{\alpha} t^{1 - \alpha} \dd{x}.
    \end{align*}
    Mit dieser Substitution folgt:
    \begin{align*}
        \int_{0}^{\infty} e^{-t^{\alpha}} \dd{t}
        & = \int_{0}^{\infty} \frac{1}{\alpha} t^{1 - \alpha} e^{-x} \dd{x} \\
        & = \frac{1}{\alpha} \int_{0}^{\infty} x^{\frac{1 - \alpha}{\alpha}} e^{-x} \dd{x}
    \end{align*}
\end{frame}



\begin{frame}{Beweis}
    Sei \( \alpha > 0 \) eine positive reelle Zahl. Wir substituieren \( x = t^{\alpha} \). Daraus folgt \( t = x^{\frac{1}{\alpha}} \) und:
    \begin{align*}
        \frac{\dd{x}}{\dd{t}}
        & = \alpha t^{\alpha - 1}
        \quad\quad \Leftrightarrow \quad\quad dt = \frac{1}{\alpha} t^{1 - \alpha} \dd{x}.
    \end{align*}
    Mit dieser Substitution folgt:
    \begin{align*}
        \int_{0}^{\infty} e^{-t^{\alpha}} \dd{t}
        & = \int_{0}^{\infty} \frac{1}{\alpha} t^{1 - \alpha} e^{-x} \dd{x} \\
        & = \frac{1}{\alpha} \int_{0}^{\infty} x^{\frac{1 - \alpha}{\alpha}} e^{-x} \dd{x} \\
        & = \frac{1}{\alpha} \int_{0}^{\infty} x^{\frac{1}{\alpha} - 1} e^{-x} \dd{x}
    \end{align*}
\end{frame}



\begin{frame}{Beweis}
    Sei \( \alpha > 0 \) eine positive reelle Zahl. Wir substituieren \( x = t^{\alpha} \). Daraus folgt \( t = x^{\frac{1}{\alpha}} \) und:
    \begin{align*}
        \frac{\dd{x}}{\dd{t}}
        & = \alpha t^{\alpha - 1}
        \quad\quad \Leftrightarrow \quad\quad dt = \frac{1}{\alpha} t^{1 - \alpha} \dd{x}.
    \end{align*}
    Mit dieser Substitution folgt:
    \begin{align*}
        \int_{0}^{\infty} e^{-t^{\alpha}} \dd{t}
        & = \int_{0}^{\infty} \frac{1}{\alpha} t^{1 - \alpha} e^{-x} \dd{x} \\
        & = \frac{1}{\alpha} \int_{0}^{\infty} x^{\frac{1 - \alpha}{\alpha}} e^{-x} \dd{x} \\
        & = \frac{1}{\alpha} \int_{0}^{\infty} x^{\frac{1}{\alpha} - 1} e^{-x} \dd{x} \\
        & = \frac{1}{\alpha} \Gamma\left( \frac{1}{\alpha} \right).
    \end{align*}
\end{frame}
% ============================================================
\end{document}