\documentclass[10pt]{beamer}

\title{}
\author{Artur's \( \oint \) Mathematikstübchen}
\date{}


% ===== Packages =========
\usepackage[utf8]{inputenc}

\usepackage[natbibapa]{apacite}
\bibliographystyle{apacite}
\usepackage[ngerman]{babel}
\usepackage{graphicx}
\usepackage{fancyhdr}
\usepackage{amsmath}
\usepackage{amssymb}
\usepackage{graphicx}
\usepackage{MnSymbol}
\usepackage{enumitem}
\usepackage{amsthm}
\usepackage{mleftright}
\usepackage{dsfont}
\usepackage{tikz-cd}


\def\bC{\mathbb{C}}
\def\bD{\mathbb{D}}               
\def\bE{\mathbb{E}}
\def\bG{\mathbb{G}}
\def\bN{\mathbb{N}}
\def\bP{\mathbb{P}}
\def\bQ{\mathbb{Q}}
\def\bR{\mathbb{R}}
\def\bBarR{\bar{\mathbb{R}}}
\def\bY{\mathbb{Y}}



\def\mA{\mathcal{A}}
\def\mB{\mathcal{B}}
\def\mD{\mathcal{D}}
\def\mE{\mathcal{E}}
\def\mF{\mathcal{F}}
\def\mG{\mathcal{G}}
\def\mH{\mathcal{H}}
\def\mL{\mathcal{L}}
\def\mN{\mathcal{N}}
\def\mP{\mathcal{P}}
\def\mS{\mathcal{S}}
\def\mT{\mathcal{T}}
\def\mX{\mathcal{X}}
\def\mY{\mathcal{Y}}



\usetheme{Madrid}



% ======================== Beginn Document ========================

\begin{document}





% ======================== Begrüßung ==================

\begin{frame}
    \begin{center}
        \textbf{\huge Willkommen in der guten Stube \newline \newline :D}
    \end{center}
\end{frame}
% =====================================================



% ======================== Präsentation ==================

\begin{frame}
    \begin{alertblock}{Aufgabe}
        Man zeige, für alle \( n \in \bN \) gilt:
        \begin{align*}
            \Gamma\left( \frac{n}{2} \right) \cdot \Gamma\left( \frac{n + 1}{2} \right) 
            = \frac{\sqrt{\pi}}{2^{n - 1}} \cdot \Gamma\left( n \right).
        \end{align*}
    \end{alertblock}
\end{frame}



\begin{frame}{Eigenschaften}
    
\end{frame}



\begin{frame}{Eigenschaften}
    Für den Beweis verwenden wir die folgenden Eigenschaften der Gammafunktion:
\end{frame}



\begin{frame}{Eigenschaften}
    Für den Beweis verwenden wir die folgenden Eigenschaften der Gammafunktion:
    \begin{itemize}
        \item<1-> 1. \( \Gamma\left( \frac{1}{2} \right) = \sqrt{\pi} \).
        \item<2->  2. Für alle \( x \in \bR \setminus \left\{ 0, -1, -2, \ldots \right\} \) gilt: \( \Gamma\left( x + 1 \right) = x \cdot \Gamma\left( x \right) \).
    \end{itemize}
\end{frame}



\begin{frame}{Beweis}
 
\end{frame}



\begin{frame}{Beweis}
    Wir führen den Beweis mittels vollständiger Induktion. 
\end{frame}



\begin{frame}{Beweis}
    Wir führen den Beweis mittels vollständiger Induktion.

    \textbf{Induktionsanfang:}
\end{frame}



\begin{frame}{Beweis}
    Wir führen den Beweis mittels vollständiger Induktion.

    \textbf{Induktionsanfang:}

    Für \( n = 1 \) gilt:
\end{frame}



\begin{frame}{Beweis}
    Wir führen den Beweis mittels vollständiger Induktion.

    \textbf{Induktionsanfang:}

    Für \( n = 1 \) gilt:
    \begin{align*}
        \Gamma\left( \frac{1}{2} \right) \cdot \Gamma\left( \frac{1 + 1}{2} \right)
    \end{align*}
\end{frame}



\begin{frame}{Beweis}
    Wir führen den Beweis mittels vollständiger Induktion.

    \textbf{Induktionsanfang:}

    Für \( n = 1 \) gilt:
    \begin{align*}
        \Gamma\left( \frac{1}{2} \right) \cdot \Gamma\left( \frac{1 + 1}{2} \right)
        & = \Gamma\left( \frac{1}{2} \right) \cdot \Gamma\left( 1 \right) 
    \end{align*}
\end{frame}



\begin{frame}{Beweis}
    Wir führen den Beweis mittels vollständiger Induktion.

    \textbf{Induktionsanfang:}

    Für \( n = 1 \) gilt:
    \begin{align*}
        \Gamma\left( \frac{1}{2} \right) \cdot \Gamma\left( \frac{1 + 1}{2} \right)
        & = \Gamma\left( \frac{1}{2} \right) \cdot \Gamma\left( 1 \right) \\
        & = \sqrt{\pi} \cdot \Gamma\left( 1 \right) 
    \end{align*}
\end{frame}



\begin{frame}{Beweis}
    Wir führen den Beweis mittels vollständiger Induktion.

    \textbf{Induktionsanfang:}

    Für \( n = 1 \) gilt:
    \begin{align*}
        \Gamma\left( \frac{1}{2} \right) \cdot \Gamma\left( \frac{1 + 1}{2} \right)
        & = \Gamma\left( \frac{1}{2} \right) \cdot \Gamma\left( 1 \right) \\
        & = \sqrt{\pi} \cdot \Gamma\left( 1 \right) \\
        & = \frac{\sqrt{\pi}}{2^{1 - 1}} \cdot \Gamma\left( 1 \right).
    \end{align*}
\end{frame}



\begin{frame}{Beweis}
    Wir führen den Beweis mittels vollständiger Induktion.

    \textbf{Induktionsanfang:}

    Für \( n = 1 \) gilt:
    \begin{align*}
        \Gamma\left( \frac{1}{2} \right) \cdot \Gamma\left( \frac{1 + 1}{2} \right)
        & = \Gamma\left( \frac{1}{2} \right) \cdot \Gamma\left( 1 \right) \\
        & = \sqrt{\pi} \cdot \Gamma\left( 1 \right) \\
        & = \frac{\sqrt{\pi}}{2^{1 - 1}} \cdot \Gamma\left( 1 \right).
    \end{align*}
    \textbf{Induktionsvoraussetzung:}
\end{frame}



\begin{frame}{Beweis}
    Wir führen den Beweis mittels vollständiger Induktion.

    \textbf{Induktionsanfang:}

    Für \( n = 1 \) gilt:
    \begin{align*}
        \Gamma\left( \frac{1}{2} \right) \cdot \Gamma\left( \frac{1 + 1}{2} \right)
        & = \Gamma\left( \frac{1}{2} \right) \cdot \Gamma\left( 1 \right) \\
        & = \sqrt{\pi} \cdot \Gamma\left( 1 \right) \\
        & = \frac{\sqrt{\pi}}{2^{1 - 1}} \cdot \Gamma\left( 1 \right).
    \end{align*}
    \textbf{Induktionsvoraussetzung:}

    Angenommen für ein \( n \in \bN \) gelte:
\end{frame}



\begin{frame}{Beweis}
    Wir führen den Beweis mittels vollständiger Induktion.

    \textbf{Induktionsanfang:}

    Für \( n = 1 \) gilt:
    \begin{align*}
        \Gamma\left( \frac{1}{2} \right) \cdot \Gamma\left( \frac{1 + 1}{2} \right)
        & = \Gamma\left( \frac{1}{2} \right) \cdot \Gamma\left( 1 \right) \\
        & = \sqrt{\pi} \cdot \Gamma\left( 1 \right) \\
        & = \frac{\sqrt{\pi}}{2^{1 - 1}} \cdot \Gamma\left( 1 \right).
    \end{align*}
    \textbf{Induktionsvoraussetzung:}

    Angenommen für ein \( n \in \bN \) gelte:
    \begin{align*}
        \Gamma\left( \frac{n}{2} \right) \cdot \Gamma\left( \frac{n + 1}{2} \right) 
        = \frac{\sqrt{\pi}}{2^{n - 1}} \cdot \Gamma\left( n \right).
    \end{align*}
\end{frame}



\begin{frame}{Beweis}
    \textbf{Induktionsschritt:}
\end{frame}



\begin{frame}{Beweis}
    \textbf{Induktionsschritt:}

    Dann folgt für \( \left( n + 1 \right) \):
\end{frame}



\begin{frame}{Beweis}
    \textbf{Induktionsschritt:}

    Dann folgt für \( \left( n + 1 \right) \):
    \begin{align*}
        \Gamma\left( \frac{n + 1}{2} \right) \cdot \Gamma\left( \frac{n + 2}{2} \right)
    \end{align*}
\end{frame}



\begin{frame}{Beweis}
    \textbf{Induktionsschritt:}

    Dann folgt für \( \left( n + 1 \right) \):
    \begin{align*}
        \Gamma\left( \frac{n + 1}{2} \right) \cdot \Gamma\left( \frac{n + 2}{2} \right)
        & = \Gamma\left( \frac{n + 1}{2} \right) \cdot \Gamma\left( 1 + \frac{n}{2} \right)
    \end{align*}
\end{frame}



\begin{frame}{Beweis}
    \textbf{Induktionsschritt:}

    Dann folgt für \( \left( n + 1 \right) \):
    \begin{align*}
        \Gamma\left( \frac{n + 1}{2} \right) \cdot \Gamma\left( \frac{n + 2}{2} \right)
        & = \Gamma\left( \frac{n + 1}{2} \right) \cdot \Gamma\left( 1 + \frac{n}{2} \right) \\
        & = \Gamma\left( \frac{n + 1}{2} \right) \cdot \frac{n}{2} \cdot \Gamma\left( \frac{n}{2} \right)
    \end{align*}
\end{frame}



\begin{frame}{Beweis}
    \textbf{Induktionsschritt:}

    Dann folgt für \( \left( n + 1 \right) \):
    \begin{align*}
        \Gamma\left( \frac{n + 1}{2} \right) \cdot \Gamma\left( \frac{n + 2}{2} \right)
        & = \Gamma\left( \frac{n + 1}{2} \right) \cdot \Gamma\left( 1 + \frac{n}{2} \right) \\
        & = \Gamma\left( \frac{n + 1}{2} \right) \cdot \frac{n}{2} \cdot \Gamma\left( \frac{n}{2} \right) \\
        & = \frac{n}{2} \cdot \frac{\sqrt{\pi}}{2^{n - 1}} \cdot \Gamma\left( n \right)
    \end{align*}
\end{frame}



\begin{frame}{Beweis}
    \textbf{Induktionsschritt:}

    Dann folgt für \( \left( n + 1 \right) \):
    \begin{align*}
        \Gamma\left( \frac{n + 1}{2} \right) \cdot \Gamma\left( \frac{n + 2}{2} \right)
        & = \Gamma\left( \frac{n + 1}{2} \right) \cdot \Gamma\left( 1 + \frac{n}{2} \right) \\
        & = \Gamma\left( \frac{n + 1}{2} \right) \cdot \frac{n}{2} \cdot \Gamma\left( \frac{n}{2} \right) \\
        & = \frac{n}{2} \cdot \frac{\sqrt{\pi}}{2^{n - 1}} \cdot \Gamma\left( n \right) \\
        & = \frac{\sqrt{\pi}}{2^{n}} \cdot n \cdot \Gamma\left( n \right)
    \end{align*}
\end{frame}



\begin{frame}{Beweis}
    \textbf{Induktionsschritt:}

    Dann folgt für \( \left( n + 1 \right) \):
    \begin{align*}
        \Gamma\left( \frac{n + 1}{2} \right) \cdot \Gamma\left( \frac{n + 2}{2} \right)
        & = \Gamma\left( \frac{n + 1}{2} \right) \cdot \Gamma\left( 1 + \frac{n}{2} \right) \\
        & = \Gamma\left( \frac{n + 1}{2} \right) \cdot \frac{n}{2} \cdot \Gamma\left( \frac{n}{2} \right) \\
        & = \frac{n}{2} \cdot \frac{\sqrt{\pi}}{2^{n - 1}} \cdot \Gamma\left( n \right) \\
        & = \frac{\sqrt{\pi}}{2^{n}} \cdot n \cdot \Gamma\left( n \right) \\
        & = \frac{\sqrt{\pi}}{2^{n}} \cdot \Gamma\left( n + 1 \right).
    \end{align*}
\end{frame}



\begin{frame}{Bemerkung}
    
\end{frame}



\begin{frame}{Bemerkung}
    Die Formel
    \begin{align*}
        \Gamma\left( \frac{n}{2} \right) \cdot \Gamma\left( \frac{n + 1}{2} \right) 
            = \frac{\sqrt{\pi}}{2^{n - 1}} \cdot \Gamma\left( n \right)
    \end{align*}
\end{frame}



\begin{frame}{Bemerkung}
    Die Formel
    \begin{align*}
        \Gamma\left( \frac{n}{2} \right) \cdot \Gamma\left( \frac{n + 1}{2} \right) 
        = \frac{\sqrt{\pi}}{2^{n - 1}} \cdot \Gamma\left( n \right)
    \end{align*}
    ist ein Spezialfall der Legendreschen Verdopplungsformel
\end{frame}



\begin{frame}{Bemerkung}
    Die Formel
    \begin{align*}
        \Gamma\left( \frac{n}{2} \right) \cdot \Gamma\left( \frac{n + 1}{2} \right) 
        = \frac{\sqrt{\pi}}{2^{n - 1}} \cdot \Gamma\left( n \right)
    \end{align*}
    ist ein Spezialfall der Legendreschen Verdopplungsformel
    \begin{align*}
        \Gamma\left( \frac{z}{2} \right) \cdot \Gamma\left( \frac{z + 1}{2} \right) 
        = \frac{\sqrt{\pi}}{2^{z - 1}} \cdot \Gamma\left( z \right), \quad z \in \bC \setminus \left\{ 0, -1, -2, \ldots \right\}.
    \end{align*}
\end{frame}
% ============================================================

\end{document}