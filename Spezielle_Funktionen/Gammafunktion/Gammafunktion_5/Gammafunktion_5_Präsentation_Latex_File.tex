\documentclass[10pt]{beamer}

\title{}
\author{Artur's \( \oint \) Mathematikstübchen}
\date{}


% ===== Packages =========
\usepackage[utf8]{inputenc}

\usepackage[natbibapa]{apacite}
\bibliographystyle{apacite}
\usepackage[ngerman]{babel}
\usepackage{graphicx}
\usepackage{fancyhdr}
\usepackage{amsmath}
\usepackage{amssymb}
\usepackage{graphicx}
\usepackage{MnSymbol}
\usepackage{enumitem}
\usepackage{amsthm}
\usepackage{mleftright}
\usepackage{dsfont}
\usepackage{tikz-cd}
\usepackage{physics}


\def\bC{\mathbb{C}}
\def\bD{\mathbb{D}}               
\def\bE{\mathbb{E}}
\def\bG{\mathbb{G}}
\def\bN{\mathbb{N}}
\def\bP{\mathbb{P}}
\def\bQ{\mathbb{Q}}
\def\bR{\mathbb{R}}
\def\bBarR{\bar{\mathbb{R}}}
\def\bY{\mathbb{Y}}



\def\mA{\mathcal{A}}
\def\mB{\mathcal{B}}
\def\mD{\mathcal{D}}
\def\mE{\mathcal{E}}
\def\mF{\mathcal{F}}
\def\mG{\mathcal{G}}
\def\mH{\mathcal{H}}
\def\mL{\mathcal{L}}
\def\mN{\mathcal{N}}
\def\mP{\mathcal{P}}
\def\mS{\mathcal{S}}
\def\mT{\mathcal{T}}
\def\mX{\mathcal{X}}
\def\mY{\mathcal{Y}}



\usetheme{Madrid}



% ======================== Beginn Document ========================

\begin{document}





% ======================== Begrüßung ==================

\begin{frame}
    \begin{center}
        \textbf{\huge Willkommen in der guten Stube \newline \newline :D}
    \end{center}
\end{frame}
% =====================================================



% ======================== Präsentation ==================

\begin{frame}
    \begin{alertblock}{Aufgabe}
        Man zeige, für alle \( z \in \bC \) mit \( Re(z) > 0 \) die Gültigkeit der folgenden Abschätzung:
        \begin{align*}
            \left\vert \Gamma( z ) \right\vert
            \leq \Gamma( Re( z ) ).
        \end{align*}
    \end{alertblock}
\end{frame}



\begin{frame}{Hilfsmittel}
    
\end{frame}



\begin{frame}{Hilfsmittel}
    Für den Beweis verwenden wir folgende Hilfsmittel:
\end{frame}



\begin{frame}{Hilfsmittel}
    Für den Beweis verwenden wir folgende Hilfsmittel:
    \begin{itemize}
        \item<1->  1. Für alle \( z \in \bC \) mit \( Re( z ) > 0 \) gilt: \( \Gamma( z ) = \int_{0}^{\infty} t^{z - 1} e^{-t} \dd{t} \).
        \item<2->  2. Es gilt die Dreiecksungleichung für Integrale: \( \left\vert \int f( z ) \dd{z} \right\vert \leq \int \left\vert f( z ) \right\vert \dd{z} \).
        \item<3->  3. Für alle \( w, z \in \bC \) gilt: \( e^{w + z} = e^{w} \cdot e^{z} \).
        \item<4->  4. Für \( a > 0 \) und \( z \in \bC \) ist: \( a^{z} = e^{z \ln(a)} \). Insbesondere ist für alle \( w, z \in \bC \)
        \begin{align*}
            a^{w + z} 
            & = e^{\left( w + z \right) \ln(a)} \\
            & = e^{w \ln(a) + z \ln(a)} \\
            & = e^{w \ln(a)} \cdot e^{z \ln(a)} \\
            & = a^{w} \cdot a^{z}.
        \end{align*}
        \item<5->  5. Für \( x \in \bR \) gilt: \( \left\vert e^{i \cdot x} \right\vert = 1 \). Daraus folgt für \( a > 0 \) und \( x \in \bR \) \( \left\vert a^{i \cdot x} \right\vert = 1 \).
    \end{itemize}
\end{frame}



\begin{frame}{Beweis}
    
\end{frame}



\begin{frame}{Beweis}
    Sei \( z \in \bC \) eine komplexe Zahl mit \( Re(z) > 0 \).
\end{frame}



\begin{frame}{Beweis}
    Sei \( z \in \bC \) eine komplexe Zahl mit \( Re(z) > 0 \). Dann gilt:
\end{frame}



\begin{frame}{Beweis}
    Sei \( z \in \bC \) eine komplexe Zahl mit \( Re(z) > 0 \). Dann gilt:
    \begin{align*}
        \left\vert \Gamma( z ) \right\vert
    \end{align*}
\end{frame}



\begin{frame}{Beweis}
    Sei \( z \in \bC \) eine komplexe Zahl mit \( Re(z) > 0 \). Dann gilt:
    \begin{align*}
        \left\vert \Gamma( z ) \right\vert
        & = \left\vert \int_{0}^{\infty} t^{z - 1} e^{-t} \dd{t} \right\vert
    \end{align*}
\end{frame}



\begin{frame}{Beweis}
    Sei \( z \in \bC \) eine komplexe Zahl mit \( Re(z) > 0 \). Dann gilt:
    \begin{align*}
        \left\vert \Gamma( z ) \right\vert
        & = \left\vert \int_{0}^{\infty} t^{z - 1} e^{-t} \dd{t} \right\vert \\
        & \leq \int_{0}^{\infty} \left\vert t^{z - 1} e^{-t} \right\vert \dd{t}
    \end{align*}
\end{frame}



\begin{frame}{Beweis}
    Sei \( z \in \bC \) eine komplexe Zahl mit \( Re(z) > 0 \). Dann gilt:
    \begin{align*}
        \left\vert \Gamma( z ) \right\vert
        & = \left\vert \int_{0}^{\infty} t^{z - 1} e^{-t} \dd{t} \right\vert \\
        & \leq \int_{0}^{\infty} \left\vert t^{z - 1} e^{-t} \right\vert \dd{t} \\
        & = \int_{0}^{\infty} \left\vert t^{z}t^{-1} e^{-t} \right\vert \dd{t}
    \end{align*}
\end{frame}



\begin{frame}{Beweis}
    Sei \( z \in \bC \) eine komplexe Zahl mit \( Re(z) > 0 \). Dann gilt:
    \begin{align*}
        \left\vert \Gamma( z ) \right\vert
        & = \left\vert \int_{0}^{\infty} t^{z - 1} e^{-t} \dd{t} \right\vert \\
        & \leq \int_{0}^{\infty} \left\vert t^{z - 1} e^{-t} \right\vert \dd{t} \\
        & = \int_{0}^{\infty} \left\vert t^{z}t^{-1} e^{-t} \right\vert \dd{t} \\
        & = \int_{0}^{\infty} \left\vert t^{z} \right\vert t^{-1} e^{-t} \dd{t}
    \end{align*}
\end{frame}



\begin{frame}{Beweis}
    Sei \( z \in \bC \) eine komplexe Zahl mit \( Re(z) > 0 \). Dann gilt:
    \begin{align*}
        \left\vert \Gamma( z ) \right\vert
        & = \left\vert \int_{0}^{\infty} t^{z - 1} e^{-t} \dd{t} \right\vert \\
        & \leq \int_{0}^{\infty} \left\vert t^{z - 1} e^{-t} \right\vert \dd{t} \\
        & = \int_{0}^{\infty} \left\vert t^{z}t^{-1} e^{-t} \right\vert \dd{t} \\
        & = \int_{0}^{\infty} \left\vert t^{z} \right\vert t^{-1} e^{-t} \dd{t} \\
        & = \int_{0}^{\infty} \left\vert t^{Re(z) + i \cdot Im(z)} \right\vert t^{-1} e^{-t} \dd{t}
    \end{align*}
\end{frame}



\begin{frame}{Beweis}
    Sei \( z \in \bC \) eine komplexe Zahl mit \( Re(z) > 0 \). Dann gilt:
    \begin{align*}
        \left\vert \Gamma( z ) \right\vert
        & = \left\vert \int_{0}^{\infty} t^{z - 1} e^{-t} \dd{t} \right\vert \\
        & \leq \int_{0}^{\infty} \left\vert t^{z - 1} e^{-t} \right\vert \dd{t} \\
        & = \int_{0}^{\infty} \left\vert t^{z}t^{-1} e^{-t} \right\vert \dd{t} \\
        & = \int_{0}^{\infty} \left\vert t^{z} \right\vert t^{-1} e^{-t} \dd{t} \\
        & = \int_{0}^{\infty} \left\vert t^{Re(z) + i \cdot Im(z)} \right\vert t^{-1} e^{-t} \dd{t} \\
        & = \int_{0}^{\infty} \left\vert t^{Re(z)} t^{i \cdot Im(z)} \right\vert t^{-1} e^{-t} \dd{t}
    \end{align*}
\end{frame}



\begin{frame}{Beweis}
    Sei \( z \in \bC \) eine komplexe Zahl mit \( Re(z) > 0 \). Dann gilt:
    \begin{align*}
        \left\vert \Gamma( z ) \right\vert
        & = \left\vert \int_{0}^{\infty} t^{z - 1} e^{-t} \dd{t} \right\vert \\
        & \leq \int_{0}^{\infty} \left\vert t^{z - 1} e^{-t} \right\vert \dd{t} \\
        & = \int_{0}^{\infty} \left\vert t^{z}t^{-1} e^{-t} \right\vert \dd{t} \\
        & = \int_{0}^{\infty} \left\vert t^{z} \right\vert t^{-1} e^{-t} \dd{t} \\
        & = \int_{0}^{\infty} \left\vert t^{Re(z) + i \cdot Im(z)} \right\vert t^{-1} e^{-t} \dd{t} \\
        & = \int_{0}^{\infty} \left\vert t^{Re(z)} t^{i \cdot Im(z)} \right\vert t^{-1} e^{-t} \dd{t} \\
        & = \int_{0}^{\infty} \left\vert t^{Re(z)} \right\vert \left\vert t^{i \cdot Im(z)} \right\vert t^{-1} e^{-t} \dd{t}
    \end{align*}
\end{frame}



\begin{frame}{Beweis}
    Sei \( z \in \bC \) eine komplexe Zahl mit \( Re(z) > 0 \). Dann gilt:
    \begin{align*}
        \left\vert \Gamma( z ) \right\vert
        & = \left\vert \int_{0}^{\infty} t^{z - 1} e^{-t} \dd{t} \right\vert \\
        & \leq \int_{0}^{\infty} \left\vert t^{z - 1} e^{-t} \right\vert \dd{t} \\
        & = \int_{0}^{\infty} \left\vert t^{z}t^{-1} e^{-t} \right\vert \dd{t} \\
        & = \int_{0}^{\infty} \left\vert t^{z} \right\vert t^{-1} e^{-t} \dd{t} \\
        & = \int_{0}^{\infty} \left\vert t^{Re(z) + i \cdot Im(z)} \right\vert t^{-1} e^{-t} \dd{t} \\
        & = \int_{0}^{\infty} \left\vert t^{Re(z)} t^{i \cdot Im(z)} \right\vert t^{-1} e^{-t} \dd{t} \\
        & = \int_{0}^{\infty} \left\vert t^{Re(z)} \right\vert \left\vert t^{i \cdot Im(z)} \right\vert t^{-1} e^{-t} \dd{t} \\
        & = \int_{0}^{\infty} t^{Re(z) - 1} e^{-t} \dd{t}
    \end{align*}
\end{frame}



\begin{frame}{Beweis}
    Sei \( z \in \bC \) eine komplexe Zahl mit \( Re(z) > 0 \). Dann gilt:
    \begin{align*}
        \left\vert \Gamma( z ) \right\vert
        & = \left\vert \int_{0}^{\infty} t^{z - 1} e^{-t} \dd{t} \right\vert \\
        & \leq \int_{0}^{\infty} \left\vert t^{z - 1} e^{-t} \right\vert \dd{t} \\
        & = \int_{0}^{\infty} \left\vert t^{z}t^{-1} e^{-t} \right\vert \dd{t} \\
        & = \int_{0}^{\infty} \left\vert t^{z} \right\vert t^{-1} e^{-t} \dd{t} \\
        & = \int_{0}^{\infty} \left\vert t^{Re(z) + i \cdot Im(z)} \right\vert t^{-1} e^{-t} \dd{t} \\
        & = \int_{0}^{\infty} \left\vert t^{Re(z)} t^{i \cdot Im(z)} \right\vert t^{-1} e^{-t} \dd{t} \\
        & = \int_{0}^{\infty} \left\vert t^{Re(z)} \right\vert \left\vert t^{i \cdot Im(z)} \right\vert t^{-1} e^{-t} \dd{t} \\
        & = \int_{0}^{\infty} t^{Re(z) - 1} e^{-t} \dd{t} \\
        & = \Gamma( Re( z ) ).
    \end{align*}
\end{frame}
% ============================================================
\end{document}