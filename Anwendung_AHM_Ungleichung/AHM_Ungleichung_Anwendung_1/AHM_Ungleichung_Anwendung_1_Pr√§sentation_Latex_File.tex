\documentclass[10pt]{beamer}

\title{}
\author{Artur's \( \oint \) Mathematikstübchen}
\date{}


% ===== Packages =========
\usepackage[utf8]{inputenc}

\usepackage[natbibapa]{apacite}
\bibliographystyle{apacite}
\usepackage[ngerman]{babel}
\usepackage{graphicx}
\usepackage{fancyhdr}
\usepackage{amsmath}
\usepackage{amssymb}
\usepackage{graphicx}
\usepackage{MnSymbol}
\usepackage{enumitem}
\usepackage{amsthm}
\usepackage{mleftright}
\usepackage{dsfont}
\usepackage{tikz-cd}


\def\bC{\mathbb{C}}
\def\bD{\mathbb{D}}               
\def\bE{\mathbb{E}}
\def\bG{\mathbb{G}}
\def\bN{\mathbb{N}}
\def\bP{\mathbb{P}}
\def\bQ{\mathbb{Q}}
\def\bR{\mathbb{R}}
\def\bBarR{\bar{\mathbb{R}}}
\def\bY{\mathbb{Y}}



\def\mA{\mathcal{A}}
\def\mB{\mathcal{B}}
\def\mD{\mathcal{D}}
\def\mE{\mathcal{E}}
\def\mF{\mathcal{F}}
\def\mG{\mathcal{G}}
\def\mH{\mathcal{H}}
\def\mL{\mathcal{L}}
\def\mN{\mathcal{N}}
\def\mP{\mathcal{P}}
\def\mS{\mathcal{S}}
\def\mT{\mathcal{T}}
\def\mX{\mathcal{X}}
\def\mY{\mathcal{Y}}



\usetheme{Madrid}



% ======================== Beginn Document ========================

\begin{document}





% ======================== Begrüßung ==================

\begin{frame}
    \begin{center}
        \textbf{\huge Willkommen in der guten Stube \newline \newline :D}
    \end{center}
\end{frame}
% =====================================================



% ======================== Präsentation ==================

\begin{frame}
    \begin{alertblock}{Aufgabe}
        Seien \( s, t, u > 0 \) drei positive reelle Zahlen mit \( s \neq t + u \), \( t \neq s+ u \) und \( u \neq s + t \). Man zeige die Gültigkeit der folgenden Abschätzung:
        \begin{align*}
            \frac{1}{t + u - s} + \frac{1}{s + u - t} + \frac{1}{s + t - u}
            \geq \frac{9}{s + t + u}.
        \end{align*}
    \end{alertblock}
\end{frame}



\begin{frame}{Hilfsabschätzung}
    
\end{frame}



\begin{frame}{Hilfsabschätzung}
    Für den Beweis verwenden wir die Ungleichung zwischen dem harmonischen und arithmetischen Mittel:
\end{frame}



\begin{frame}{Hilfsabschätzung}
    Für den Beweis verwenden wir die Ungleichung zwischen dem harmonischen und arithmetischen Mittel:
    \begin{block}{Hilfsabschätzung}
        Für alle \( a_{1}, a_{2}, \ldots, a_{p} > 0 \), \( p \in \bN \), gilt die Abschätzung:
        \begin{align*}
            \frac{p}{\frac{1}{a_{1}} + \frac{1}{a_{2}} + \ldots + \frac{1}{a_{p}}}
            \leq \frac{a_{1} + a_{2} + \ldots + a_{p}}{p}.
        \end{align*}
    \end{block}
\end{frame}



\begin{frame}{Hilfsabschätzung}
    Für den Beweis verwenden wir die Ungleichung zwischen dem harmonischen und arithmetischen Mittel:
    \begin{block}{Hilfsabschätzung}
        Für alle \( a_{1}, a_{2}, \ldots, a_{p} > 0 \), \( p \in \bN \), gilt die Abschätzung:
        \begin{align*}
            \frac{p}{\frac{1}{a_{1}} + \frac{1}{a_{2}} + \ldots + \frac{1}{a_{p}}}
            \leq \frac{a_{1} + a_{2} + \ldots + a_{p}}{p}.
        \end{align*}
    \end{block}
    Insbesondere gilt für \( p = 3 \):   
\end{frame}



\begin{frame}{Hilfsabschätzung}
    Für den Beweis verwenden wir die Ungleichung zwischen dem geometrischen und arithmetischen Mittel:
    \begin{block}{Hilfsabschätzung}
        Für alle \( a_{1}, a_{2}, \ldots, a_{p} > 0 \), \( p \in \bN \), gilt die Abschätzung:
        \begin{align*}
            \frac{p}{\frac{1}{a_{1}} + \frac{1}{a_{2}} + \ldots + \frac{1}{a_{p}}}
            \leq \frac{a_{1} + a_{2} + \ldots + a_{p}}{p}.
        \end{align*}
    \end{block}
    Insbesondere gilt für \( p = 3 \): 
    \begin{align*}
        \frac{3}{\frac{1}{a_{1}} + \frac{1}{a_{2}} + \frac{1}{a_{3}}}
        \leq \frac{a_{1} + a_{2} + a_{3}}{3},
    \end{align*}
    für alle \( a_{1}, a_{2}, a_{3} > 0 \).
\end{frame}



\begin{frame}{Beweis}
    
\end{frame}


\begin{frame}{Beweis}
    Seien \( s, t, u > 0 \) mit \( s \neq t + u \), \( t \neq s+ u \) und \( u \neq s + t \).
\end{frame}



\begin{frame}{Beweis}
    Seien \( s, t, u > 0 \) mit \( s \neq t + u \), \( t \neq s+ u \) und \( u \neq s + t \). Dann gilt:
\end{frame}



\begin{frame}{Beweis}
    Seien \( s, t, u > 0 \) mit \( s \neq t + u \), \( t \neq s+ u \) und \( u \neq s + t \). Dann gilt:
    \begin{align*}
        \frac{1}{t + u - s} + \frac{1}{s + u - t} + \frac{1}{s + t - u}
    \end{align*}
\end{frame}



\begin{frame}{Beweis}
    Seien \( s, t, u > 0 \) mit \( s \neq t + u \), \( t \neq s+ u \) und \( u \neq s + t \). Dann gilt:
    \begin{align*}
        \frac{1}{t + u - s} + \frac{1}{s + u - t} + \frac{1}{s + t - u}
        & = 3 \cdot \frac{\frac{1}{t + u - s} + \frac{1}{s + u - t} + \frac{1}{s + t - u}}{3}
    \end{align*}
\end{frame}



\begin{frame}{Beweis}
    Seien \( s, t, u > 0 \) mit \( s \neq t + u \), \( t \neq s+ u \) und \( u \neq s + t \). Dann gilt:
    \begin{align*}
        \frac{1}{t + u - s} + \frac{1}{s + u - t} + \frac{1}{s + t - u}
        & = 3 \cdot \frac{\frac{1}{t + u - s} + \frac{1}{s + u - t} + \frac{1}{s + t - u}}{3} \\
        & \geq 3 \cdot \frac{3}{\frac{1}{\frac{1}{t + u - s}} + \frac{1}{\frac{1}{s + u - t}} + \frac{1}{\frac{1}{s + t - u}}}
    \end{align*}
\end{frame}



\begin{frame}{Beweis}
    Seien \( s, t, u > 0 \) mit \( s \neq t + u \), \( t \neq s+ u \) und \( u \neq s + t \). Dann gilt:
    \begin{align*}
        \frac{1}{t + u - s} + \frac{1}{s + u - t} + \frac{1}{s + t - u}
        & = 3 \cdot \frac{\frac{1}{t + u - s} + \frac{1}{s + u - t} + \frac{1}{s + t - u}}{3} \\
        & \geq 3 \cdot \frac{3}{\frac{1}{\frac{1}{t + u - s}} + \frac{1}{\frac{1}{s + u - t}} + \frac{1}{\frac{1}{s + t - u}}} \\
        & = \frac{9}{t + u - s + s + u - t + s + t - u}
    \end{align*}
\end{frame}



\begin{frame}{Beweis}
    Seien \( s, t, u > 0 \) mit \( s \neq t + u \), \( t \neq s+ u \) und \( u \neq s + t \). Dann gilt:
    \begin{align*}
        \frac{1}{t + u - s} + \frac{1}{s + u - t} + \frac{1}{s + t - u}
        & = 3 \cdot \frac{\frac{1}{t + u - s} + \frac{1}{s + u - t} + \frac{1}{s + t - u}}{3} \\
        & \geq 3 \cdot \frac{3}{\frac{1}{\frac{1}{t + u - s}} + \frac{1}{\frac{1}{s + u - t}} + \frac{1}{\frac{1}{s + t - u}}} \\
        & = \frac{9}{t + u - s + s + u - t + s + t - u} \\
        & = \frac{9}{s + t + u}.
    \end{align*}
\end{frame}
% ============================================================
\end{document}