\documentclass[10pt]{beamer}

\title{}
\author{Artur's \( \oint \) Mathematikstübchen}
\date{}


% ===== Packages =========
\usepackage[utf8]{inputenc}

\usepackage[natbibapa]{apacite}
\bibliographystyle{apacite}
\usepackage[ngerman]{babel}
\usepackage{graphicx}
\usepackage{fancyhdr}
\usepackage{amsmath}
\usepackage{amssymb}
\usepackage{graphicx}
\usepackage{MnSymbol}
\usepackage{enumitem}
\usepackage{amsthm}
\usepackage{mleftright}
\usepackage{dsfont}
\usepackage{tikz-cd}


\def\bC{\mathbb{C}}
\def\bD{\mathbb{D}}               
\def\bE{\mathbb{E}}
\def\bG{\mathbb{G}}
\def\bN{\mathbb{N}}
\def\bP{\mathbb{P}}
\def\bQ{\mathbb{Q}}
\def\bR{\mathbb{R}}
\def\bBarR{\bar{\mathbb{R}}}
\def\bY{\mathbb{Y}}



\def\mA{\mathcal{A}}
\def\mB{\mathcal{B}}
\def\mD{\mathcal{D}}
\def\mE{\mathcal{E}}
\def\mF{\mathcal{F}}
\def\mG{\mathcal{G}}
\def\mH{\mathcal{H}}
\def\mL{\mathcal{L}}
\def\mN{\mathcal{N}}
\def\mP{\mathcal{P}}
\def\mS{\mathcal{S}}
\def\mT{\mathcal{T}}
\def\mX{\mathcal{X}}
\def\mY{\mathcal{Y}}



\usetheme{Madrid}



% ======================== Beginn Document ========================

\begin{document}





% ======================== Begrüßung ==================

\begin{frame}
    \begin{center}
        \textbf{\huge Willkommen in der guten Stube \newline \newline :D}
    \end{center}
\end{frame}
% =====================================================



% ======================== Präsentation ==================

\begin{frame}
    \begin{alertblock}{Aufgabe}
        Sei \( x > 0 \) eine beliebige positive reelle Zahl. Man zeige für alle \( n \in \bN \) die Gültigkeit der Abschätzung:
        \begin{align*}
            \frac{n}{x} + x^{n}
            \geq n + 1.
        \end{align*}
    \end{alertblock}
\end{frame}



\begin{frame}{Hilfsabschätzung}
    
\end{frame}



\begin{frame}{Hilfsabschätzung}
    \begin{block}{Bernoulli-Ungleichung}
        Für alle \( x \in \bR \) mit \( x \geq - 1 \) und alle \( n \in \bN_{0} \) gilt die Abschätzung:
        \begin{align*}
            \left( 1 + x \right)^{n}
            \geq 1 + n \cdot x.
        \end{align*}
    \end{block}
\end{frame}



\begin{frame}{Hilfsabschätzung}
    \begin{block}{Bernoulli-Ungleichung}
        Für alle \( x \in \bR \) mit \( x \geq - 1 \) und alle \( n \in \bN_{0} \) gilt die Abschätzung:
        \begin{align*}
            \left( 1 + x \right)^{n}
            \geq 1 + n \cdot x.
        \end{align*}
    \end{block}
    Weiter gilt für alle \( x > 0 \) die Abschätzung:
\end{frame}



\begin{frame}{Hilfsabschätzung}
    \begin{block}{Bermoulli-Ungleichung}
        Für alle \( x \in \bR \) mit \( x \geq - 1 \) und alle \( n \in \bN_{0} \) gilt die Abschätzung:
        \begin{align*}
            \left( 1 + x \right)^{n}
            \geq 1 + n \cdot x.
        \end{align*}
    \end{block}
    Weiter gilt für alle \( x > 0 \) die Abschätzung:
    \begin{align*}
        x + \frac{1}{x} 
        \geq 2.
    \end{align*}
\end{frame}



\begin{frame}{Beweis}
    
\end{frame}



\begin{frame}{Beweis}
    Sei \( x > 0 \) eine positive reelle Zahl und \( n \in \bN \) eine natürliche Zahl.
\end{frame}



\begin{frame}{Beweis}
    Sei \( x > 0 \) eine positive reelle Zahl und \( n \in \bN \) eine natürliche Zahl. Dann gilt:
\end{frame}



\begin{frame}{Beweis}
    Sei \( x > 0 \) eine positive reelle Zahl und \( n \in \bN \) eine natürliche Zahl. Dann gilt:
    \begin{align*}
        \frac{n}{x} + x^{n}
    \end{align*}
\end{frame}



\begin{frame}{Beweis}
    Sei \( x > 0 \) eine positive reelle Zahl und \( n \in \bN \) eine natürliche Zahl. Dann gilt:
    \begin{align*}
        \frac{n}{x} + x^{n}
        & = \frac{n}{x} + \left(1 + x - 1 \right)^{n}
    \end{align*}
\end{frame}



\begin{frame}{Beweis}
    Sei \( x > 0 \) eine positive reelle Zahl und \( n \in \bN \) eine natürliche Zahl. Dann gilt:
    \begin{align*}
        \frac{n}{x} + x^{n}
        & = \frac{n}{x} + \left(1 + x - 1 \right)^{n} \\
        & \geq \frac{n}{x} + 1 + n \cdot \left(x - 1 \right)
    \end{align*}
\end{frame}



\begin{frame}{Beweis}
    Sei \( x > 0 \) eine positive reelle Zahl und \( n \in \bN \) eine natürliche Zahl. Dann gilt:
    \begin{align*}
        \frac{n}{x} + x^{n}
        & = \frac{n}{x} + \left(1 + x - 1 \right)^{n} \\
        & \geq \frac{n}{x} + 1 + n \cdot \left(x - 1 \right) \\
        & = \frac{n}{x} + n \cdot x - n + 1
    \end{align*}
\end{frame}



\begin{frame}{Beweis}
    Sei \( x > 0 \) eine positive reelle Zahl und \( n \in \bN \) eine natürliche Zahl. Dann gilt:
    \begin{align*}
        \frac{n}{x} + x^{n}
        & = \frac{n}{x} + \left(1 + x - 1 \right)^{n} \\
        & \geq \frac{n}{x} + 1 + n \cdot \left(x - 1 \right) \\
        & = \frac{n}{x} + n \cdot x - n + 1 \\
        & = \left( \frac{1}{x} + x \right) \cdot n - n + 1
    \end{align*}
\end{frame}



\begin{frame}{Beweis}
    Sei \( x > 0 \) eine positive reelle Zahl und \( n \in \bN \) eine natürliche Zahl. Dann gilt:
    \begin{align*}
        \frac{n}{x} + x^{n}
        & = \frac{n}{x} + \left(1 + x - 1 \right)^{n} \\
        & \geq \frac{n}{x} + 1 + n \cdot \left(x - 1 \right) \\
        & = \frac{n}{x} + n \cdot x - n + 1 \\
        & = \left( \frac{1}{x} + x \right) \cdot n - n + 1 \\
        & \geq 2 \cdot n - n + 1
    \end{align*}
\end{frame}



\begin{frame}{Beweis}
    Sei \( x > 0 \) eine positive reelle Zahl und \( n \in \bN \) eine natürliche Zahl. Dann gilt:
    \begin{align*}
        \frac{n}{x} + x^{n}
        & = \frac{n}{x} + \left(1 + x - 1 \right)^{n} \\
        & \geq \frac{n}{x} + 1 + n \cdot \left(x - 1 \right) \\
        & = \frac{n}{x} + n \cdot x - n + 1 \\
        & = \left( \frac{1}{x} + x \right) \cdot n - n + 1 \\
        & \geq 2 \cdot n - n + 1 \\
        & = n + 1.
    \end{align*}
\end{frame}
% ============================================================
\end{document}