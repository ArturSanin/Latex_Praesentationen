\documentclass[10pt]{beamer}

\title{}
\author{Artur's \( \oint \) Mathematikstübchen}
\date{}


% ===== Packages =========
\usepackage[utf8]{inputenc}

\usepackage[natbibapa]{apacite}
\bibliographystyle{apacite}
\usepackage[ngerman]{babel}
\usepackage{graphicx}
\usepackage{fancyhdr}
\usepackage{amsmath}
\usepackage{amssymb}
\usepackage{graphicx}
\usepackage{MnSymbol}
\usepackage{enumitem}
\usepackage{amsthm}
\usepackage{mleftright}
\usepackage{dsfont}
\usepackage{tikz-cd}


\def\bC{\mathbb{C}}
\def\bD{\mathbb{D}}               
\def\bE{\mathbb{E}}
\def\bG{\mathbb{G}}
\def\bN{\mathbb{N}}
\def\bP{\mathbb{P}}
\def\bQ{\mathbb{Q}}
\def\bR{\mathbb{R}}
\def\bBarR{\bar{\mathbb{R}}}
\def\bY{\mathbb{Y}}



\def\mA{\mathcal{A}}
\def\mB{\mathcal{B}}
\def\mD{\mathcal{D}}
\def\mE{\mathcal{E}}
\def\mF{\mathcal{F}}
\def\mG{\mathcal{G}}
\def\mH{\mathcal{H}}
\def\mL{\mathcal{L}}
\def\mN{\mathcal{N}}
\def\mP{\mathcal{P}}
\def\mS{\mathcal{S}}
\def\mT{\mathcal{T}}
\def\mX{\mathcal{X}}
\def\mY{\mathcal{Y}}



\usetheme{Madrid}



% ======================== Beginn Document ========================

\begin{document}





% ======================== Begrüßung ==================

\begin{frame}
    \begin{center}
        \textbf{\huge Willkommen in der guten Stube \newline \newline :D}
    \end{center}
\end{frame}
% =====================================================



% ======================== Präsentation ==================

\begin{frame}
    \begin{alertblock}{Aufgabe}
        Sei \( x \in \bR \) eine beliebige reelle Zahl. Man zeige die Gültigkeit der Abschätzung:
        \begin{align*}
            e^{x}
            \geq 1 + x.
        \end{align*}
    \end{alertblock}
\end{frame}



\begin{frame}{Hilfsabschätzung}
    
\end{frame}



\begin{frame}{Hilfsabschätzung}
    \begin{block}{Bernoulli-Ungleichung}
        Für alle \( x \in \bR \) mit \( x \geq - 1 \) und alle \( n \in \bN_{0} \) gilt die Abschätzung:
        \begin{align*}
            \left( 1 + x \right)^{n}
            \geq 1 + n \cdot x.
        \end{align*}
    \end{block}
\end{frame}



\begin{frame}{Hilfsabschätzung}
    \begin{block}{Bernoulli-Ungleichung}
        Für alle \( x \in \bR \) mit \( x \geq - 1 \) und alle \( n \in \bN_{0} \) gilt die Abschätzung:
        \begin{align*}
            \left( 1 + x \right)^{n}
            \geq 1 + n \cdot x.
        \end{align*}
    \end{block}
    Weiter gilt für alle \( x \in \bR \) die folgende Grenzwert-Darstellung der Exponentialfunktion:
\end{frame}



\begin{frame}{Hilfsabschätzung}
    \begin{block}{Bernoulli-Ungleichung}
        Für alle \( x \in \bR \) mit \( x \geq - 1 \) und alle \( n \in \bN_{0} \) gilt die Abschätzung:
        \begin{align*}
            \left( 1 + x \right)^{n}
            \geq 1 + n \cdot x.
        \end{align*}
    \end{block}
    Weiter gilt für alle \( x \in \bR \) die folgende Grenzwert-Darstellung der Exponentialfunktion:
    \begin{align*}
        e^{x} 
        & = \lim_{n \to \infty} \left( 1 + \frac{x}{n} \right)^{n}.
    \end{align*}
\end{frame}



\begin{frame}{Beweis}
    
\end{frame}



\begin{frame}{Beweis}
    Sei \( x \in \bR \) eine beliebige reelle Zahl.
\end{frame}



\begin{frame}{Beweis}
    Sei \( x \in \bR \) eine beliebige reelle Zahl. Es gilt:
    \begin{align*}
        \lim_{n \to \infty} \frac{x}{n} = 0. 
    \end{align*}
\end{frame}



\begin{frame}{Beweis}
    Sei \( x \in \bR \) eine beliebige reelle Zahl. Es gilt:
    \begin{align*}
        \lim_{n \to \infty} \frac{x}{n} = 0. 
    \end{align*}
    Hieraus folgt die Existenz einer natürlichen Zahl \( n_{0}(x) \in \bN \), so dass für jedes \( n \in \bN \) mit \( n \geq n_{0}(x) \) gilt:
    \begin{align*}
        \frac{x}{n} 
        \geq - 1.
    \end{align*}
\end{frame}



\begin{frame}{Beweis}
    Sei \( x \in \bR \) eine beliebige reelle Zahl. Es gilt:
    \begin{align*}
        \lim_{n \to \infty} \frac{x}{n} = 0. 
    \end{align*}
    Hieraus folgt die Existenz einer natürlichen Zahl \( n_{0}(x) \in \bN \), so dass für jedes \( n \in \bN \) mit \( n \geq n_{0}(x) \) gilt:
    \begin{align*}
        \frac{x}{n} 
        \geq - 1.
    \end{align*}
    Für jedes solche \( n \) gilt zusammen mit der Bernoulli-Ungleichung:
\end{frame}



\begin{frame}{Beweis}
    Sei \( x \in \bR \) eine beliebige reelle Zahl. Es gilt:
    \begin{align*}
        \lim_{n \to \infty} \frac{x}{n} = 0. 
    \end{align*}
    Hieraus folgt die Existenz einer natürlichen Zahl \( n_{0}(x) \in \bN \), so dass für jedes \( n \in \bN \) mit \( n \geq n_{0}(x) \) gilt:
    \begin{align*}
        \frac{x}{n} 
        \geq - 1.
    \end{align*}
    Für jedes solche \( n \) gilt zusammen mit der Bernoulli-Ungleichung:
    \begin{align*}
        \left( 1 + \frac{x}{n} \right)^{n}
    \end{align*}
\end{frame}



\begin{frame}{Beweis}
    Sei \( x \in \bR \) eine beliebige reelle Zahl. Es gilt:
    \begin{align*}
        \lim_{n \to \infty} \frac{x}{n} = 0. 
    \end{align*}
    Hieraus folgt die Existenz einer natürlichen Zahl \( n_{0}(x) \in \bN \), so dass für jedes \( n \in \bN \) mit \( n \geq n_{0}(x) \) gilt:
    \begin{align*}
        \frac{x}{n} 
        \geq - 1.
    \end{align*}
    Für jedes solche \( n \) gilt zusammen mit der Bernoulli-Ungleichung:
    \begin{align*}
        \left( 1 + \frac{x}{n} \right)^{n}
        & \geq 1 + n \cdot \frac{x}{n}
    \end{align*}
\end{frame}



\begin{frame}{Beweis}
    Sei \( x \in \bR \) eine beliebige reelle Zahl. Es gilt:
    \begin{align*}
        \lim_{n \to \infty} \frac{x}{n} = 0. 
    \end{align*}
    Hieraus folgt die Existenz einer natürlichen Zahl \( n_{0}(x) \in \bN \), so dass für jedes \( n \in \bN \) mit \( n \geq n_{0}(x) \) gilt:
    \begin{align*}
        \frac{x}{n} 
        \geq - 1.
    \end{align*}
    Für jedes solche \( n \) gilt zusammen mit der Bernoulli-Ungleichung:
    \begin{align*}
        \left( 1 + \frac{x}{n} \right)^{n}
        & \geq 1 + n \cdot \frac{x}{n} \\
        & = 1 + x.
    \end{align*}
\end{frame}



\begin{frame}{Beweis}
    Aus der Betrachtung \( n \to \infty \) folgt:
\end{frame}



\begin{frame}{Beweis}
    Aus der Betrachtung \( n \to \infty \) folgt:
    \begin{align*}
        e^{x}
    \end{align*}
\end{frame}



\begin{frame}{Beweis}
    Aus der Betrachtung \( n \to \infty \) folgt:
    \begin{align*}
        e^{x}
        & = \lim_{n \to \infty} \left( 1 + \frac{x}{n} \right)^{n}
    \end{align*}
\end{frame}



\begin{frame}{Beweis}
    Aus der Betrachtung \( n \to \infty \) folgt:
    \begin{align*}
        e^{x}
        & = \lim_{n \to \infty} \left( 1 + \frac{x}{n} \right)^{n} \\
        & \geq \lim_{n \to \infty} \left( 1 + x \right)
    \end{align*}
\end{frame}



\begin{frame}{Beweis}
    Aus der Betrachtung \( n \to \infty \) folgt:
    \begin{align*}
        e^{x}
        & = \lim_{n \to \infty} \left( 1 + \frac{x}{n} \right)^{n} \\
        & \geq \lim_{n \to \infty} \left( 1 + x \right) \\
        & = 1 + x.
    \end{align*}
\end{frame}
% ============================================================
\end{document}