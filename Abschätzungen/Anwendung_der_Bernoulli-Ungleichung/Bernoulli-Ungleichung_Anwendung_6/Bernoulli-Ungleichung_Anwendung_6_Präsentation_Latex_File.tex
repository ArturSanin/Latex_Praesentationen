\documentclass[10pt]{beamer}

\title{}
\author{Artur's \( \oint \) Mathematikstübchen}
\date{}


% ===== Packages =========
\usepackage[utf8]{inputenc}

\usepackage[natbibapa]{apacite}
\bibliographystyle{apacite}
\usepackage[ngerman]{babel}
\usepackage{graphicx}
\usepackage{fancyhdr}
\usepackage{amsmath}
\usepackage{amssymb}
\usepackage{graphicx}
\usepackage{MnSymbol}
\usepackage{enumitem}
\usepackage{amsthm}
\usepackage{mleftright}
\usepackage{dsfont}
\usepackage{tikz-cd}


\def\bC{\mathbb{C}}
\def\bD{\mathbb{D}}               
\def\bE{\mathbb{E}}
\def\bG{\mathbb{G}}
\def\bN{\mathbb{N}}
\def\bP{\mathbb{P}}
\def\bQ{\mathbb{Q}}
\def\bR{\mathbb{R}}
\def\bBarR{\bar{\mathbb{R}}}
\def\bY{\mathbb{Y}}



\def\mA{\mathcal{A}}
\def\mB{\mathcal{B}}
\def\mD{\mathcal{D}}
\def\mE{\mathcal{E}}
\def\mF{\mathcal{F}}
\def\mG{\mathcal{G}}
\def\mH{\mathcal{H}}
\def\mL{\mathcal{L}}
\def\mN{\mathcal{N}}
\def\mP{\mathcal{P}}
\def\mS{\mathcal{S}}
\def\mT{\mathcal{T}}
\def\mX{\mathcal{X}}
\def\mY{\mathcal{Y}}



\usetheme{Madrid}



% ======================== Beginn Document ========================

\begin{document}





% ======================== Begrüßung ==================

\begin{frame}
    \begin{center}
        \textbf{\huge Willkommen in der guten Stube \newline \newline :D}
    \end{center}
\end{frame}
% =====================================================



% ======================== Präsentation ==================

\begin{frame}
    \begin{alertblock}{Aufgabe}
        Seien \( x_{1}, \ldots, x_{n}, x_{n + 1} > 0 \), \( n \in \bN \), positive reelle Zahlen und \( \bar{x}_{n} \coloneq \frac{1}{n} \sum_{k = 1}^{n} x_{k} \). Man zeige die Gültigkeit der Abschätzung:
        \begin{align*}
            \left( \frac{\sum_{k = 1}^{n + 1} x_{k}}{\left( n + 1 \right) \cdot \bar{x}_{n}} \right)^{n + 1}
            \geq \frac{x_{n + 1}}{\bar{x}_{n}}.
        \end{align*}
    \end{alertblock}
\end{frame}



\begin{frame}{Hilfsabschätzung}
    
\end{frame}



\begin{frame}{Hilfsabschätzung}
    \begin{block}{Bernoulli-Ungleichung}
        Für alle \( x \in \bR \) mit \( x \geq - 1 \) und alle \( n \in \bN_{0} \) gilt die Abschätzung:
        \begin{align*}
            \left( 1 + x \right)^{n}
            \geq 1 + n \cdot x.
        \end{align*}
    \end{block}
\end{frame}



\begin{frame}{Beweis}
    
\end{frame}



\begin{frame}{Beweis}
    Für \( n \in \bN \) seien \( x_{1}, \ldots, x_{n}, x_{n + 1} > 0 \).
\end{frame}



\begin{frame}{Beweis}
    Für \( n \in \bN \) seien \( x_{1}, \ldots, x_{n}, x_{n + 1} > 0 \). Dann gilt:
\end{frame}



\begin{frame}{Beweis}
    Für \( n \in \bN \) seien \( x_{1}, \ldots, x_{n}, x_{n + 1} > 0 \). Dann gilt:
    \begin{align*}
        \left( \frac{\sum_{k = 1}^{n + 1} x_{k}}{\left( n + 1 \right) \cdot \bar{x}_{n}} \right)^{n + 1}
    \end{align*}
\end{frame}



\begin{frame}{Beweis}
    Für \( n \in \bN \) seien \( x_{1}, \ldots, x_{n}, x_{n + 1} > 0 \). Dann gilt:
    \begin{align*}
        \left( \frac{\sum_{k = 1}^{n + 1} x_{k}}{\left( n + 1 \right) \cdot \bar{x}_{n}} \right)^{n + 1}
        & = \left( 1 + \frac{\sum_{k = 1}^{n + 1} x_{k}}{\left( n + 1 \right) \cdot \bar{x}_{n}} - 1 \right)^{n + 1}
    \end{align*}
\end{frame}



\begin{frame}{Beweis}
    Für \( n \in \bN \) seien \( x_{1}, \ldots, x_{n}, x_{n + 1} > 0 \). Dann gilt:
    \begin{align*}
        \left( \frac{\sum_{k = 1}^{n + 1} x_{k}}{\left( n + 1 \right) \cdot \bar{x}_{n}} \right)^{n + 1}
        & = \left( 1 + \frac{\sum_{k = 1}^{n + 1} x_{k}}{\left( n + 1 \right) \cdot \bar{x}_{n}} - 1 \right)^{n + 1} \\
        & \geq 1 + \left( n + 1 \right) \cdot \left( \frac{\sum_{k = 1}^{n + 1} x_{k}}{\left( n + 1 \right) \cdot \bar{x}_{n}} - 1 \right)
    \end{align*}
\end{frame}



\begin{frame}{Beweis}
    Für \( n \in \bN \) seien \( x_{1}, \ldots, x_{n}, x_{n + 1} > 0 \). Dann gilt:
    \begin{align*}
        \left( \frac{\sum_{k = 1}^{n + 1} x_{k}}{\left( n + 1 \right) \cdot \bar{x}_{n}} \right)^{n + 1}
        & = \left( 1 + \frac{\sum_{k = 1}^{n + 1} x_{k}}{\left( n + 1 \right) \cdot \bar{x}_{n}} - 1 \right)^{n + 1} \\
        & \geq 1 + \left( n + 1 \right) \cdot \left( \frac{\sum_{k = 1}^{n + 1} x_{k}}{\left( n + 1 \right) \cdot \bar{x}_{n}} - 1 \right) \\
        & = 1 + \left( n + 1 \right) \cdot \frac{\sum_{k = 1}^{n + 1} x_{k}}{\left( n + 1 \right) \cdot \bar{x}_{n}} - \left( n + 1 \right)
    \end{align*}
\end{frame}



\begin{frame}{Beweis}
    Für \( n \in \bN \) seien \( x_{1}, \ldots, x_{n}, x_{n + 1} > 0 \). Dann gilt:
    \begin{align*}
        \left( \frac{\sum_{k = 1}^{n + 1} x_{k}}{\left( n + 1 \right) \cdot \bar{x}_{n}} \right)^{n + 1}
        & = \left( 1 + \frac{\sum_{k = 1}^{n + 1} x_{k}}{\left( n + 1 \right) \cdot \bar{x}_{n}} - 1 \right)^{n + 1} \\
        & \geq 1 + \left( n + 1 \right) \cdot \left( \frac{\sum_{k = 1}^{n + 1} x_{k}}{\left( n + 1 \right) \cdot \bar{x}_{n}} - 1 \right) \\
        & = 1 + \left( n + 1 \right) \cdot \frac{\sum_{k = 1}^{n + 1} x_{k}}{\left( n + 1 \right) \cdot \bar{x}_{n}} - \left( n + 1 \right) \\
        & = 1 + \frac{\sum_{k = 1}^{n + 1} x_{k}}{\bar{x}_{n}} - n - 1
    \end{align*}
\end{frame}



\begin{frame}{Beweis}
    Für \( n \in \bN \) seien \( x_{1}, \ldots, x_{n}, x_{n + 1} > 0 \). Dann gilt:
    \begin{align*}
        \left( \frac{\sum_{k = 1}^{n + 1} x_{k}}{\left( n + 1 \right) \cdot \bar{x}_{n}} \right)^{n + 1}
        & = \left( 1 + \frac{\sum_{k = 1}^{n + 1} x_{k}}{\left( n + 1 \right) \cdot \bar{x}_{n}} - 1 \right)^{n + 1} \\
        & \geq 1 + \left( n + 1 \right) \cdot \left( \frac{\sum_{k = 1}^{n + 1} x_{k}}{\left( n + 1 \right) \cdot \bar{x}_{n}} - 1 \right) \\
        & = 1 + \left( n + 1 \right) \cdot \frac{\sum_{k = 1}^{n + 1} x_{k}}{\left( n + 1 \right) \cdot \bar{x}_{n}} - \left( n + 1 \right) \\
        & = 1 + \frac{\sum_{k = 1}^{n + 1} x_{k}}{\bar{x}_{n}} - n - 1 \\
        & = \frac{\sum_{k = 1}^{n + 1} x_{k}}{\bar{x}_{n}} - n
    \end{align*}
\end{frame}



\begin{frame}{Beweis}
    Für \( n \in \bN \) seien \( x_{1}, \ldots, x_{n}, x_{n + 1} > 0 \). Dann gilt:
    \begin{align*}
        \left( \frac{\sum_{k = 1}^{n + 1} x_{k}}{\left( n + 1 \right) \cdot \bar{x}_{n}} \right)^{n + 1}
        & = \left( 1 + \frac{\sum_{k = 1}^{n + 1} x_{k}}{\left( n + 1 \right) \cdot \bar{x}_{n}} - 1 \right)^{n + 1} \\
        & \geq 1 + \left( n + 1 \right) \cdot \left( \frac{\sum_{k = 1}^{n + 1} x_{k}}{\left( n + 1 \right) \cdot \bar{x}_{n}} - 1 \right) \\
        & = 1 + \left( n + 1 \right) \cdot \frac{\sum_{k = 1}^{n + 1} x_{k}}{\left( n + 1 \right) \cdot \bar{x}_{n}} - \left( n + 1 \right) \\
        & = 1 + \frac{\sum_{k = 1}^{n + 1} x_{k}}{\bar{x}_{n}} - n - 1 \\
        & = \frac{\sum_{k = 1}^{n + 1} x_{k}}{\bar{x}_{n}} - n \\
        & = \frac{\sum_{k = 1}^{n + 1} x_{k} - n \cdot \bar{x}_{n}}{\bar{x}_{n}}
    \end{align*}
\end{frame}



\begin{frame}{Beweis}
    \begin{align*}
         & = \frac{\sum_{k = 1}^{n + 1} x_{k} - n \cdot \bar{x}_{n}}{\bar{x}_{n}}
    \end{align*}
\end{frame}



\begin{frame}{Beweis}
    \begin{align*}
         & = \frac{\sum_{k = 1}^{n + 1} x_{k} - n \cdot \bar{x}_{n}}{\bar{x}_{n}} \\
         & = \frac{\sum_{k = 1}^{n + 1} x_{k} - n \cdot \frac{1}{n} \sum_{k = 1}^{n} x_{k}}{\bar{x}_{n}}
    \end{align*}
\end{frame}



\begin{frame}{Beweis}
    \begin{align*}
         & = \frac{\sum_{k = 1}^{n + 1} x_{k} - n \cdot \bar{x}_{n}}{\bar{x}_{n}} \\
         & = \frac{\sum_{k = 1}^{n + 1} x_{k} - n \cdot \frac{1}{n} \sum_{k = 1}^{n} x_{k}}{\bar{x}_{n}} \\
         & = \frac{\sum_{k = 1}^{n + 1} x_{k} - \sum_{k = 1}^{n} x_{k}}{\bar{x}_{n}}
    \end{align*}
\end{frame}



\begin{frame}{Beweis}
    \begin{align*}
         & = \frac{\sum_{k = 1}^{n + 1} x_{k} - n \cdot \bar{x}_{n}}{\bar{x}_{n}} \\
         & = \frac{\sum_{k = 1}^{n + 1} x_{k} - n \cdot \frac{1}{n} \sum_{k = 1}^{n} x_{k}}{\bar{x}_{n}} \\
         & = \frac{\sum_{k = 1}^{n + 1} x_{k} - \sum_{k = 1}^{n} x_{k}}{\bar{x}_{n}} \\
         & = \frac{x_{n + 1}}{\bar{x}_{n}}.
    \end{align*}
\end{frame}
% ============================================================
\end{document}