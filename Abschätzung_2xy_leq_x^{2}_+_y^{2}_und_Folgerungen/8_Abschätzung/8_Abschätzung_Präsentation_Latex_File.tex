\documentclass[10pt]{beamer}

\title{}
\author{Artur's \( \oint \) Mathematikstübchen}
\date{}


% ===== Packages =========
\usepackage[utf8]{inputenc}

\usepackage[natbibapa]{apacite}
\bibliographystyle{apacite}
\usepackage[ngerman]{babel}
\usepackage{graphicx}
\usepackage{fancyhdr}
\usepackage{amsmath}
\usepackage{amssymb}
\usepackage{graphicx}
\usepackage{MnSymbol}
\usepackage{enumitem}
\usepackage{amsthm}
\usepackage{mleftright}
\usepackage{dsfont}
\usepackage{tikz-cd}


\def\bC{\mathbb{C}}
\def\bD{\mathbb{D}}               
\def\bE{\mathbb{E}}
\def\bG{\mathbb{G}}
\def\bN{\mathbb{N}}
\def\bP{\mathbb{P}}
\def\bQ{\mathbb{Q}}
\def\bR{\mathbb{R}}
\def\bBarR{\bar{\mathbb{R}}}
\def\bY{\mathbb{Y}}



\def\mA{\mathcal{A}}
\def\mB{\mathcal{B}}
\def\mD{\mathcal{D}}
\def\mE{\mathcal{E}}
\def\mF{\mathcal{F}}
\def\mG{\mathcal{G}}
\def\mH{\mathcal{H}}
\def\mL{\mathcal{L}}
\def\mN{\mathcal{N}}
\def\mP{\mathcal{P}}
\def\mS{\mathcal{S}}
\def\mT{\mathcal{T}}
\def\mX{\mathcal{X}}
\def\mY{\mathcal{Y}}



\usetheme{Madrid}



% ======================== Beginn Document ========================

\begin{document}





% ======================== Begrüßung ==================

\begin{frame}
    \begin{center}
        \textbf{\huge Willkommen in der guten Stube \newline \newline :D}
    \end{center}
\end{frame}
% =====================================================



% ======================== Präsentation ==================

\begin{frame}
    \begin{alertblock}{Aufgabe}
        Seien \( x_{1}, x_{2}, \ldots, x_{n} \in \bR \), \( n \in \bN \), beliebige reelle Zahlen. Man zeige die Gültigkeit der folgenden Abschätzung:
        \begin{align*}
            \sum_{k = 1}^{n} x_{k}x_{n + 1 - k}
            \leq \sum_{k = 1}^{n} x_{k}^{2}.
        \end{align*}
    \end{alertblock}
\end{frame}



\begin{frame}{Hilfsabschätzung}
    
\end{frame}



\begin{frame}{Hilfsabschätzung}
    \begin{block}{Abschätzung}
        Für alle \( x, y \in \bR \) gilt:
        \begin{align*}
            2xy 
            \leq x^{2} + y^{2}.
        \end{align*}
    \end{block}
\end{frame}



\begin{frame}{Hilfsabschätzung}
    \begin{block}{Abschätzung}
        Für alle \( x, y \in \bR \) gilt:
        \begin{align*}
            2xy 
            \leq x^{2} + y^{2}.
        \end{align*}
    \end{block}
    Daraus folgt die Abschätzung:
    \begin{align*}
        xy
        \leq \frac{x^{2} + y^{2}}{2}.
    \end{align*}
\end{frame}



\begin{frame}{Beweis}
    
\end{frame}



\begin{frame}{Beweis}
    Seien \( x_{1}, x_{2}, \ldots, x_{n} \in \bR \) beliebige reelle Zahlen.
\end{frame}



\begin{frame}{Beweis}
    Seien \( x_{1}, x_{2}, \ldots, x_{n} \in \bR \) beliebige reelle Zahlen. Dann folgt zusammen mit der Hilfsabschätzung:
\end{frame}



\begin{frame}{Beweis}
    Seien \( x_{1}, x_{2}, \ldots, x_{n} \in \bR \) beliebige reelle Zahlen. Dann folgt zusammen mit der Hilfsabschätzung:
    \begin{align*}
        \sum_{k = 1}^{n} x_{k}x_{n + 1 - k}
    \end{align*}
\end{frame}



\begin{frame}{Beweis}
    Seien \( x_{1}, x_{2}, \ldots, x_{n} \in \bR \) beliebige reelle Zahlen. Dann folgt zusammen mit der Hilfsabschätzung:
    \begin{align*}
        \sum_{k = 1}^{n} x_{k}x_{n + 1 - k}
        & \leq \sum_{k = 1}^{n} \frac{x_{k}^{2} + x_{n + 1 - k}^{2}}{2}
    \end{align*}
\end{frame}



\begin{frame}{Beweis}
    Seien \( x_{1}, x_{2}, \ldots, x_{n} \in \bR \) beliebige reelle Zahlen. Dann folgt zusammen mit der Hilfsabschätzung:
    \begin{align*}
        \sum_{k = 1}^{n} x_{k}x_{n + 1 - k}
        & \leq \sum_{k = 1}^{n} \frac{x_{k}^{2} + x_{n + 1 - k}^{2}}{2} \\
        & = \frac{1}{2} \cdot \sum_{k = 1}^{n} x_{k}^{2} + \frac{1}{2} \cdot \sum_{k = 1}^{n} x_{n + 1 - k}^{2}
    \end{align*}
\end{frame}



\begin{frame}{Beweis}
    Seien \( x_{1}, x_{2}, \ldots, x_{n} \in \bR \) beliebige reelle Zahlen. Dann folgt zusammen mit der Hilfsabschätzung:
    \begin{align*}
        \sum_{k = 1}^{n} x_{k}x_{n + 1 - k}
        & \leq \sum_{k = 1}^{n} \frac{x_{k}^{2} + x_{n + 1 - k}^{2}}{2} \\
        & = \frac{1}{2} \cdot \sum_{k = 1}^{n} x_{k}^{2} + \frac{1}{2} \cdot \sum_{k = 1}^{n} x_{n + 1 - k}^{2} \\
        & = \frac{1}{2} \cdot \sum_{k = 1}^{n} x_{k}^{2} + \frac{1}{2} \cdot \sum_{k = 1}^{n} x_{k}^{2} 
    \end{align*}
\end{frame}



\begin{frame}{Beweis}
    Seien \( x_{1}, x_{2}, \ldots, x_{n} \in \bR \) beliebige reelle Zahlen. Dann folgt zusammen mit der Hilfsabschätzung:
    \begin{align*}
        \sum_{k = 1}^{n} x_{k}x_{n + 1 - k}
        & \leq \sum_{k = 1}^{n} \frac{x_{k}^{2} + x_{n + 1 - k}^{2}}{2} \\
        & = \frac{1}{2} \cdot \sum_{k = 1}^{n} x_{k}^{2} + \frac{1}{2} \cdot \sum_{k = 1}^{n} x_{n + 1 - k}^{2} \\
        & = \frac{1}{2} \cdot \sum_{k = 1}^{n} x_{k}^{2} + \frac{1}{2} \cdot \sum_{k = 1}^{n} x_{k}^{2} \\
        & = \sum_{k = 1}^{n} x_{k}^{2}.
    \end{align*}
\end{frame}
% ============================================================
\end{document}