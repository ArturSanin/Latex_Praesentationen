\documentclass[10pt]{beamer}

\title{}
\author{Artur's Mathematikstübchen}
\date{}


% ===== Packages =========
\usepackage[utf8]{inputenc}

\usepackage[natbibapa]{apacite}
\bibliographystyle{apacite}
\usepackage[ngerman]{babel}
\usepackage{graphicx}
\usepackage{fancyhdr}
\usepackage{amsmath}
\usepackage{amssymb}
\usepackage{graphicx}
\usepackage{MnSymbol}
\usepackage{tikz-cd}
\usepackage{enumitem}
\usepackage{amsthm}
\usepackage{mleftright}
\usepackage{dsfont}



\def\bD{\mathbb{D}}               
\def\bE{\mathbb{E}}
\def\bG{\mathbb{G}}
\def\bN{\mathbb{N}}
\def\bP{\mathbb{P}}
\def\bQ{\mathbb{Q}}
\def\bR{\mathbb{R}}
\def\bBarR{\bar{\mathbb{R}}}
\def\bY{\mathbb{Y}}



\def\mA{\mathcal{A}}
\def\mB{\mathcal{B}}
\def\mD{\mathcal{D}}
\def\mE{\mathcal{E}}
\def\mF{\mathcal{F}}
\def\mG{\mathcal{G}}
\def\mH{\mathcal{H}}
\def\mL{\mathcal{L}}
\def\mN{\mathcal{N}}
\def\mP{\mathcal{P}}
\def\mS{\mathcal{S}}
\def\mT{\mathcal{T}}
\def\mX{\mathcal{X}}
\def\mY{\mathcal{Y}}



\usetheme{Madrid}



% ======================== Beginn Document ========================

\begin{document}





% ======================== Begrüßung ==================

\begin{frame}
    \begin{center}
        \textbf{\huge Willkommen in der guten Stube \newline \newline :D}
    \end{center}
\end{frame}
% =====================================================



% ======================== Präsentation ==================

\begin{frame}
    \begin{alertblock}{Aufgabe}
        Für alle \( x, y \in \bR \) zeige man die Abschätzung:
        \begin{align*}
            2xy
            \leq x^{2} + y^{2}.
        \end{align*}
    \end{alertblock}
\end{frame}



\begin{frame}{Hilfsmittel}
    \begin{itemize}
        \item<1-> 1. Für alle \( p \in \bR \) ist \( p^{2} \geq 0 \).
        \item<2-> 2. Für alle \( p, q \in \bR \) ist \( \left( p - q \right)^{2}
            = p^{2} - 2pq + q^{2}. \)
    \end{itemize}
\end{frame}



\begin{frame}{Beweis: Erster Ansatz}
    Seien \( x, y \in \bR \) zwei beliebige reelle Zahlen.
\end{frame}



\begin{frame}{Beweis: Erster Ansatz}
    Seien \( x, y \in \bR \) zwei beliebige reelle Zahlen. Dann ist:
    \begin{align*}
        x^{2} - 2xy + y^{2}
    \end{align*}
\end{frame}



\begin{frame}{Beweis: Erster Ansatz}
    Seien \( x, y \in \bR \) zwei beliebige reelle Zahlen. Dann ist:
    \begin{align*}
        x^{2} - 2xy + y^{2}
        = \left( x - y \right)^{2}  
    \end{align*}
\end{frame}



\begin{frame}{Beweis: Erster Ansatz}
    Seien \( x, y \in \bR \) zwei beliebige reelle Zahlen. Dann ist:
    \begin{align*}
        x^{2} - 2xy + y^{2}
        = \left( x - y \right)^{2}
        \geq 0  
    \end{align*}
\end{frame}



\begin{frame}{Beweis: Erster Ansatz}
    Seien \( x, y \in \bR \) zwei beliebige reelle Zahlen. Dann ist:
    \begin{align*}
        x^{2} - 2xy + y^{2}
        = \left( x - y \right)^{2}
        \geq 0
        \quad \Leftrightarrow \quad x^{2} + y^{2} \geq 2xy.
    \end{align*}
\end{frame}



\begin{frame}{Beweis: Zweiter Ansatz}
    Seien \( x, y \in \bR \) zwei beliebige reelle Zahlen.
\end{frame}



\begin{frame}{Beweis: Zweiter Ansatz}
    Seien \( x, y \in \bR \) zwei beliebige reelle Zahlen. Dann gilt:
\end{frame}



\begin{frame}{Beweis: Zweiter Ansatz}
    Seien \( x, y \in \bR \) zwei beliebige reelle Zahlen. Dann gilt:
    \begin{align*}
        2xy
    \end{align*}
\end{frame}



\begin{frame}{Beweis: Zweiter Ansatz}
    Seien \( x, y \in \bR \) zwei beliebige reelle Zahlen. Dann gilt:
    \begin{align*}
        2xy
        & = x^{2} - x^{2} + 2xy \\
    \end{align*}
\end{frame}



\begin{frame}{Beweis: Zweiter Ansatz}
    Seien \( x, y \in \bR \) zwei beliebige reelle Zahlen. Dann gilt:
    \begin{align*}
        2xy
        & = x^{2} - x^{2} + 2xy \\
        & = x^{2} - x^{2} + 2xy + y^{2} - y^{2} \\
    \end{align*}
\end{frame}



\begin{frame}{Beweis: Zweiter Ansatz}
    Seien \( x, y \in \bR \) zwei beliebige reelle Zahlen. Dann gilt:
    \begin{align*}
        2xy
        & = x^{2} - x^{2} + 2xy \\
        & = x^{2} - x^{2} + 2xy + y^{2} - y^{2} \\
        & = x^{2} + y^{2} - \left( x^{2} - 2xy + y^{2} \right) \\ 
    \end{align*}
\end{frame}



\begin{frame}{Beweis: Zweiter Ansatz}
    Seien \( x, y \in \bR \) zwei beliebige reelle Zahlen. Dann gilt:
    \begin{align*}
        2xy
        & = x^{2} - x^{2} + 2xy \\
        & = x^{2} - x^{2} + 2xy + y^{2} - y^{2} \\
        & = x^{2} + y^{2} - \left( x^{2} - 2xy + y^{2} \right) \\ 
        & = x^{2} + y^{2} - \left( x - y \right)^{2} \\ 
    \end{align*}
\end{frame}



\begin{frame}{Beweis: Zweiter Ansatz}
    Seien \( x, y \in \bR \) zwei beliebige reelle Zahlen. Dann gilt:
    \begin{align*}
        2xy
        & = x^{2} - x^{2} + 2xy \\
        & = x^{2} - x^{2} + 2xy + y^{2} - y^{2} \\
        & = x^{2} + y^{2} - \left( x^{2} - 2xy + y^{2} \right) \\ 
        & = x^{2} + y^{2} - \underbrace{\left( x - y \right)^{2}}_{\text{\( \geq 0 \)}} \\ 
    \end{align*}
\end{frame}



\begin{frame}{Beweis: Zweiter Ansatz}
    Seien \( x, y \in \bR \) zwei beliebige reelle Zahlen. Dann gilt:
    \begin{align*}
        2xy
        & = x^{2} - x^{2} + 2xy \\
        & = x^{2} - x^{2} + 2xy + y^{2} - y^{2} \\
        & = x^{2} + y^{2} - \left( x^{2} - 2xy + y^{2} \right) \\ 
        & = x^{2} + y^{2} - \underbrace{\left( x - y \right)^{2}}_{\text{\( \geq 0 \)}} \\
        & \leq x^{2} + y^{2}.
    \end{align*}
\end{frame}



\begin{frame}{Folgerung 1}
    Aus der eben gezeigten Abschätzung folgt direkt: 
\end{frame}



\begin{frame}{Folgerung 1}
    Aus der eben gezeigten Abschätzung folgt direkt: Für alle \( x, y \in \bR \) gilt:
    \begin{align*}
        xy
        \leq \frac{x^{2} + y^{2}}{2}.
    \end{align*}
\end{frame}



\begin{frame}{Geometrische Interpretation der Abschätzung}
    
\end{frame}



\begin{frame}{Geometrische Interpretation der Abschätzung}
    \begin{center}
        \begin{tikzpicture}
            \draw(0, 0) rectangle (3, 1);
        \end{tikzpicture}
    \end{center}
\end{frame}



\begin{frame}{Geometrische Interpretation der Abschätzung}
    \begin{center}
        \begin{tikzpicture}
            \draw(0, 0) rectangle (3, 1);
            \node[below] at (1.5, 0) { x };
        \end{tikzpicture}
    \end{center}
\end{frame}



\begin{frame}{Geometrische Interpretation der Abschätzung}
    \begin{center}
        \begin{tikzpicture}
            \draw(0, 0) rectangle (3, 1);
            \node[below] at (1.5, 0) { x };
            \node[right] at (3, 0.5) { y };
        \end{tikzpicture}
    \end{center}
\end{frame}



\begin{frame}{Geometrische Interpretation der Abschätzung}
    \begin{center}
        \begin{tikzpicture}
            \draw(0, 0) rectangle (3, 1);
            \node[below] at (1.5, 0) { \( x \) };
            \node[right] at (3, 0.5) { \( y \) };
            \node at (1.5, 0.5) { \( xy \) };
        \end{tikzpicture}
    \end{center}
\end{frame}



\begin{frame}{Geometrische Interpretation der Abschätzung}
    \begin{center}
        \begin{tikzpicture}
            \draw(0, 0) rectangle (3, 1);
            \node[below] at (1.5, 0) { \( x \) };
            \node[right] at (3, 0.5) { \( y \) };
            \node at (1.5, 0.5) { \( xy \) };
            
            \draw(0, 0) -- (0, -3) -- (3, 0);
            \node[left] at (0, -1.5) { \( x \) };
        \end{tikzpicture}
    \end{center}
\end{frame}



\begin{frame}{Geometrische Interpretation der Abschätzung}
    \begin{center}
        \begin{tikzpicture}
            \draw(0, 0) rectangle (3, 1);
            \node[below] at (1.5, 0) { \( x \) };
            \node[right] at (3, 0.5) { \( y \) };
            \node at (1.5, 0.5) { \( xy \) };
            
            \draw(0, 0) -- (0, -3) -- (3, 0);
            \node[left] at (0, -1.5) { \( x \) };
            \node at (1, -1) {\( \frac{1}{2}x^{2} \)};
        \end{tikzpicture}
    \end{center}
\end{frame}



\begin{frame}{Geometrische Interpretation der Abschätzung}
    \begin{center}
        \begin{tikzpicture}
            \draw(0, 0) rectangle (3, 1);
            \node[below] at (1.5, 0) { \( x \) };
            \node[left] at (3, 0.5) { \( y \) };
            \node at (1.5, 0.5) { \( xy \) };
            
            \draw(0, 0) -- (0, -3) -- (3, 0);
            \node[left] at (0, -1.5) { \( x \) };
            \node at (1, -1) {\( \frac{1}{2}x^{2} \)};

            \draw(3, 0) -- (4, 0) -- (3, 1);
            \node[below] at (3.5, 0) {\( y \)};
        \end{tikzpicture}
    \end{center}
\end{frame}



\begin{frame}{Geometrische Interpretation der Abschätzung}
    \begin{center}
        \begin{tikzpicture}
            \draw(0, 0) rectangle (3, 1);
            \node[below] at (1.5, 0) { \( x \) };
            \node[left] at (3, 0.5) { \( y \) };
            \node at (1.5, 0.5) { \( xy \) };
            
            \draw(0, 0) -- (0, -3) -- (3, 0);
            \node[left] at (0, -1.5) { \( x \) };
            \node at (1, -1) {\( \frac{1}{2}x^{2} \)};

            \draw(3, 0) -- (4, 0) -- (3, 1);
            \node[below] at (3.5, 0) {\( y \)};
            \node[node font = \tiny] at (3.3, 0.3) {\( \frac{1}{2}y^{2} \)};
        \end{tikzpicture}
    \end{center}
\end{frame}



\begin{frame}{Geometrische Interpretation der Abschätzung}
    
\end{frame}



\begin{frame}{Geometrische Interpretation der Abschätzung}
    \begin{tikzpicture}
        \draw[blue!40!white, fill](0, 0) rectangle (3, 1);
        \node at (1.5, 0.5) { \( xy \) };
    \end{tikzpicture}
\end{frame}



\begin{frame}{Geometrische Interpretation der Abschätzung}
    \begin{tikzpicture}
        \draw[blue!40!white, fill](0, 0) rectangle (3, 1);
        \node at (1.5, 0.5) { \( xy \) };
        \node[node font = \Huge] at (4, 0.5)  {\( \leq \)};
    \end{tikzpicture}
\end{frame}



\begin{frame}{Geometrische Interpretation der Abschätzung}
    \begin{tikzpicture}
        \draw[blue!40!white, fill](0, 0) rectangle (3, 1);
        \node at (1.5, 0.5) { \( xy \) };
        \node[node font = \Huge] at (4, 0.5)  {\( \leq \)};

        \draw[blue!40!white, fill](5, 0) -- (5, 3) -- (8, 0);
        \node at (6, 1.0) { \( \frac{1}{2}x^{2} \) };
    \end{tikzpicture}
\end{frame}



\begin{frame}{Geometrische Interpretation der Abschätzung}
    \begin{tikzpicture}
        \draw[blue!40!white, fill](0, 0) rectangle (3, 1);
        \node at (1.5, 0.5) { \( xy \) };
        \node[node font = \Huge] at (4, 0.5)  {\( \leq \)};

        \draw[blue!40!white, fill](5, 0) -- (5, 3) -- (8, 0);
        \node at (6, 1.0) { \( \frac{1}{2}x^{2} \) };

        \node[node font = \Huge] at (9, 0.5)  {\( + \)};
    \end{tikzpicture}
\end{frame}



\begin{frame}{Geometrische Interpretation der Abschätzung}
    \begin{tikzpicture}
        \draw[blue!40!white, fill](0, 0) rectangle (3, 1);
        \node at (1.5, 0.5) { \( xy \) };
        \node[node font = \Huge] at (4, 0.5)  {\( \leq \)};

        \draw[blue!40!white, fill](5, 0) -- (5, 3) -- (8, 0);
        \node at (6, 1.0) { \( \frac{1}{2}x^{2} \) };

        \node[node font = \Huge] at (9, 0.5)  {\( + \)};

        \draw[blue!40!white, fill](10, 0) -- (10, 1) -- (11, 0);
        \node[node font = \tiny] at (10.3, 0.3) { \( \frac{1}{2}y^{2} \) };
    \end{tikzpicture}
\end{frame}



\begin{frame}{Folgerung 2}
    Wählen wir \( x = \sqrt{\Tilde{x}} \) und \( y = \sqrt{\Tilde{y}} \), wobei \( \Tilde{x}, \Tilde{y} \in \bR_{\geq 0} \), so erhalten wir die Abschätzung: 
\end{frame}



\begin{frame}{Folgerung 2}
    Wählen wir \( x = \sqrt{\Tilde{x}} \) und \( y = \sqrt{\Tilde{y}} \), wobei \( \Tilde{x}, \Tilde{y} \in \bR_{\geq 0} \), so erhalten wir die Abschätzung:
    \begin{align*}
        \sqrt{\Tilde{x} \cdot \Tilde{y}}
    \end{align*}
\end{frame}



\begin{frame}{Folgerung 2}
    Wählen wir \( x = \sqrt{\Tilde{x}} \) und \( y = \sqrt{\Tilde{y}} \), wobei \( \Tilde{x}, \Tilde{y} \in \bR_{\geq 0} \), so erhalten wir die Abschätzung:
    \begin{align*}
        \sqrt{\Tilde{x} \cdot \Tilde{y}}
        & = x \cdot y \\
    \end{align*}
\end{frame}



\begin{frame}{Folgerung 2}
    Wählen wir \( x = \sqrt{\Tilde{x}} \) und \( y = \sqrt{\Tilde{y}} \), wobei \( \Tilde{x}, \Tilde{y} \in \bR_{\geq 0} \), so erhalten wir die Abschätzung:
    \begin{align*}
        \sqrt{\Tilde{x} \cdot \Tilde{y}}
        & = x \cdot y \\
        & \leq \frac{x^{2} + y^{2}}{2} \\
    \end{align*}
\end{frame}



\begin{frame}{Folgerung 2}
    Wählen wir \( x = \sqrt{\Tilde{x}} \) und \( y = \sqrt{\Tilde{y}} \), wobei \( \Tilde{x}, \Tilde{y} \in \bR_{\geq 0} \), so erhalten wir die Abschätzung:
    \begin{align*}
        \sqrt{\Tilde{x} \cdot \Tilde{y}}
        & = x \cdot y \\
        & \leq \frac{x^{2} + y^{2}}{2} \\
        & = \frac{\left( \sqrt{\Tilde{x}} \right)^{2} + \left( \sqrt{\Tilde{y}} \right)^{2}}{2} \\
    \end{align*}
\end{frame}



\begin{frame}{Folgerung 2}
    Wählen wir \( x = \sqrt{\Tilde{x}} \) und \( y = \sqrt{\Tilde{y}} \), wobei \( \Tilde{x}, \Tilde{y} \in \bR_{\geq 0} \), so erhalten wir die Abschätzung:
    \begin{align*}
        \sqrt{\Tilde{x} \cdot \Tilde{y}}
        & = x \cdot y \\
        & \leq \frac{x^{2} + y^{2}}{2} \\
        & = \frac{\left( \sqrt{\Tilde{x}} \right)^{2} + \left( \sqrt{\Tilde{y}} \right)^{2}}{2} \\
        & = \frac{\Tilde{x} + \Tilde{y}}{2}.
    \end{align*}
\end{frame}



\begin{frame}{Folgerung 2}
    Wählen wir \( x = \sqrt{\Tilde{x}} \) und \( y = \sqrt{\Tilde{y}} \), wobei \( \Tilde{x}, \Tilde{y} \in \bR_{\geq 0} \), so erhalten wir die Abschätzung:
    \begin{align*}
        \sqrt{\Tilde{x} \cdot \Tilde{y}}
        & = x \cdot y \\
        & \leq \frac{x^{2} + y^{2}}{2} \\
        & = \frac{\left( \sqrt{\Tilde{x}} \right)^{2} + \left( \sqrt{\Tilde{y}} \right)^{2}}{2} \\
        & = \frac{\Tilde{x} + \Tilde{y}}{2}.
    \end{align*}
    Das ist die Abschätzung zwischen dem geometrischen und arithmetischen Mittel für zwei Zahlen.
\end{frame}



\begin{frame}{Folgerung 3}
    Wählen wir \( x = \sin\left( \Tilde{x} \right) \) und \( y = \cos\left( \Tilde{x} \right) \), wobei \( \Tilde{x} \in \bR \), so erhalten wir die Abschätzung: 
\end{frame}



\begin{frame}{Folgerung 3}
    Wählen wir \( x = \sin\left( \Tilde{x} \right) \) und \( y = \cos\left( \Tilde{x} \right) \), wobei \( \Tilde{x} \in \bR \), so erhalten wir die Abschätzung:
    \begin{align*}
        \sin\left( \Tilde{x} \right) \cdot \cos\left( \Tilde{x} \right)
    \end{align*}
\end{frame}



\begin{frame}{Folgerung 3}
    Wählen wir \( x = \sin\left( \Tilde{x} \right) \) und \( y = \cos\left( \Tilde{x} \right) \), wobei \( \Tilde{x} \in \bR \), so erhalten wir die Abschätzung:
    \begin{align*}
        \sin\left( \Tilde{x} \right) \cdot \cos\left( \Tilde{x} \right)
        & = x \cdot y \\
    \end{align*}
\end{frame}



\begin{frame}{Folgerung 3}
    Wählen wir \( x = \sin\left( \Tilde{x} \right) \) und \( y = \cos\left( \Tilde{x} \right) \), wobei \( \Tilde{x} \in \bR \), so erhalten wir die Abschätzung:
    \begin{align*}
        \sin\left( \Tilde{x} \right) \cdot \cos\left( \Tilde{y} \right)
        & = x \cdot y \\
        & \leq \frac{x^{2} + y^{2}}{2} \\
    \end{align*}
\end{frame}



\begin{frame}{Folgerung 3}
    Wählen wir \( x = \sin\left( \Tilde{x} \right) \) und \( y = \cos\left( \Tilde{x} \right) \), wobei \( \Tilde{x} \in \bR \), so erhalten wir die Abschätzung:
    \begin{align*}
        \sin\left( \Tilde{x} \right) \cdot \cos\left( \Tilde{x} \right)
        & = x \cdot y \\
        & \leq \frac{x^{2} + y^{2}}{2} \\ 
        & = \frac{\sin^{2}\left( \Tilde{x} \right) + \cos^{2}\left( \Tilde{x} \right)}{2} \\
    \end{align*}
\end{frame}



\begin{frame}{Folgerung 3}
    Wählen wir \( x = \sin\left( \Tilde{x} \right) \) und \( y = \cos\left( \Tilde{x} \right) \), wobei \( \Tilde{x} \in \bR \), so erhalten wir die Abschätzung:
    \begin{align*}
        \sin\left( \Tilde{x} \right) \cdot \cos\left( \Tilde{x} \right)
        & = x \cdot y \\
        & \leq \frac{x^{2} + y^{2}}{2} \\ 
        & = \frac{\sin^{2}\left( \Tilde{x} \right) + \cos^{2}\left( \Tilde{x} \right)}{2} \\
        & = \frac{1}{2}.
    \end{align*}
\end{frame}



\begin{frame}{Folgerung 4}
    Wählen wir \( x = e^{\frac{\Tilde{x}}{2}} \) und \( y = e^{\frac{\Tilde{y}}{2}} \), wobei \( \Tilde{x}, \Tilde{y} \in \bR \) beliebig, so erhalten wir die Abschätzung: 
\end{frame}



\begin{frame}{Folgerung 4}
    Wählen wir \( x = e^{\frac{\Tilde{x}}{2}} \) und \( y = e^{\frac{\Tilde{y}}{2}} \), wobei \( \Tilde{x}, \Tilde{y} \in \bR \) beliebig, so erhalten wir die Abschätzung:
    \begin{align*}
        e^{\frac{\Tilde{x}}{2} + \frac{\Tilde{y}}{2}}
    \end{align*}
\end{frame}



\begin{frame}{Folgerung 4}
    Wählen wir \( x = e^{\frac{\Tilde{x}}{2}} \) und \( y = e^{\frac{\Tilde{y}}{2}} \), wobei \( \Tilde{x}, \Tilde{y} \in \bR \) beliebig, so erhalten wir die Abschätzung:
    \begin{align*}
        e^{\frac{\Tilde{x}}{2} + \frac{\Tilde{y}}{2}}
        & = e^{\frac{\Tilde{x}}{2}} \cdot e^{\frac{\Tilde{y}}{2}} \\
    \end{align*}
\end{frame}



\begin{frame}{Folgerung 4}
    Wählen wir \( x = e^{\frac{\Tilde{x}}{2}} \) und \( y = e^{\frac{\Tilde{y}}{2}} \), wobei \( \Tilde{x}, \Tilde{y} \in \bR \) beliebig, so erhalten wir die Abschätzung:
    \begin{align*}
        e^{\frac{\Tilde{x}}{2} + \frac{\Tilde{y}}{2}}
        & = e^{\frac{\Tilde{x}}{2}} \cdot e^{\frac{\Tilde{y}}{2}} \\
        & = x \cdot y \\
    \end{align*}
\end{frame}



\begin{frame}{Folgerung 4}
    Wählen wir \( x = e^{\frac{\Tilde{x}}{2}} \) und \( y = e^{\frac{\Tilde{y}}{2}} \), wobei \( \Tilde{x}, \Tilde{y} \in \bR \) beliebig, so erhalten wir die Abschätzung:
    \begin{align*}
        e^{\frac{\Tilde{x}}{2} + \frac{\Tilde{y}}{2}}
        & = e^{\frac{\Tilde{x}}{2}} \cdot e^{\frac{\Tilde{y}}{2}} \\
        & = x \cdot y \\
        & \leq \frac{x^{2} + y^{2}}{2} \\
    \end{align*}
\end{frame}



\begin{frame}{Folgerung 4}
    Wählen wir \( x = e^{\frac{\Tilde{x}}{2}} \) und \( y = e^{\frac{\Tilde{y}}{2}} \), wobei \( \Tilde{x}, \Tilde{y} \in \bR \) beliebig, so erhalten wir die Abschätzung:
    \begin{align*}
        e^{\frac{\Tilde{x}}{2} + \frac{\Tilde{y}}{2}}
        & = e^{\frac{\Tilde{x}}{2}} \cdot e^{\frac{\Tilde{y}}{2}} \\
        & = x \cdot y \\
        & \leq \frac{x^{2} + y^{2}}{2} \\
        & = \frac{\left( e^{\frac{\Tilde{x}}{2}} \right)^{2} + \left( e^{\frac{\Tilde{y}}{2}} \right)^{2}}{2} \\
    \end{align*}
\end{frame}



\begin{frame}{Folgerung 4}
    Wählen wir \( x = e^{\frac{\Tilde{x}}{2}} \) und \( y = e^{\frac{\Tilde{y}}{2}} \), wobei \( \Tilde{x}, \Tilde{y} \in \bR \) beliebig, so erhalten wir die Abschätzung:
    \begin{align*}
        e^{\frac{\Tilde{x}}{2} + \frac{\Tilde{y}}{2}}
        & = e^{\frac{\Tilde{x}}{2}} \cdot e^{\frac{\Tilde{y}}{2}} \\
        & = x \cdot y \\
        & \leq \frac{x^{2} + y^{2}}{2} \\
        & = \frac{\left( e^{\frac{\Tilde{x}}{2}} \right)^{2} + \left( e^{\frac{\Tilde{y}}{2}} \right)^{2}}{2} \\
        & = \frac{e^{\Tilde{x}} + e^{\Tilde{y}}}{2}.
    \end{align*}
\end{frame}
% ============================================================

\end{document}