\documentclass[10pt]{beamer}

\title{}
\author{Artur's Mathematikstübchen}
\date{}


% ===== Packages =========
\usepackage[utf8]{inputenc}

\usepackage[natbibapa]{apacite}
\bibliographystyle{apacite}
\usepackage[ngerman]{babel}
\usepackage{graphicx}
\usepackage{fancyhdr}
\usepackage{amsmath}
\usepackage{amssymb}
\usepackage{graphicx}
\usepackage{MnSymbol}
\usepackage{enumitem}
\usepackage{amsthm}
\usepackage{mleftright}
\usepackage{dsfont}
\usepackage{tikz-cd}


\def\bD{\mathbb{D}}               
\def\bE{\mathbb{E}}
\def\bG{\mathbb{G}}
\def\bN{\mathbb{N}}
\def\bP{\mathbb{P}}
\def\bQ{\mathbb{Q}}
\def\bR{\mathbb{R}}
\def\bBarR{\bar{\mathbb{R}}}
\def\bY{\mathbb{Y}}



\def\mA{\mathcal{A}}
\def\mB{\mathcal{B}}
\def\mD{\mathcal{D}}
\def\mE{\mathcal{E}}
\def\mF{\mathcal{F}}
\def\mG{\mathcal{G}}
\def\mH{\mathcal{H}}
\def\mL{\mathcal{L}}
\def\mN{\mathcal{N}}
\def\mP{\mathcal{P}}
\def\mS{\mathcal{S}}
\def\mT{\mathcal{T}}
\def\mX{\mathcal{X}}
\def\mY{\mathcal{Y}}



\usetheme{Madrid}



% ======================== Beginn Document ========================

\begin{document}





% ======================== Begrüßung ==================

\begin{frame}
    \begin{center}
        \textbf{\huge Willkommen in der guten Stube \newline \newline :D}
    \end{center}
\end{frame}
% =====================================================



% ======================== Präsentation ==================

\begin{frame}
    \begin{alertblock}{Aufgabe}
        Für alle \( x, y \in \bR_{> 0} \) zeige man die Abschätzung:
        \begin{align*}
            \frac{\sqrt{x} + \sqrt{y}}{x + y}
            \leq \frac{1}{2} \left( \frac{1}{\sqrt{x}} + \frac{1}{\sqrt{y}} \right).
        \end{align*}
    \end{alertblock}
\end{frame}



\begin{frame}{Hilfsabschätzung}
    
\end{frame}



\begin{frame}{Hilfsabschätzung}
    \begin{block}{Hilfsabschätzung}
        Für alle \( u, v \in \bR_{> 0} \) gilt die Abschätzung:
        \begin{align*}
            \sqrt{uv}
            \leq \frac{u + v}{2}.
        \end{align*}
    \end{block}
\end{frame}



\begin{frame}{Beweis}
    
\end{frame}



\begin{frame}{Beweis}
    Seien \( x, y > 0 \) zwei positive reelle Zahlen.
\end{frame}



\begin{frame}{Beweis}
    Seien \( x, y > 0 \) zwei positive reelle Zahlen. Dann folgt zunächst:
\end{frame}



\begin{frame}{Beweis}
    Seien \( x, y > 0 \) zwei positive reelle Zahlen. Dann folgt zunächst:
    \begin{align*}
        \sqrt{x} + \sqrt{y}
    \end{align*}
\end{frame}



\begin{frame}{Beweis}
    Seien \( x, y > 0 \) zwei positive reelle Zahlen. Dann folgt zunächst:
    \begin{align*}
        \sqrt{x} + \sqrt{y}
        & = \frac{\sqrt{x} \cdot \sqrt{y}}{\sqrt{y}} + \frac{\sqrt{y} \cdot \sqrt{x}}{\sqrt{x}}
    \end{align*}
\end{frame}



\begin{frame}{Beweis}
    Seien \( x, y > 0 \) zwei positive reelle Zahlen. Dann folgt zunächst:
    \begin{align*}
        \sqrt{x} + \sqrt{y}
        & = \frac{\sqrt{x} \cdot \sqrt{y}}{\sqrt{y}} + \frac{\sqrt{y} \cdot \sqrt{x}}{\sqrt{x}} \\
        & \leq \frac{1}{2}\frac{1}{\sqrt{x}}\left( x + y \right) + \frac{1}{2}\frac{1}{\sqrt{y}} \left( x + y \right)
    \end{align*}
\end{frame}



\begin{frame}{Beweis}
    Seien \( x, y > 0 \) zwei positive reelle Zahlen. Dann folgt zunächst:
    \begin{align*}
        \sqrt{x} + \sqrt{y}
        & = \frac{\sqrt{x} \cdot \sqrt{y}}{\sqrt{y}} + \frac{\sqrt{y} \cdot \sqrt{x}}{\sqrt{x}} \\
        & \leq \frac{1}{2}\frac{1}{\sqrt{x}}\left( x + y \right) + \frac{1}{2}\frac{1}{\sqrt{y}} \left( x + y \right) \\
        & = \frac{1}{2} \left( x + y \right) \left( \frac{1}{\sqrt{x}} + \frac{1}{\sqrt{y}} \right).
    \end{align*}
\end{frame}



\begin{frame}{Beweis}
    Damit erhalten wir:
\end{frame}



\begin{frame}{Beweis}
    Damit erhalten wir:
    \begin{align*}
        \frac{\sqrt{x} + \sqrt{y}}{x + y}
    \end{align*}
\end{frame}



\begin{frame}{Beweis}
    Damit erhalten wir:
    \begin{align*}
        \frac{\sqrt{x} + \sqrt{y}}{x + y}
        & = \frac{1}{x + y} \cdot \left( \sqrt{x} + \sqrt{y} \right)
    \end{align*}
\end{frame}



\begin{frame}{Beweis}
    Damit erhalten wir:
    \begin{align*}
        \frac{\sqrt{x} + \sqrt{y}}{x + y}
        & = \frac{1}{x + y} \cdot \left( \sqrt{x} + \sqrt{y} \right) \\
        & \leq \frac{1}{x + y} \cdot \frac{1}{2}\left( x + y \right) \left( \frac{1}{\sqrt{x}} + \frac{1}{\sqrt{y}} \right)
    \end{align*}
\end{frame}



\begin{frame}{Beweis}
    Damit erhalten wir:
    \begin{align*}
        \frac{\sqrt{x} + \sqrt{y}}{x + y}
        & = \frac{1}{x + y} \cdot \left( \sqrt{x} + \sqrt{y} \right) \\
        & \leq \frac{1}{x + y} \cdot \frac{1}{2}\left( x + y \right)\left( \frac{1}{\sqrt{x}} + \frac{1}{\sqrt{y}} \right) \\
        & = \frac{1}{2}\left( \frac{1}{\sqrt{x}} + \frac{1}{\sqrt{y}} \right).
    \end{align*}
\end{frame}
% ============================================================

\end{document}