\documentclass[10pt]{beamer}

\title{}
\author{Artur's \( \oint \) Mathematikstübchen}
\date{}


% ===== Packages =========
\usepackage[utf8]{inputenc}

\usepackage[natbibapa]{apacite}
\bibliographystyle{apacite}
\usepackage[ngerman]{babel}
\usepackage{graphicx}
\usepackage{fancyhdr}
\usepackage{amsmath}
\usepackage{amssymb}
\usepackage{graphicx}
\usepackage{MnSymbol}
\usepackage{enumitem}
\usepackage{amsthm}
\usepackage{mleftright}
\usepackage{dsfont}
\usepackage{tikz-cd}


\def\bC{\mathbb{C}}
\def\bD{\mathbb{D}}               
\def\bE{\mathbb{E}}
\def\bG{\mathbb{G}}
\def\bN{\mathbb{N}}
\def\bP{\mathbb{P}}
\def\bQ{\mathbb{Q}}
\def\bR{\mathbb{R}}
\def\bBarR{\bar{\mathbb{R}}}
\def\bY{\mathbb{Y}}



\def\mA{\mathcal{A}}
\def\mB{\mathcal{B}}
\def\mD{\mathcal{D}}
\def\mE{\mathcal{E}}
\def\mF{\mathcal{F}}
\def\mG{\mathcal{G}}
\def\mH{\mathcal{H}}
\def\mL{\mathcal{L}}
\def\mN{\mathcal{N}}
\def\mP{\mathcal{P}}
\def\mS{\mathcal{S}}
\def\mT{\mathcal{T}}
\def\mX{\mathcal{X}}
\def\mY{\mathcal{Y}}



\usetheme{Madrid}



% ======================== Beginn Document ========================

\begin{document}





% ======================== Begrüßung ==================

\begin{frame}
    \begin{center}
        \textbf{\huge Willkommen in der guten Stube \newline \newline :D}
    \end{center}
\end{frame}
% =====================================================



% ======================== Präsentation ==================

\begin{frame}
    \begin{alertblock}{Aufgabe}
        Seien \( x_{1}, x_{2}, \ldots, x_{n}, y_{1}, y_{2}, \ldots, y_{n} > 0 \) positive reelle Zahlen. Man zeige die Gültigkeit der folgenden Abschätzung:
        \begin{align*}
            \min\left\{ \frac{x_{k}}{y_{k}} : k = 1, 2, \ldots, n \right\}
            \leq \frac{\sum_{k = 1}^{n} x_{k}}{\sum_{k = 1}^{n} y_{k}}
            \leq \max\left\{ \frac{x_{k}}{y_{k}} : k = 1, 2, \ldots, n \right\}.
        \end{align*}
    \end{alertblock}
\end{frame}



\begin{frame}{Beweis}
    
\end{frame}



\begin{frame}{Beweis}
    Seien \( x_{1}, x_{2}, \ldots, x_{n}, y_{1}, y_{2}, \ldots, y_{n} > 0 \).
\end{frame}


\begin{frame}{Beweis}
     Seien \( x_{1}, x_{2}, \ldots, x_{n}, y_{1}, y_{2}, \ldots, y_{n} > 0 \). Weiter setzten wir \( m \coloneq \min\left\{ \frac{x_{k}}{y_{k}} : k = 1, 2, \ldots, n \right\} \) und \( M \coloneq \max\left\{ \frac{x_{k}}{y_{k}} : k = 1, 2, \ldots, n \right\} \).
\end{frame}



\begin{frame}{Beweis}
     Seien \( x_{1}, x_{2}, \ldots, x_{n}, y_{1}, y_{2}, \ldots, y_{n} > 0 \). Weiter setzten wir \( m \coloneq \min\left\{ \frac{x_{k}}{y_{k}} : k = 1, 2, \ldots, n \right\} \) und \( M \coloneq \max\left\{ \frac{x_{k}}{y_{k}} : k = 1, 2, \ldots, n \right\} \). Es folgen die Abschätzungen:
     \begin{align*}
        \sum_{k = 1}^{n} x_{k}
     \end{align*}
\end{frame}



\begin{frame}{Beweis}
     Seien \( x_{1}, x_{2}, \ldots, x_{n}, y_{1}, y_{2}, \ldots, y_{n} > 0 \). Weiter setzten wir \( m \coloneq \min\left\{ \frac{x_{k}}{y_{k}} : k = 1, 2, \ldots, n \right\} \) und \( M \coloneq \max\left\{ \frac{x_{k}}{y_{k}} : k = 1, 2, \ldots, n \right\} \). Es folgen die Abschätzungen:
     \begin{align*}
        \sum_{k = 1}^{n} x_{k}
        & = \sum_{k = 1}^{n} y_{k} \cdot \frac{x_{k}}{y_{k}}
     \end{align*}
\end{frame}



\begin{frame}{Beweis}
     Seien \( x_{1}, x_{2}, \ldots, x_{n}, y_{1}, y_{2}, \ldots, y_{n} > 0 \). Weiter setzten wir \( m \coloneq \min\left\{ \frac{x_{k}}{y_{k}} : k = 1, 2, \ldots, n \right\} \) und \( M \coloneq \max\left\{ \frac{x_{k}}{y_{k}} : k = 1, 2, \ldots, n \right\} \). Es folgen die Abschätzungen:
     \begin{align*}
        \sum_{k = 1}^{n} x_{k}
        & = \sum_{k = 1}^{n} y_{k} \cdot \frac{x_{k}}{y_{k}} \\
        & \geq \sum_{k = 1}^{n} y_{k} \cdot m
     \end{align*}
\end{frame}



\begin{frame}{Beweis}
     Seien \( x_{1}, x_{2}, \ldots, x_{n}, y_{1}, y_{2}, \ldots, y_{n} > 0 \). Weiter setzten wir \( m \coloneq \min\left\{ \frac{x_{k}}{y_{k}} : k = 1, 2, \ldots, n \right\} \) und \( M \coloneq \max\left\{ \frac{x_{k}}{y_{k}} : k = 1, 2, \ldots, n \right\} \). Es folgen die Abschätzungen:
     \begin{align*}
        \sum_{k = 1}^{n} x_{k}
        & = \sum_{k = 1}^{n} y_{k} \cdot \frac{x_{k}}{y_{k}} \\
        & \geq \sum_{k = 1}^{n} y_{k} \cdot m \\
        & = m \cdot \sum_{k = 1}^{n} y_{k}
     \end{align*}
\end{frame}



\begin{frame}{Beweis}
     Seien \( x_{1}, x_{2}, \ldots, x_{n}, y_{1}, y_{2}, \ldots, y_{n} > 0 \). Weiter setzten wir \( m \coloneq \min\left\{ \frac{x_{k}}{y_{k}} : k = 1, 2, \ldots, n \right\} \) und \( M \coloneq \max\left\{ \frac{x_{k}}{y_{k}} : k = 1, 2, \ldots, n \right\} \). Es folgen die Abschätzungen:
     \begin{align*}
        \sum_{k = 1}^{n} x_{k}
        & = \sum_{k = 1}^{n} y_{k} \cdot \frac{x_{k}}{y_{k}} \\
        & \geq \sum_{k = 1}^{n} y_{k} \cdot m \\
        & = m \cdot \sum_{k = 1}^{n} y_{k}
     \end{align*}
     \begin{align*}
        \sum_{k = 1}^{n} x_{k}
     \end{align*}
\end{frame}



\begin{frame}{Beweis}
     Seien \( x_{1}, x_{2}, \ldots, x_{n}, y_{1}, y_{2}, \ldots, y_{n} > 0 \). Weiter setzten wir \( m \coloneq \min\left\{ \frac{x_{k}}{y_{k}} : k = 1, 2, \ldots, n \right\} \) und \( M \coloneq \max\left\{ \frac{x_{k}}{y_{k}} : k = 1, 2, \ldots, n \right\} \). Es folgen die Abschätzungen:
     \begin{align*}
        \sum_{k = 1}^{n} x_{k}
        & = \sum_{k = 1}^{n} y_{k} \cdot \frac{x_{k}}{y_{k}} \\
        & \geq \sum_{k = 1}^{n} y_{k} \cdot m \\
        & = m \cdot \sum_{k = 1}^{n} y_{k}
     \end{align*}
     \begin{align*}
        \sum_{k = 1}^{n} x_{k}
        & = \sum_{k = 1}^{n} y_{k} \cdot \frac{x_{k}}{y_{k}}
     \end{align*}
\end{frame}



\begin{frame}{Beweis}
     Seien \( x_{1}, x_{2}, \ldots, x_{n}, y_{1}, y_{2}, \ldots, y_{n} > 0 \). Weiter setzten wir \( m \coloneq \min\left\{ \frac{x_{k}}{y_{k}} : k = 1, 2, \ldots, n \right\} \) und \( M \coloneq \max\left\{ \frac{x_{k}}{y_{k}} : k = 1, 2, \ldots, n \right\} \). Es folgen die Abschätzungen:
     \begin{align*}
        \sum_{k = 1}^{n} x_{k}
        & = \sum_{k = 1}^{n} y_{k} \cdot \frac{x_{k}}{y_{k}} \\
        & \geq \sum_{k = 1}^{n} y_{k} \cdot m \\
        & = m \cdot \sum_{k = 1}^{n} y_{k}
     \end{align*}
     \begin{align*}
        \sum_{k = 1}^{n} x_{k}
        & = \sum_{k = 1}^{n} y_{k} \cdot \frac{x_{k}}{y_{k}} \\
        & \leq \sum_{k = 1}^{n} y_{k} \cdot M
     \end{align*}
\end{frame}



\begin{frame}{Beweis}
     Seien \( x_{1}, x_{2}, \ldots, x_{n}, y_{1}, y_{2}, \ldots, y_{n} > 0 \). Weiter setzten wir \( m \coloneq \min\left\{ \frac{x_{k}}{y_{k}} : k = 1, 2, \ldots, n \right\} \) und \( M \coloneq \max\left\{ \frac{x_{k}}{y_{k}} : k = 1, 2, \ldots, n \right\} \). Es folgen die Abschätzungen:
     \begin{align*}
        \sum_{k = 1}^{n} x_{k}
        & = \sum_{k = 1}^{n} y_{k} \cdot \frac{x_{k}}{y_{k}} \\
        & \geq \sum_{k = 1}^{n} y_{k} \cdot m \\
        & = m \cdot \sum_{k = 1}^{n} y_{k}
     \end{align*}
     \begin{align*}
        \sum_{k = 1}^{n} x_{k}
        & = \sum_{k = 1}^{n} y_{k} \cdot \frac{x_{k}}{y_{k}} \\
        & \leq \sum_{k = 1}^{n} y_{k} \cdot M \\
        & = M \cdot \sum_{k = 1}^{n} y_{k}.
     \end{align*}
\end{frame}



\begin{frame}{Beweis}
     Daraus folgt schließlich:
\end{frame}



\begin{frame}{Beweis}
     Daraus folgt schließlich:
     \begin{align*}
         m
     \end{align*}
\end{frame}



\begin{frame}{Beweis}
     Daraus folgt schließlich:
     \begin{align*}
         m
         = \frac{m \cdot \sum_{k = 1}^{n} y_{k}}{\sum_{k = 1}^{n} y_{k}}
     \end{align*}
\end{frame}



\begin{frame}{Beweis}
     Daraus folgt schließlich:
     \begin{align*}
         m
         = \frac{m \cdot \sum_{k = 1}^{n} y_{k}}{\sum_{k = 1}^{n} y_{k}}
         \leq \frac{\sum_{k = 1}^{n} x_{k}}{\sum_{k = 1}^{n} y_{k}}
     \end{align*}
\end{frame}



\begin{frame}{Beweis}
     Daraus folgt schließlich:
     \begin{align*}
         m
         = \frac{m \cdot \sum_{k = 1}^{n} y_{k}}{\sum_{k = 1}^{n} y_{k}}
         \leq \frac{\sum_{k = 1}^{n} x_{k}}{\sum_{k = 1}^{n} y_{k}}
         \leq \frac{M \cdot \sum_{k = 1}^{n} y_{k}}{\sum_{k = 1}^{n} y_{k}}
     \end{align*}
\end{frame}



\begin{frame}{Beweis}
     Daraus folgt schließlich:
     \begin{align*}
         m
         = \frac{m \cdot \sum_{k = 1}^{n} y_{k}}{\sum_{k = 1}^{n} y_{k}}
         \leq \frac{\sum_{k = 1}^{n} x_{k}}{\sum_{k = 1}^{n} y_{k}}
         \leq \frac{M \cdot \sum_{k = 1}^{n} y_{k}}{\sum_{k = 1}^{n} y_{k}}
         = M.
     \end{align*}
\end{frame}
% ============================================================
\end{document}