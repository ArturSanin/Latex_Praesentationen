\documentclass[10pt]{beamer}

\title{}
\author{Artur's \( \oint \) Mathematikstübchen}
\date{}


% ===== Packages =========
\usepackage[utf8]{inputenc}

\usepackage[natbibapa]{apacite}
\bibliographystyle{apacite}
\usepackage[ngerman]{babel}
\usepackage{graphicx}
\usepackage{fancyhdr}
\usepackage{amsmath}
\usepackage{amssymb}
\usepackage{graphicx}
\usepackage{MnSymbol}
\usepackage{enumitem}
\usepackage{amsthm}
\usepackage{mleftright}
\usepackage{dsfont}
\usepackage{tikz-cd}


\def\bC{\mathbb{C}}
\def\bD{\mathbb{D}}               
\def\bE{\mathbb{E}}
\def\bG{\mathbb{G}}
\def\bN{\mathbb{N}}
\def\bP{\mathbb{P}}
\def\bQ{\mathbb{Q}}
\def\bR{\mathbb{R}}
\def\bBarR{\bar{\mathbb{R}}}
\def\bY{\mathbb{Y}}



\def\mA{\mathcal{A}}
\def\mB{\mathcal{B}}
\def\mD{\mathcal{D}}
\def\mE{\mathcal{E}}
\def\mF{\mathcal{F}}
\def\mG{\mathcal{G}}
\def\mH{\mathcal{H}}
\def\mL{\mathcal{L}}
\def\mN{\mathcal{N}}
\def\mP{\mathcal{P}}
\def\mS{\mathcal{S}}
\def\mT{\mathcal{T}}
\def\mX{\mathcal{X}}
\def\mY{\mathcal{Y}}



\usetheme{Madrid}



% ======================== Beginn Document ========================

\begin{document}





% ======================== Begrüßung ==================

\begin{frame}
    \begin{center}
        \textbf{\huge Willkommen in der guten Stube \newline \newline :D}
    \end{center}
\end{frame}
% =====================================================



% ======================== Präsentation ==================

\begin{frame}
    \begin{alertblock}{Aufgabe}
        Für alle \( x, y \in \bR \) zeige man die Abschätzung:
        \begin{align*}
            \left( x + y \right)^{2}
            \leq 2 \cdot \left( x^{2} + y^{2} \right).
        \end{align*}
    \end{alertblock}
\end{frame}



\begin{frame}{Beweis: Erster Ansatz}
    
\end{frame}



\begin{frame}{Beweis: Erster Ansatz}
    Seien \( x, y \in \bR \) zwei beliebige reelle Zahlen.
\end{frame}



\begin{frame}{Beweis: Erster Ansatz}
    Seien \( x, y \in \bR \) zwei beliebige reelle Zahlen. Dann gilt:
\end{frame}



\begin{frame}{Beweis: Erster Ansatz}
    Seien \( x, y \in \bR \) zwei beliebige reelle Zahlen. Dann gilt:
    \begin{align*}
        \left( x + y \right)^{2}
    \end{align*}
\end{frame}



\begin{frame}{Beweis: Erster Ansatz}
    Seien \( x, y \in \bR \) zwei beliebige reelle Zahlen. Dann gilt:
    \begin{align*}
        \left( x + y \right)^{2}
        & = x^{2} + 2xy + y^{2}
    \end{align*}
\end{frame}



\begin{frame}{Beweis: Erster Ansatz}
    Seien \( x, y \in \bR \) zwei beliebige reelle Zahlen. Dann gilt:
    \begin{align*}
        \left( x + y \right)^{2}
        & = x^{2} + 2xy + y^{2} \\
        & = x^{2} + x^{2} - x^{2} + 2xy + y^{2} \\
    \end{align*}
\end{frame}



\begin{frame}{Beweis: Erster Ansatz}
    Seien \( x, y \in \bR \) zwei beliebige reelle Zahlen. Dann gilt:
    \begin{align*}
        \left( x + y \right)^{2}
        & = x^{2} + 2xy + y^{2} \\
        & = x^{2} + x^{2} - x^{2} + 2xy + y^{2} \\
        & = x^{2} + x^{2} - x^{2} + 2xy + y^{2} + y^{2} - y^{2} \\
    \end{align*}
\end{frame}



\begin{frame}{Beweis: Erster Ansatz}
    Seien \( x, y \in \bR \) zwei beliebige reelle Zahlen. Dann gilt:
    \begin{align*}
        \left( x + y \right)^{2}
        & = x^{2} + 2xy + y^{2} \\
        & = x^{2} + x^{2} - x^{2} + 2xy + y^{2} \\
        & = x^{2} + x^{2} - x^{2} + 2xy + y^{2} + y^{2} - y^{2} \\
        & = 2x^{2} - x^{2} + 2xy + 2y^{2} - y^{2}
    \end{align*}
\end{frame}



\begin{frame}{Beweis: Erster Ansatz}
    Seien \( x, y \in \bR \) zwei beliebige reelle Zahlen. Dann gilt:
    \begin{align*}
        \left( x + y \right)^{2}
        & = x^{2} + 2xy + y^{2} \\
        & = x^{2} + x^{2} - x^{2} + 2xy + y^{2} \\
        & = x^{2} + x^{2} - x^{2} + 2xy + y^{2} + y^{2} - y^{2} \\
        & = 2x^{2} - x^{2} + 2xy + 2y^{2} - y^{2} \\
        & = 2x^{2} + 2y^{2} - \left( x^{2} - 2xy + y^{2} \right)
    \end{align*}
\end{frame}



\begin{frame}{Beweis: Erster Ansatz}
    Seien \( x, y \in \bR \) zwei beliebige reelle Zahlen. Dann gilt:
    \begin{align*}
        \left( x + y \right)^{2}
        & = x^{2} + 2xy + y^{2} \\
        & = x^{2} + x^{2} - x^{2} + 2xy + y^{2} \\
        & = x^{2} + x^{2} - x^{2} + 2xy + y^{2} + y^{2} - y^{2} \\
        & = 2x^{2} - x^{2} + 2xy + 2y^{2} - y^{2} \\
        & = 2x^{2} + 2y^{2} - \left( x^{2} - 2xy + y^{2} \right) \\
        & = 2x^{2} + 2y^{2} - \left( x - y \right)^{2} 
    \end{align*}
\end{frame}



\begin{frame}{Beweis: Erster Ansatz}
    Seien \( x, y \in \bR \) zwei beliebige reelle Zahlen. Dann gilt:
    \begin{align*}
        \left( x + y \right)^{2}
        & = x^{2} + 2xy + y^{2} \\
        & = x^{2} + x^{2} - x^{2} + 2xy + y^{2} \\
        & = x^{2} + x^{2} - x^{2} + 2xy + y^{2} + y^{2} - y^{2} \\
        & = 2x^{2} - x^{2} + 2xy + 2y^{2} - y^{2} \\
        & = 2x^{2} + 2y^{2} - \left( x^{2} - 2xy + y^{2} \right) \\
        & = 2x^{2} + 2y^{2} - \left( x - y \right)^{2} \\
        & \leq 2x^{2} + 2y^{2}.
    \end{align*}
\end{frame}



\begin{frame}{Beweis: Erster Ansatz}
    Seien \( x, y \in \bR \) zwei beliebige reelle Zahlen. Dann gilt:
    \begin{align*}
        \left( x + y \right)^{2}
        & = x^{2} + 2xy + y^{2} \\
        & = x^{2} + x^{2} - x^{2} + 2xy + y^{2} \\
        & = x^{2} + x^{2} - x^{2} + 2xy + y^{2} + y^{2} - y^{2} \\
        & = 2x^{2} - x^{2} + 2xy + 2y^{2} - y^{2} \\
        & = 2x^{2} + 2y^{2} - \left( x^{2} - 2xy + y^{2} \right) \\
        & = 2x^{2} + 2y^{2} - \left( x - y \right)^{2} \\
        & \leq 2x^{2} + 2y^{2} \\
        & = 2 \cdot \left( x^{2} + y^{2} \right).
    \end{align*}
\end{frame}



\begin{frame}{Beweis: Zweiter Ansatz}
    
\end{frame}



\begin{frame}{Beweis: Zweiter Ansatz}
    Für alle \( \Tilde{x}, \Tilde{y} \in \bR \) gilt die Abschätzung:
\end{frame}



\begin{frame}{Beweis: Zweiter Ansatz}
    Für alle \( \Tilde{x}, \Tilde{y} \in \bR \) gilt die Abschätzung:
    \begin{align*}
        2\Tilde{x}\Tilde{y}
        \leq \Tilde{x}^{2} + \Tilde{y}^{2}.
    \end{align*}
\end{frame}



\begin{frame}{Beweis: Zweiter Ansatz}
    Für alle \( \Tilde{x}, \Tilde{y} \in \bR \) gilt die Abschätzung:
    \begin{align*}
        2\Tilde{x}\Tilde{y}
        \leq \Tilde{x}^{2} + \Tilde{y}^{2}.
    \end{align*}
    Seien nun \( x, y \in \bR \) zwei beliebige reelle Zahlen.
\end{frame}



\begin{frame}{Beweis: Zweiter Ansatz}
    Für alle \( \Tilde{x}, \Tilde{y} \in \bR \) gilt die Abschätzung:
    \begin{align*}
        2\Tilde{x}\Tilde{y}
        \leq \Tilde{x}^{2} + \Tilde{y}^{2}.
    \end{align*}
    Seien nun \( x, y \in \bR \) zwei beliebige reelle Zahlen. Dann folgt:
\end{frame}




\begin{frame}{Beweis: Zweiter Ansatz}
    Für alle \( \Tilde{x}, \Tilde{y} \in \bR \) gilt die Abschätzung:
    \begin{align*}
        2\Tilde{x}\Tilde{y}
        \leq \Tilde{x}^{2} + \Tilde{y}^{2}.
    \end{align*}
    Seien nun \( x, y \in \bR \) zwei beliebige reelle Zahlen. Dann folgt:
    \begin{align*}
        \left( x + y \right)^{2}
    \end{align*}
\end{frame}



\begin{frame}{Beweis: Zweiter Ansatz}
    Für alle \( \Tilde{x}, \Tilde{y} \in \bR \) gilt die Abschätzung:
    \begin{align*}
        2\Tilde{x}\Tilde{y}
        \leq \Tilde{x}^{2} + \Tilde{y}^{2}.
    \end{align*}
    Seien nun \( x, y \in \bR \) zwei beliebige reelle Zahlen. Dann folgt:
    \begin{align*}
        \left( x + y \right)^{2}
        & = x^{2} + 2xy + y^{2}
    \end{align*}
\end{frame}



\begin{frame}{Beweis: Zweiter Ansatz}
    Für alle \( \Tilde{x}, \Tilde{y} \in \bR \) gilt die Abschätzung:
    \begin{align*}
        2\Tilde{x}\Tilde{y}
        \leq \Tilde{x}^{2} + \Tilde{y}^{2}.
    \end{align*}
    Seien nun \( x, y \in \bR \) zwei beliebige reelle Zahlen. Dann folgt:
    \begin{align*}
        \left( x + y \right)^{2}
        & = x^{2} + 2xy + y^{2} \\
        & \leq x^{2} + x^{2} + y^{2} + y^{2}
    \end{align*}
\end{frame}



\begin{frame}{Beweis: Zweiter Ansatz}
    Für alle \( \Tilde{x}, \Tilde{y} \in \bR \) gilt die Abschätzung:
    \begin{align*}
        2\Tilde{x}\Tilde{y}
        \leq \Tilde{x}^{2} + \Tilde{y}^{2}.
    \end{align*}
    Seien nun \( x, y \in \bR \) zwei beliebige reelle Zahlen. Dann folgt:
    \begin{align*}
        \left( x + y \right)^{2}
        & = x^{2} + 2xy + y^{2} \\
        & \leq x^{2} + x^{2} + y^{2} + y^{2} \\
        & = 2x^{2} + 2y^{2}.
    \end{align*}
\end{frame}



\begin{frame}{Beweis: Zweiter Ansatz}
    Für alle \( \Tilde{x}, \Tilde{y} \in \bR \) gilt die Abschätzung:
    \begin{align*}
        2\Tilde{x}\Tilde{y}
        \leq \Tilde{x}^{2} + \Tilde{y}^{2}.
    \end{align*}
    Seien nun \( x, y \in \bR \) zwei beliebige reelle Zahlen. Dann folgt:
    \begin{align*}
        \left( x + y \right)^{2}
        & = x^{2} + 2xy + y^{2} \\
        & \leq x^{2} + x^{2} + y^{2} + y^{2} \\
        & = 2x^{2} + 2y^{2} \\
        & = 2 \cdot \left( x^{2} + y^{2} \right).
    \end{align*}
\end{frame}
% ============================================================
\end{document}