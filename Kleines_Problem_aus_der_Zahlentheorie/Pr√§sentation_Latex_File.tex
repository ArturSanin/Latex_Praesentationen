\documentclass[10pt]{beamer}

\title{}
\author{Artur's Mathematikstübchen}
\date{}


% ===== Packages =========
\usepackage[utf8]{inputenc}

\usepackage[natbibapa]{apacite}
\bibliographystyle{apacite}
\usepackage[ngerman]{babel}
\usepackage{graphicx}
\usepackage{fancyhdr}
\usepackage{amsmath}
\usepackage{amssymb}
\usepackage{graphicx}
\usepackage{MnSymbol}
\usepackage{enumitem}
\usepackage{amsthm}
\usepackage{mleftright}
\usepackage{dsfont}
\usepackage{tikz-cd}


\def\bD{\mathbb{D}}               
\def\bE{\mathbb{E}}
\def\bG{\mathbb{G}}
\def\bN{\mathbb{N}}
\def\bP{\mathbb{P}}
\def\bQ{\mathbb{Q}}
\def\bR{\mathbb{R}}
\def\bBarR{\bar{\mathbb{R}}}
\def\bY{\mathbb{Y}}
\def\bZ{\mathbb{Z}}



\def\mA{\mathcal{A}}
\def\mB{\mathcal{B}}
\def\mD{\mathcal{D}}
\def\mE{\mathcal{E}}
\def\mF{\mathcal{F}}
\def\mG{\mathcal{G}}
\def\mH{\mathcal{H}}
\def\mL{\mathcal{L}}
\def\mN{\mathcal{N}}
\def\mP{\mathcal{P}}
\def\mS{\mathcal{S}}
\def\mT{\mathcal{T}}
\def\mX{\mathcal{X}}
\def\mY{\mathcal{Y}}



\usetheme{Madrid}



% ======================== Beginn Document ========================

\begin{document}





% ======================== Begrüßung ==================

\begin{frame}
    \begin{center}
        \textbf{\huge Willkommen in der guten Stube \newline \newline :D}
    \end{center}
\end{frame}
% =====================================================



% ======================== Präsentation ==================

\begin{frame}
    \begin{alertblock}{Aufgabe}
        Seien \( a = m^{2} + n^{2} \) und \( b = p^{2} + q^{2} \) zwei Zahlen, die als Summe zweier quadrate ganzer Zahlen geschrieben werden können. Man zeige, es existieren \( k, l \in \bZ \) mit \( ab = k^{2} + l^{2}  \).
    \end{alertblock}
\end{frame}



\begin{frame}{Beweis}
    
\end{frame}



\begin{frame}{Beweis}
    Seien \( a = m^{2} + n^{2} \) und \( b = p^{2} + q^{2} \) für \( m, n, p, q \in \bZ \).
\end{frame}



\begin{frame}{Beweis}
    Seien \( a = m^{2} + n^{2} \) und \( b = p^{2} + q^{2} \) für \( m, n, p, q \in \bZ \). Dann gilt:
\end{frame}



\begin{frame}{Beweis}
    Seien \( a = m^{2} + n^{2} \) und \( b = p^{2} + q^{2} \) für \( m, n, p, q \in \bZ \). Dann gilt:
    \begin{align*}
        ab
    \end{align*}
\end{frame}



\begin{frame}{Beweis}
    Seien \( a = m^{2} + n^{2} \) und \( b = p^{2} + q^{2} \) für \( m, n, p, q \in \bZ \). Dann gilt:
    \begin{align*}
        ab
        & = \left( m^{2} + n^{2} \right) \left( p^{2} + q^{2} \right)
    \end{align*}
\end{frame}



\begin{frame}{Beweis}
    Seien \( a = m^{2} + n^{2} \) und \( b = p^{2} + q^{2} \) für \( m, n, p, q \in \bZ \). Dann gilt:
    \begin{align*}
        ab
        & = \left( m^{2} + n^{2} \right) \left( p^{2} + q^{2} \right) \\
        & = m^{2}p^{2} + m^{2}q^{2} + n^{2}p^{2} + n^{2}q^{2}
    \end{align*}
\end{frame}



\begin{frame}{Beweis}
    Seien \( a = m^{2} + n^{2} \) und \( b = p^{2} + q^{2} \) für \( m, n, p, q \in \bZ \). Dann gilt:
    \begin{align*}
        ab
        & = \left( m^{2} + n^{2} \right) \left( p^{2} + q^{2} \right) \\
        & = m^{2}p^{2} + m^{2}q^{2} + n^{2}p^{2} + n^{2}q^{2} \\
        & = \left( mp \right)^{2} + \left( mq \right)^{2} + \left( np \right)^{2} + \left( nq \right)^{2}
    \end{align*}
\end{frame}



\begin{frame}{Beweis}
    Seien \( a = m^{2} + n^{2} \) und \( b = p^{2} + q^{2} \) für \( m, n, p, q \in \bZ \). Dann gilt:
    \begin{align*}
        ab
        & = \left( m^{2} + n^{2} \right) \left( p^{2} + q^{2} \right) \\
        & = m^{2}p^{2} + m^{2}q^{2} + n^{2}p^{2} + n^{2}q^{2} \\
        & = \left( mp \right)^{2} + \left( mq \right)^{2} + \left( np \right)^{2} + \left( nq \right)^{2} \\
        & = \left( mp \right)^{2} + \left( nq \right)^{2} + \left( mq \right)^{2} + \left( np \right)^{2} 
    \end{align*}
\end{frame}



\begin{frame}{Beweis}
    Seien \( a = m^{2} + n^{2} \) und \( b = p^{2} + q^{2} \) für \( m, n, p, q \in \bZ \). Dann gilt:
    \begin{align*}
        ab
        & = \left( m^{2} + n^{2} \right) \left( p^{2} + q^{2} \right) \\
        & = m^{2}p^{2} + m^{2}q^{2} + n^{2}p^{2} + n^{2}q^{2} \\
        & = \left( mp \right)^{2} + \left( mq \right)^{2} + \left( np \right)^{2} + \left( nq \right)^{2} \\
        & = \left( mp \right)^{2} + \left( nq \right)^{2} + \left( mq \right)^{2} + \left( np \right)^{2} \\
        & = \left( mp \right)^{2} + 2mpnq + \left( nq \right)^{2} + \left( mq \right)^{2} - 2mqnp + \left( np \right)^{2} \\
    \end{align*}
\end{frame}



\begin{frame}{Beweis}
    Seien \( a = m^{2} + n^{2} \) und \( b = p^{2} + q^{2} \) für \( m, n, p, q \in \bZ \). Dann gilt:
    \begin{align*}
        ab
        & = \left( m^{2} + n^{2} \right) \left( p^{2} + q^{2} \right) \\
        & = m^{2}p^{2} + m^{2}q^{2} + n^{2}p^{2} + n^{2}q^{2} \\
        & = \left( mp \right)^{2} + \left( mq \right)^{2} + \left( np \right)^{2} + \left( nq \right)^{2} \\
        & = \left( mp \right)^{2} + \left( nq \right)^{2} + \left( mq \right)^{2} + \left( np \right)^{2} \\
        & = \left( mp \right)^{2} + 2mpnq + \left( nq \right)^{2} + \left( mq \right)^{2} - 2mqnp + \left( np \right)^{2} \\
        & = \left( mp + nq \right)^{2} + \left( mq - np \right)^{2}.
    \end{align*}
\end{frame}
% ============================================================

\end{document}