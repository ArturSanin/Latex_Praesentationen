\documentclass[10pt]{beamer}

\title{}
\author{Artur's \( \oint \) Mathematikstübchen}
\date{}


% ===== Packages =========
\usepackage[utf8]{inputenc}

\usepackage[natbibapa]{apacite}
\bibliographystyle{apacite}
\usepackage[ngerman]{babel}
\usepackage{graphicx}
\usepackage{fancyhdr}
\usepackage{amsmath}
\usepackage{amssymb}
\usepackage{graphicx}
\usepackage{MnSymbol}
\usepackage{enumitem}
\usepackage{amsthm}
\usepackage{mleftright}
\usepackage{dsfont}
\usepackage{tikz-cd}


\def\bC{\mathbb{C}}
\def\bD{\mathbb{D}}               
\def\bE{\mathbb{E}}
\def\bG{\mathbb{G}}
\def\bN{\mathbb{N}}
\def\bP{\mathbb{P}}
\def\bQ{\mathbb{Q}}
\def\bR{\mathbb{R}}
\def\bBarR{\bar{\mathbb{R}}}
\def\bY{\mathbb{Y}}



\def\mA{\mathcal{A}}
\def\mB{\mathcal{B}}
\def\mD{\mathcal{D}}
\def\mE{\mathcal{E}}
\def\mF{\mathcal{F}}
\def\mG{\mathcal{G}}
\def\mH{\mathcal{H}}
\def\mL{\mathcal{L}}
\def\mN{\mathcal{N}}
\def\mP{\mathcal{P}}
\def\mS{\mathcal{S}}
\def\mT{\mathcal{T}}
\def\mX{\mathcal{X}}
\def\mY{\mathcal{Y}}



\usetheme{Madrid}



% ======================== Beginn Document ========================

\begin{document}





% ======================== Begrüßung ==================

\begin{frame}
    \begin{center}
        \textbf{\huge Willkommen in der guten Stube \newline \newline :D}
    \end{center}
\end{frame}
% =====================================================



% ======================== Präsentation ==================

\begin{frame}
    \begin{alertblock}{Aufgabe}
        Seien \( x, y, z > 0 \) drei positive reellen Zahlen. Man zeige die Gültigkeit der folgenden Abschätzung:
        \begin{align*}
            \frac{yz}{y + z} + \frac{xz}{x + z} + \frac{xy}{x + y}
            \leq \frac{1}{2} \left( x + y + z \right).
        \end{align*}
    \end{alertblock}
\end{frame}



\begin{frame}{Hilfsabschätzung}
    
\end{frame}



\begin{frame}{Hilfsabschätzung}
    Für den Beweis verwenden wir die Ungleichung zwischen dem geometrischen und arithmetischen Mittel:
\end{frame}



\begin{frame}{Hilfsabschätzung}
    Für den Beweis verwenden wir die Ungleichung zwischen dem geometrischen und arithmetischen Mittel:
    \begin{block}{Hilfsabschätzung}
        Für alle \( a_{1}, \ldots, a_{p} \geq 0 \), \( p \in \bN \), gilt die Abschätzung:
        \begin{align*}
            \sqrt[p]{a_{1} \cdot \ldots \cdot a_{p}} 
            \leq \frac{a_{1} + \ldots + a_{p}}{p}.
        \end{align*}
    \end{block}
\end{frame}



\begin{frame}{Hilfsabschätzung}
    Für den Beweis verwenden wir die Ungleichung zwischen dem geometrischen und arithmetischen Mittel:
    \begin{block}{Hilfsabschätzung}
        Für alle \( a_{1}, \ldots, a_{p} \geq 0 \), \( p \in \bN \), gilt die Abschätzung:
        \begin{align*}
            \sqrt[p]{a_{1} \cdot \ldots \cdot a_{p}} 
            \leq \frac{a_{1} + \ldots + a_{p}}{p}.
        \end{align*}
    \end{block}
    Insbesondere gilt für \( p = 2 \):   
\end{frame}



\begin{frame}{Hilfsabschätzung}
    Für den Beweis verwenden wir die Ungleichung zwischen dem geometrischen und arithmetischen Mittel:
    \begin{block}{Hilfsabschätzung}
        Für alle \( a_{1}, \ldots, a_{p} \geq 0 \), \( p \in \bN \), gilt die Abschätzung:
        \begin{align*}
            \sqrt[p]{a_{1} \cdot \ldots \cdot a_{p}} 
            \leq \frac{a_{1} + \ldots + a_{p}}{p}.
        \end{align*}
    \end{block}
    Insbesondere gilt für \( p = 2 \): 
    \begin{align*}
        \sqrt{a_{1} a_{2}}
        \leq \frac{1}{2} \left( a_{1} + a_{2} \right)
    \end{align*}
    woraus folgt:
    \begin{align*}
        2 \sqrt{a_{1} a_{2}}
        \leq a_{1} + a_{2}, 
    \end{align*}
    für alle \( a_{1}, a_{2} \geq 0 \)
\end{frame}



\begin{frame}{Beweis}
    
\end{frame}



\begin{frame}{Beweis}
    Seien \( x, y, z > 0 \) drei postive reelle Zahlen.
\end{frame}



\begin{frame}{Beweis}
    Seien \( x, y, z > 0 \) drei postive reelle Zahlen. Dann folgt:
\end{frame}



\begin{frame}{Beweis}
    Seien \( x, y, z > 0 \) drei postive reelle Zahlen. Dann folgt:
    \begin{align*}
        \frac{yz}{y + z} + \frac{xz}{x + z} + \frac{xy}{x + y}
    \end{align*}
\end{frame}



\begin{frame}{Beweis}
    Seien \( x, y, z > 0 \) drei postive reelle Zahlen. Dann folgt:
    \begin{align*}
        \frac{yz}{y + z} + \frac{xz}{x + z} + \frac{xy}{x + y}
        & \leq \frac{yz}{2\sqrt{yz}} + \frac{xz}{x + z} + \frac{xy}{x + y} \\
    \end{align*}
\end{frame}



\begin{frame}{Beweis}
    Seien \( x, y, z > 0 \) drei postive reelle Zahlen. Dann folgt:
    \begin{align*}
        \frac{yz}{y + z} + \frac{xz}{x + z} + \frac{xy}{x + y}
        & \leq \frac{yz}{2\sqrt{yz}} + \frac{xz}{x + z} + \frac{xy}{x + y} \\
        & \leq \frac{yz}{2\sqrt{yz}} + \frac{xz}{2\sqrt{xz}} + \frac{xy}{x + y}
    \end{align*}
\end{frame}



\begin{frame}{Beweis}
    Seien \( x, y, z > 0 \) drei postive reelle Zahlen. Dann folgt:
    \begin{align*}
        \frac{yz}{y + z} + \frac{xz}{x + z} + \frac{xy}{x + y}
        & \leq \frac{yz}{2\sqrt{yz}} + \frac{xz}{x + z} + \frac{xy}{x + y} \\
        & \leq \frac{yz}{2\sqrt{yz}} + \frac{xz}{2\sqrt{xz}} + \frac{xy}{x + y} \\
        & \leq \frac{yz}{2\sqrt{yz}} + \frac{xz}{2\sqrt{xz}} + \frac{xy}{2\sqrt{xy}} \\
    \end{align*}
\end{frame}



\begin{frame}{Beweis}
    Seien \( x, y, z > 0 \) drei postive reelle Zahlen. Dann folgt:
    \begin{align*}
        \frac{yz}{y + z} + \frac{xz}{x + z} + \frac{xy}{x + y}
        & \leq \frac{yz}{2\sqrt{yz}} + \frac{xz}{x + z} + \frac{xy}{x + y} \\
        & \leq \frac{yz}{2\sqrt{yz}} + \frac{xz}{2\sqrt{xz}} + \frac{xy}{x + y} \\
        & \leq \frac{yz}{2\sqrt{yz}} + \frac{xz}{2\sqrt{xz}} + \frac{xy}{2\sqrt{xy}} \\
        & = \frac{1}{2}\sqrt{yz} + \frac{1}{2}\sqrt{xz} + \frac{1}{2}\sqrt{xy}
    \end{align*}
\end{frame}



\begin{frame}{Beweis}
    Seien \( x, y, z > 0 \) drei postive reelle Zahlen. Dann folgt:
    \begin{align*}
        \frac{yz}{y + z} + \frac{xz}{x + z} + \frac{xy}{x + y}
        & \leq \frac{yz}{2\sqrt{yz}} + \frac{xz}{x + z} + \frac{xy}{x + y} \\
        & \leq \frac{yz}{2\sqrt{yz}} + \frac{xz}{2\sqrt{xz}} + \frac{xy}{x + y} \\
        & \leq \frac{yz}{2\sqrt{yz}} + \frac{xz}{2\sqrt{xz}} + \frac{xy}{2\sqrt{xy}} \\
        & = \frac{1}{2}\sqrt{yz} + \frac{1}{2}\sqrt{xz} + \frac{1}{2}\sqrt{xy} \\
        & \leq \frac{1}{2} \cdot \frac{1}{2} \left( y + z \right) + \frac{1}{2} \cdot \frac{1}{2} \left( x + z \right) + \frac{1}{2} \cdot \frac{1}{2} \left( x + y \right)
    \end{align*}
\end{frame}



\begin{frame}{Beweis}
    Seien \( x, y, z > 0 \) drei postive reelle Zahlen. Dann folgt:
    \begin{align*}
        \frac{yz}{y + z} + \frac{xz}{x + z} + \frac{xy}{x + y}
        & \leq \frac{yz}{2\sqrt{yz}} + \frac{xz}{x + z} + \frac{xy}{x + y} \\
        & \leq \frac{yz}{2\sqrt{yz}} + \frac{xz}{2\sqrt{xz}} + \frac{xy}{x + y} \\
        & \leq \frac{yz}{2\sqrt{yz}} + \frac{xz}{2\sqrt{xz}} + \frac{xy}{2\sqrt{xy}} \\
        & = \frac{1}{2}\sqrt{yz} + \frac{1}{2}\sqrt{xz} + \frac{1}{2}\sqrt{xy} \\
        & \leq \frac{1}{2} \cdot \frac{1}{2} \left( y + z \right) + \frac{1}{2} \cdot \frac{1}{2} \left( x + z \right) + \frac{1}{2} \cdot \frac{1}{2} \left( x + y \right) \\
        & = \frac{1}{4}y + \frac{1}{4}z + \frac{1}{4}x + \frac{1}{4}z + \frac{1}{4}x + \frac{1}{4}y
    \end{align*}
\end{frame}



\begin{frame}{Beweis}
    Seien \( x, y, z > 0 \) drei postive reelle Zahlen. Dann folgt:
    \begin{align*}
        \frac{yz}{y + z} + \frac{xz}{x + z} + \frac{xy}{x + y}
        & \leq \frac{yz}{2\sqrt{yz}} + \frac{xz}{x + z} + \frac{xy}{x + y} \\
        & \leq \frac{yz}{2\sqrt{yz}} + \frac{xz}{2\sqrt{xz}} + \frac{xy}{x + y} \\
        & \leq \frac{yz}{2\sqrt{yz}} + \frac{xz}{2\sqrt{xz}} + \frac{xy}{2\sqrt{xy}} \\
        & = \frac{1}{2}\sqrt{yz} + \frac{1}{2}\sqrt{xz} + \frac{1}{2}\sqrt{xy} \\
        & \leq \frac{1}{2} \cdot \frac{1}{2} \left( y + z \right) + \frac{1}{2} \cdot \frac{1}{2} \left( x + z \right) + \frac{1}{2} \cdot \frac{1}{2} \left( x + y \right) \\
        & = \frac{1}{4}y + \frac{1}{4}z + \frac{1}{4}x + \frac{1}{4}z + \frac{1}{4}x + \frac{1}{4}y \\
        & = \frac{1}{2}x + \frac{1}{2}y + \frac{1}{2}z
    \end{align*}
\end{frame}



\begin{frame}{Beweis}
    Seien \( x, y, z > 0 \) drei postive reelle Zahlen. Dann folgt:
    \begin{align*}
        \frac{yz}{y + z} + \frac{xz}{x + z} + \frac{xy}{x + y}
        & \leq \frac{yz}{2\sqrt{yz}} + \frac{xz}{x + z} + \frac{xy}{x + y} \\
        & \leq \frac{yz}{2\sqrt{yz}} + \frac{xz}{2\sqrt{xz}} + \frac{xy}{x + y} \\
        & \leq \frac{yz}{2\sqrt{yz}} + \frac{xz}{2\sqrt{xz}} + \frac{xy}{2\sqrt{xy}} \\
        & = \frac{1}{2}\sqrt{yz} + \frac{1}{2}\sqrt{xz} + \frac{1}{2}\sqrt{xy} \\
        & \leq \frac{1}{2} \cdot \frac{1}{2} \left( y + z \right) + \frac{1}{2} \cdot \frac{1}{2} \left( x + z \right) + \frac{1}{2} \cdot \frac{1}{2} \left( x + y \right) \\
        & = \frac{1}{4}y + \frac{1}{4}z + \frac{1}{4}x + \frac{1}{4}z + \frac{1}{4}x + \frac{1}{4}y \\
        & = \frac{1}{2}x + \frac{1}{2}y + \frac{1}{2}z \\
        & = \frac{1}{2} \left( x + y + z \right).
    \end{align*}
\end{frame}
% ============================================================

\end{document}