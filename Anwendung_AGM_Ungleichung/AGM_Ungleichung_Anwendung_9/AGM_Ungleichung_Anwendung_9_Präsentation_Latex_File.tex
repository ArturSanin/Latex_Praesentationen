\documentclass[10pt]{beamer}

\title{}
\author{Artur's \( \oint \) Mathematikstübchen}
\date{}


% ===== Packages =========
\usepackage[utf8]{inputenc}

\usepackage[natbibapa]{apacite}
\bibliographystyle{apacite}
\usepackage[ngerman]{babel}
\usepackage{graphicx}
\usepackage{fancyhdr}
\usepackage{amsmath}
\usepackage{amssymb}
\usepackage{graphicx}
\usepackage{MnSymbol}
\usepackage{enumitem}
\usepackage{amsthm}
\usepackage{mleftright}
\usepackage{dsfont}
\usepackage{tikz-cd}


\def\bC{\mathbb{C}}
\def\bD{\mathbb{D}}               
\def\bE{\mathbb{E}}
\def\bG{\mathbb{G}}
\def\bN{\mathbb{N}}
\def\bP{\mathbb{P}}
\def\bQ{\mathbb{Q}}
\def\bR{\mathbb{R}}
\def\bBarR{\bar{\mathbb{R}}}
\def\bY{\mathbb{Y}}



\def\mA{\mathcal{A}}
\def\mB{\mathcal{B}}
\def\mD{\mathcal{D}}
\def\mE{\mathcal{E}}
\def\mF{\mathcal{F}}
\def\mG{\mathcal{G}}
\def\mH{\mathcal{H}}
\def\mL{\mathcal{L}}
\def\mN{\mathcal{N}}
\def\mP{\mathcal{P}}
\def\mS{\mathcal{S}}
\def\mT{\mathcal{T}}
\def\mX{\mathcal{X}}
\def\mY{\mathcal{Y}}



\usetheme{Madrid}



% ======================== Beginn Document ========================

\begin{document}





% ======================== Begrüßung ==================

\begin{frame}
    \begin{center}
        \textbf{\huge Willkommen in der guten Stube \newline \newline :D}
    \end{center}
\end{frame}
% =====================================================



% ======================== Präsentation ==================

\begin{frame}
    \begin{alertblock}{Aufgabe}
        Seien \( x, y, z > 0 \) drei positive reelle Zahlen. Man zeige die Gültigkeit der folgenden Abschätzung:
        \begin{align*}
            \left( \frac{1}{y} + \frac{1}{z} \right) \left( \frac{1}{x} + \frac{1}{z} \right) \left( \frac{1}{x} + \frac{1}{y} \right)
            \geq \frac{8}{xyz}.
        \end{align*}
    \end{alertblock}
\end{frame}



\begin{frame}{Hilfsabschätzung}
    
\end{frame}



\begin{frame}{Hilfsabschätzung}
    Für den Beweis verwenden wir die Ungleichung zwischen dem geometrischen und arithmetischen Mittel:
\end{frame}



\begin{frame}{Hilfsabschätzung}
    Für den Beweis verwenden wir die Ungleichung zwischen dem geometrischen und arithmetischen Mittel:
    \begin{block}{Hilfsabschätzung}
        Für alle \( a_{1}, \ldots, a_{p} \geq 0 \), \( p \in \bN \), gilt die Abschätzung:
        \begin{align*}
            \sqrt[p]{a_{1} \cdot \ldots \cdot a_{p}} 
            \leq \frac{a_{1} + \ldots + a_{p}}{p}.
        \end{align*}
    \end{block}
\end{frame}



\begin{frame}{Hilfsabschätzung}
    Für den Beweis verwenden wir die Ungleichung zwischen dem geometrischen und arithmetischen Mittel:
    \begin{block}{Hilfsabschätzung}
        Für alle \( a_{1}, \ldots, a_{p} \geq 0 \), \( p \in \bN \), gilt die Abschätzung:
        \begin{align*}
            \sqrt[p]{a_{1} \cdot \ldots \cdot a_{p}} 
            \leq \frac{a_{1} + \ldots + a_{p}}{p}.
        \end{align*}
    \end{block}
    Insbesondere gilt für \( p = 2 \):   
\end{frame}



\begin{frame}{Hilfsabschätzung}
    Für den Beweis verwenden wir die Ungleichung zwischen dem geometrischen und arithmetischen Mittel:
    \begin{block}{Hilfsabschätzung}
        Für alle \( a_{1}, \ldots, a_{p} \geq 0 \), \( p \in \bN \), gilt die Abschätzung:
        \begin{align*}
            \sqrt[p]{a_{1} \cdot \ldots \cdot a_{p}} 
            \leq \frac{a_{1} + \ldots + a_{p}}{p}.
        \end{align*}
    \end{block}
    Insbesondere gilt für \( p = 2 \): 
    \begin{align*}
        \sqrt{a_{1} a_{2}}
        \leq \frac{1}{2} \left( a_{1} + a_{2} \right)
    \end{align*}
    woraus folgt:
    \begin{align*}
        2 \sqrt{a_{1} a_{2}}
        \leq a_{1} + a_{2}, 
    \end{align*}
    für alle \( a_{1}, a_{2} \geq 0 \)
\end{frame}



\begin{frame}{Beweis}
    
\end{frame}



\begin{frame}{Beweis}
    Seien \(x, y, z > 0\) drei positive reelle Zahlen.
\end{frame}



\begin{frame}{Beweis}
    Seien \(x, y, z > 0\) drei positive reelle Zahlen. Dann gilt:
\end{frame}



\begin{frame}{Beweis}
    Seien \(x, y, z > 0\) drei positive reelle Zahlen. Dann gilt:
    \begin{align*}
        \left( \frac{1}{y} + \frac{1}{z} \right) \left( \frac{1}{x} + \frac{1}{z} \right) \left( \frac{1}{x} + \frac{1}{y} \right)
    \end{align*}
\end{frame}



\begin{frame}{Beweis}
    Seien \(x, y, z > 0\) drei positive reelle Zahlen. Dann gilt:
    \begin{align*}
        \left( \frac{1}{y} + \frac{1}{z} \right) \left( \frac{1}{x} + \frac{1}{z} \right) \left( \frac{1}{x} + \frac{1}{y} \right)
        & \geq 2\sqrt{\frac{1}{yz}} \left( \frac{1}{x} + \frac{1}{z} \right) \left( \frac{1}{x} + \frac{1}{y} \right)
    \end{align*}
\end{frame}



\begin{frame}{Beweis}
    Seien \(x, y, z > 0\) drei positive reelle Zahlen. Dann gilt:
    \begin{align*}
        \left( \frac{1}{y} + \frac{1}{z} \right) \left( \frac{1}{x} + \frac{1}{z} \right) \left( \frac{1}{x} + \frac{1}{y} \right)
        & \geq 2\sqrt{\frac{1}{yz}} \left( \frac{1}{x} + \frac{1}{z} \right) \left( \frac{1}{x} + \frac{1}{y} \right) \\
        & \geq 2\sqrt{\frac{1}{yz}} 2\sqrt{\frac{1}{xz}} \left( \frac{1}{x} + \frac{1}{y} \right)
    \end{align*}
\end{frame}



\begin{frame}{Beweis}
    Seien \(x, y, z > 0\) drei positive reelle Zahlen. Dann gilt:
    \begin{align*}
        \left( \frac{1}{y} + \frac{1}{z} \right) \left( \frac{1}{x} + \frac{1}{z} \right) \left( \frac{1}{x} + \frac{1}{y} \right)
        & \geq 2\sqrt{\frac{1}{yz}} \left( \frac{1}{x} + \frac{1}{z} \right) \left( \frac{1}{x} + \frac{1}{y} \right) \\
        & \geq 2\sqrt{\frac{1}{yz}} 2\sqrt{\frac{1}{xz}} \left( \frac{1}{x} + \frac{1}{y} \right) \\
        & \geq 2\sqrt{\frac{1}{yz}} 2\sqrt{\frac{1}{xz}} 2\sqrt{\frac{1}{xy}}
    \end{align*}
\end{frame}



\begin{frame}{Beweis}
    Seien \(x, y, z > 0\) drei positive reelle Zahlen. Dann gilt:
    \begin{align*}
        \left( \frac{1}{y} + \frac{1}{z} \right) \left( \frac{1}{x} + \frac{1}{z} \right) \left( \frac{1}{x} + \frac{1}{y} \right)
        & \geq 2\sqrt{\frac{1}{yz}} \left( \frac{1}{x} + \frac{1}{z} \right) \left( \frac{1}{x} + \frac{1}{y} \right) \\
        & \geq 2\sqrt{\frac{1}{yz}} 2\sqrt{\frac{1}{xz}} \left( \frac{1}{x} + \frac{1}{y} \right) \\
        & \geq 2\sqrt{\frac{1}{yz}} 2\sqrt{\frac{1}{xz}} 2\sqrt{\frac{1}{xy}} \\
        & = 8 \frac{1}{\sqrt{yz}} \frac{1}{\sqrt{xz}} \frac{1}{\sqrt{xy}}
    \end{align*}
\end{frame}



\begin{frame}{Beweis}
    Seien \(x, y, z > 0\) drei positive reelle Zahlen. Dann gilt:
    \begin{align*}
        \left( \frac{1}{y} + \frac{1}{z} \right) \left( \frac{1}{x} + \frac{1}{z} \right) \left( \frac{1}{x} + \frac{1}{y} \right)
        & \geq 2\sqrt{\frac{1}{yz}} \left( \frac{1}{x} + \frac{1}{z} \right) \left( \frac{1}{x} + \frac{1}{y} \right) \\
        & \geq 2\sqrt{\frac{1}{yz}} 2\sqrt{\frac{1}{xz}} \left( \frac{1}{x} + \frac{1}{y} \right) \\
        & \geq 2\sqrt{\frac{1}{yz}} 2\sqrt{\frac{1}{xz}} 2\sqrt{\frac{1}{xy}} \\
        & = 8 \frac{1}{\sqrt{yz}} \frac{1}{\sqrt{xz}} \frac{1}{\sqrt{xy}} \\
        & = \frac{8}{\sqrt{yz xz xy}}
    \end{align*}
\end{frame}



\begin{frame}{Beweis}
    Seien \(x, y, z > 0\) drei positive reelle Zahlen. Dann gilt:
    \begin{align*}
        \left( \frac{1}{y} + \frac{1}{z} \right) \left( \frac{1}{x} + \frac{1}{z} \right) \left( \frac{1}{x} + \frac{1}{y} \right)
        & \geq 2\sqrt{\frac{1}{yz}} \left( \frac{1}{x} + \frac{1}{z} \right) \left( \frac{1}{x} + \frac{1}{y} \right) \\
        & \geq 2\sqrt{\frac{1}{yz}} 2\sqrt{\frac{1}{xz}} \left( \frac{1}{x} + \frac{1}{y} \right) \\
        & \geq 2\sqrt{\frac{1}{yz}} 2\sqrt{\frac{1}{xz}} 2\sqrt{\frac{1}{xy}} \\
        & = 8 \frac{1}{\sqrt{yz}} \frac{1}{\sqrt{xz}} \frac{1}{\sqrt{xy}} \\
        & = \frac{8}{\sqrt{yz xz xy}} \\
        & = \frac{8}{\sqrt{x^{2} y^{2} z^{2}}}
    \end{align*}
\end{frame}



\begin{frame}{Beweis}
    Seien \(x, y, z > 0\) drei positive reelle Zahlen. Dann gilt:
    \begin{align*}
        \left( \frac{1}{y} + \frac{1}{z} \right) \left( \frac{1}{x} + \frac{1}{z} \right) \left( \frac{1}{x} + \frac{1}{y} \right)
        & \geq 2\sqrt{\frac{1}{yz}} \left( \frac{1}{x} + \frac{1}{z} \right) \left( \frac{1}{x} + \frac{1}{y} \right) \\
        & \geq 2\sqrt{\frac{1}{yz}} 2\sqrt{\frac{1}{xz}} \left( \frac{1}{x} + \frac{1}{y} \right) \\
        & \geq 2\sqrt{\frac{1}{yz}} 2\sqrt{\frac{1}{xz}} 2\sqrt{\frac{1}{xy}} \\
        & = 8 \frac{1}{\sqrt{yz}} \frac{1}{\sqrt{xz}} \frac{1}{\sqrt{xy}} \\
        & = \frac{8}{\sqrt{yz xz xy}} \\
        & = \frac{8}{\sqrt{x^{2} y^{2} z^{2}}} \\
        & = \frac{8}{\sqrt{x^{2}} \sqrt{y^{2}} \sqrt{z^{2}}}
    \end{align*}
\end{frame}



\begin{frame}{Beweis}
    Seien \(x, y, z > 0\) drei positive reelle Zahlen. Dann gilt:
    \begin{align*}
        \left( \frac{1}{y} + \frac{1}{z} \right) \left( \frac{1}{x} + \frac{1}{z} \right) \left( \frac{1}{x} + \frac{1}{y} \right)
        & \geq 2\sqrt{\frac{1}{yz}} \left( \frac{1}{x} + \frac{1}{z} \right) \left( \frac{1}{x} + \frac{1}{y} \right) \\
        & \geq 2\sqrt{\frac{1}{yz}} 2\sqrt{\frac{1}{xz}} \left( \frac{1}{x} + \frac{1}{y} \right) \\
        & \geq 2\sqrt{\frac{1}{yz}} 2\sqrt{\frac{1}{xz}} 2\sqrt{\frac{1}{xy}} \\
        & = 8 \frac{1}{\sqrt{yz}} \frac{1}{\sqrt{xz}} \frac{1}{\sqrt{xy}} \\
        & = \frac{8}{\sqrt{yz xz xy}} \\
        & = \frac{8}{\sqrt{x^{2} y^{2} z^{2}}} \\
        & = \frac{8}{\sqrt{x^{2}} \sqrt{y^{2}} \sqrt{z^{2}}}
        = \frac{8}{xyz}.
    \end{align*}
\end{frame}
% ============================================================
\end{document}