\documentclass[10pt]{beamer}

\title{}
\author{Artur's Mathematikstübchen}
\date{}


% ===== Packages =========
\usepackage[utf8]{inputenc}

\usepackage[natbibapa]{apacite}
\bibliographystyle{apacite}
\usepackage[ngerman]{babel}
\usepackage{graphicx}
\usepackage{fancyhdr}
\usepackage{amsmath}
\usepackage{amssymb}
\usepackage{graphicx}
\usepackage{MnSymbol}
\usepackage{enumitem}
\usepackage{amsthm}
\usepackage{mleftright}
\usepackage{dsfont}
\usepackage{tikz-cd}


\def\bD{\mathbb{D}}               
\def\bE{\mathbb{E}}
\def\bG{\mathbb{G}}
\def\bN{\mathbb{N}}
\def\bP{\mathbb{P}}
\def\bQ{\mathbb{Q}}
\def\bR{\mathbb{R}}
\def\bBarR{\bar{\mathbb{R}}}
\def\bY{\mathbb{Y}}



\def\mA{\mathcal{A}}
\def\mB{\mathcal{B}}
\def\mD{\mathcal{D}}
\def\mE{\mathcal{E}}
\def\mF{\mathcal{F}}
\def\mG{\mathcal{G}}
\def\mH{\mathcal{H}}
\def\mL{\mathcal{L}}
\def\mN{\mathcal{N}}
\def\mP{\mathcal{P}}
\def\mS{\mathcal{S}}
\def\mT{\mathcal{T}}
\def\mX{\mathcal{X}}
\def\mY{\mathcal{Y}}



\usetheme{Madrid}



% ======================== Beginn Document ========================

\begin{document}





% ======================== Begrüßung ==================

\begin{frame}
    \begin{center}
        \textbf{\huge Willkommen in der guten Stube \newline \newline :D}
    \end{center}
\end{frame}
% =====================================================



% ======================== Präsentation ==================

\begin{frame}
    \begin{alertblock}{Aufgabe}
        Sei \( n \in \bN \) eine beliebige natürliche Zahl und \( x > 0 \) eine beliebige positive reelle Zahl. Man zeige die Gültigkeit der folgenden Abschätzung:
        \begin{align*}
            \frac{x^{n}}{1 + x + x^{2} + \ldots + x^{2n}}
            \leq \frac{1}{2n + 1}.
        \end{align*}
    \end{alertblock}
\end{frame}



\begin{frame}{Hilfsabschätzung}
    
\end{frame}



\begin{frame}{Hilfsabschätzung}
    Für den Beweis verwenden wir die Ungleichung zwischen dem geometrischen und arithmetischen Mittel:
\end{frame}



\begin{frame}{Hilfsabschätzung}
    Für den Beweis verwenden wir die Ungleichung zwischen dem geometrischen und arithmetischen Mittel:
    \begin{block}{Hilfsabschätzung}
        Für alle \( a_{1}, \ldots, a_{p} \geq 0 \), \( p \in \bN \), gilt die Abschätzung:
        \begin{align*}
            \sqrt[p]{a_{1} \cdot \ldots \cdot a_{p}} 
            \leq \frac{a_{1} + \ldots + a_{p}}{p}.
        \end{align*}
    \end{block}
\end{frame}



\begin{frame}{Hilfsabschätzung}
    Für den Beweis verwenden wir die Ungleichung zwischen dem geometrischen und arithmetischen Mittel:
    \begin{block}{Hilfsabschätzung}
        Für alle \( a_{1}, \ldots, a_{p} \geq 0 \), \( p \in \bN \), gilt die Abschätzung:
        \begin{align*}
            \sqrt[p]{a_{1} \cdot \ldots \cdot a_{p}} 
            \leq \frac{a_{1} + \ldots + a_{p}}{p}.
        \end{align*}
    \end{block}
    Insbesondere folgt hieraus die Abschätzung:
    \begin{align*}
        p \cdot \sqrt[p]{a_{1} \cdot \ldots \cdot a_{p}}
        \leq a_{1} + \ldots + a_{p}.
    \end{align*}
\end{frame}



\begin{frame}{Hilfsabschätzung}
    Für den Beweis verwenden wir die Ungleichung zwischen dem geometrischen und arithmetischen Mittel:
    \begin{block}{Hilfsabschätzung}
        Für alle \( a_{1}, \ldots, a_{p} \geq 0 \), \( p \in \bN \), gilt die Abschätzung:
        \begin{align*}
            \sqrt[p]{a_{1} \cdot \ldots \cdot a_{p}} 
            \leq \frac{a_{1} + \ldots + a_{p}}{p}.
        \end{align*}
    \end{block}
    Insbesondere folgt hieraus die Abschätzung:
    \begin{align*}
        p \cdot \sqrt[p]{a_{1} \cdot \ldots \cdot a_{p}}
        \leq a_{1} + \ldots + a_{p}.
    \end{align*}
    Weiter gilt für alle \( n \in \bN \) die Gauß'sche Summenformel:   
\end{frame}



\begin{frame}{Hilfsabschätzung}
    Für den Beweis verwenden wir die Ungleichung zwischen dem geometrischen und arithmetischen Mittel:
    \begin{block}{Hilfsabschätzung}
        Für alle \( a_{1}, \ldots, a_{p} \geq 0 \), \( p \in \bN \), gilt die Abschätzung:
        \begin{align*}
            \sqrt[p]{a_{1} \cdot \ldots \cdot a_{p}} 
            \leq \frac{a_{1} + \ldots + a_{p}}{p}.
        \end{align*}
    \end{block}
    Insbesondere folgt hieraus die Abschätzung:
    \begin{align*}
        p \cdot \sqrt[p]{a_{1} \cdot \ldots \cdot a_{p}}
        \leq a_{1} + \ldots + a_{p}.
    \end{align*}
    Weiter gilt für alle \( n \in \bN \) die Gauß'sche Summenformel:
    \begin{align*}
        \sum_{k = 1}^{n} k
        = \frac{n \cdot \left( n + 1 \right)}{2}.
    \end{align*}
\end{frame}



\begin{frame}{Beweis}
    
\end{frame}



\begin{frame}{Beweis}
    Sei \( n \in \bN \) eine beliebige natürliche und \( x > 0 \) eine beliebige positive reelle Zahl.    
\end{frame}



\begin{frame}{Beweis}
    Sei \( n \in \bN \) eine beliebige natürliche und \( x > 0 \) eine beliebige positive reelle Zahl. Dann schätzen wir ab:    
\end{frame}



\begin{frame}{Beweis}
    Sei \( n \in \bN \) eine beliebige natürliche und \( x > 0 \) eine beliebige positive reelle Zahl. Dann schätzen wir ab:
    \begin{align*}
        \frac{x^{n}}{1 + x + x^{2} + \ldots + x^{2n}}
    \end{align*}
\end{frame}



\begin{frame}{Beweis}
    Sei \( n \in \bN \) eine beliebige natürliche und \( x > 0 \) eine beliebige positive reelle Zahl. Dann schätzen wir ab:
    \begin{align*}
        \frac{x^{n}}{1 + x + x^{2} + \ldots + x^{2n}}
        & \leq \frac{x^{n}}{\left( 2n + 1 \right) \cdot \sqrt[2n + 1]{1 \cdot x \cdot x^{2} \cdot \ldots \cdot x^{2n}}}
    \end{align*}
\end{frame}



\begin{frame}{Beweis}
    Sei \( n \in \bN \) eine beliebige natürliche und \( x > 0 \) eine beliebige positive reelle Zahl. Dann schätzen wir ab:
    \begin{align*}
        \frac{x^{n}}{1 + x + x^{2} + \ldots + x^{2n}}
        & \leq \frac{x^{n}}{\left( 2n + 1 \right) \cdot \sqrt[2n + 1]{1 \cdot x \cdot x^{2} \cdot \ldots \cdot x^{2n}}} \\
        & \leq \frac{x^{n}}{\left( 2n + 1 \right) \cdot \sqrt[2n + 1]{x^{1 + 2 + \ldots + 2n}}}
    \end{align*}
\end{frame}



\begin{frame}{Beweis}
    Sei \( n \in \bN \) eine beliebige natürliche und \( x > 0 \) eine beliebige positive reelle Zahl. Dann schätzen wir ab:
    \begin{align*}
        \frac{x^{n}}{1 + x + x^{2} + \ldots + x^{2n}}
        & \leq \frac{x^{n}}{\left( 2n + 1 \right) \cdot \sqrt[2n + 1]{1 \cdot x \cdot x^{2} \cdot \ldots \cdot x^{2n}}} \\
        & \leq \frac{x^{n}}{\left( 2n + 1 \right) \cdot \sqrt[2n + 1]{x^{1 + 2 + \ldots + 2n}}} \\ 
        & = \frac{x^{n}}{\left( 2n + 1 \right) \cdot \sqrt[2n + 1]{x^{\frac{2n \cdot \left( 2n + 1 \right)}{2}}}}
    \end{align*}
\end{frame}



\begin{frame}{Beweis}
    Sei \( n \in \bN \) eine beliebige natürliche und \( x > 0 \) eine beliebige positive reelle Zahl. Dann schätzen wir ab:
    \begin{align*}
        \frac{x^{n}}{1 + x + x^{2} + \ldots + x^{2n}}
        & \leq \frac{x^{n}}{\left( 2n + 1 \right) \cdot \sqrt[2n + 1]{1 \cdot x \cdot x^{2} \cdot \ldots \cdot x^{2n}}} \\
        & \leq \frac{x^{n}}{\left( 2n + 1 \right) \cdot \sqrt[2n + 1]{x^{1 + 2 + \ldots + 2n}}} \\ 
        & = \frac{x^{n}}{\left( 2n + 1 \right) \cdot \sqrt[2n + 1]{x^{\frac{2n \cdot \left( 2n + 1 \right)}{2}}}} \\
        & = \frac{x^{n}}{\left( 2n + 1 \right) \cdot x^{\frac{2n \cdot \left( 2n + 1 \right)}{2 \cdot \left( 2n + 1 \right)}}}
    \end{align*}
\end{frame}



\begin{frame}{Beweis}
    Sei \( n \in \bN \) eine beliebige natürliche und \( x > 0 \) eine beliebige positive reelle Zahl. Dann schätzen wir ab:
    \begin{align*}
        \frac{x^{n}}{1 + x + x^{2} + \ldots + x^{2n}}
        & \leq \frac{x^{n}}{\left( 2n + 1 \right) \cdot \sqrt[2n + 1]{1 \cdot x \cdot x^{2} \cdot \ldots \cdot x^{2n}}} \\
        & \leq \frac{x^{n}}{\left( 2n + 1 \right) \cdot \sqrt[2n + 1]{x^{1 + 2 + \ldots + 2n}}} \\ 
        & = \frac{x^{n}}{\left( 2n + 1 \right) \cdot \sqrt[2n + 1]{x^{\frac{2n \cdot \left( 2n + 1 \right)}{2}}}} \\
        & = \frac{x^{n}}{\left( 2n + 1 \right) \cdot x^{\frac{2n \cdot \left( 2n + 1 \right)}{2 \cdot \left( 2n + 1 \right)}}} \\
        & = \frac{x^{n}}{\left( 2n + 1 \right) \cdot x^{n}}
    \end{align*}
\end{frame}



\begin{frame}{Beweis}
    Sei \( n \in \bN \) eine beliebige natürliche und \( x > 0 \) eine beliebige positive reelle Zahl. Dann schätzen wir ab:
    \begin{align*}
        \frac{x^{n}}{1 + x + x^{2} + \ldots + x^{2n}}
        & \leq \frac{x^{n}}{\left( 2n + 1 \right) \cdot \sqrt[2n + 1]{1 \cdot x \cdot x^{2} \cdot \ldots \cdot x^{2n}}} \\
        & \leq \frac{x^{n}}{\left( 2n + 1 \right) \cdot \sqrt[2n + 1]{x^{1 + 2 + \ldots + 2n}}} \\ 
        & = \frac{x^{n}}{\left( 2n + 1 \right) \cdot \sqrt[2n + 1]{x^{\frac{2n \cdot \left( 2n + 1 \right)}{2}}}} \\
        & = \frac{x^{n}}{\left( 2n + 1 \right) \cdot x^{\frac{2n \cdot \left( 2n + 1 \right)}{2 \cdot \left( 2n + 1 \right)}}} \\
        & = \frac{x^{n}}{\left( 2n + 1 \right) \cdot x^{n}} \\
        & = \frac{1}{2n + 1}.
    \end{align*}
\end{frame}
% ============================================================

\end{document}