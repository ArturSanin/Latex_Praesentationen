\documentclass[10pt]{beamer}

\title{}
\author{Artur's Mathematikstübchen}
\date{}


% ===== Packages =========
\usepackage[utf8]{inputenc}

\usepackage[natbibapa]{apacite}
\bibliographystyle{apacite}
\usepackage[ngerman]{babel}
\usepackage{graphicx}
\usepackage{fancyhdr}
\usepackage{amsmath}
\usepackage{amssymb}
\usepackage{graphicx}
\usepackage{MnSymbol}
\usepackage{enumitem}
\usepackage{amsthm}
\usepackage{mleftright}
\usepackage{dsfont}
\usepackage{tikz-cd}


\def\bD{\mathbb{D}}               
\def\bE{\mathbb{E}}
\def\bG{\mathbb{G}}
\def\bN{\mathbb{N}}
\def\bP{\mathbb{P}}
\def\bQ{\mathbb{Q}}
\def\bR{\mathbb{R}}
\def\bBarR{\bar{\mathbb{R}}}
\def\bY{\mathbb{Y}}



\def\mA{\mathcal{A}}
\def\mB{\mathcal{B}}
\def\mD{\mathcal{D}}
\def\mE{\mathcal{E}}
\def\mF{\mathcal{F}}
\def\mG{\mathcal{G}}
\def\mH{\mathcal{H}}
\def\mL{\mathcal{L}}
\def\mN{\mathcal{N}}
\def\mP{\mathcal{P}}
\def\mS{\mathcal{S}}
\def\mT{\mathcal{T}}
\def\mX{\mathcal{X}}
\def\mY{\mathcal{Y}}



\usetheme{Madrid}



% ======================== Beginn Document ========================

\begin{document}





% ======================== Begrüßung ==================

\begin{frame}
    \begin{center}
        \textbf{\huge Willkommen in der guten Stube \newline \newline :D}
    \end{center}
\end{frame}
% =====================================================



% ======================== Präsentation ==================

\begin{frame}
    \begin{alertblock}{Aufgabe}
        Seien \( a, b, c, d, e \geq 0 \) nicht-negative reelle Zahlen. Man zeige die Abschätzung:
        \begin{align*}
            \left( a^{2} + b^{2} + c^{2} + d^{2} + e^{2} \right) \cdot \left( a^{3} + b^{3} + c^{3} + d^{3} + e^{3} \right)
            \geq 25abcde.
        \end{align*}
    \end{alertblock}
\end{frame}



\begin{frame}{Hilfsabschätzung}
    
\end{frame}



\begin{frame}{Hilfsabschätzung}
    Für den Beweis verwenden wir die Ungleichung zwischen dem geometrischen und arithmetischen Mittel:
\end{frame}



\begin{frame}{Hilfsabschätzung}
    Für den Beweis verwenden wir die Ungleichung zwischen dem geometrischen und arithmetischen Mittel:
    \begin{block}{Hilfsabschätzung}
        Für alle \( a_{1}, \ldots, a_{p} \geq 0 \), \( p \in \bN \), gilt die Abschätzung:
        \begin{align*}
            \sqrt[p]{a_{1} \cdot \ldots \cdot a_{p}} 
            \leq \frac{a_{1} + \ldots + a_{p}}{p}.
        \end{align*}
    \end{block}
\end{frame}



\begin{frame}{Hilfsabschätzung}
    Für den Beweis verwenden wir die Ungleichung zwischen dem geometrischen und arithmetischen Mittel:
    \begin{block}{Hilfsabschätzung}
        Für alle \( a_{1}, \ldots, a_{p} \geq 0 \), \( p \in \bN \), gilt die Abschätzung:
        \begin{align*}
            \sqrt[p]{a_{1} \cdot \ldots \cdot a_{p}} 
            \leq \frac{a_{1} + \ldots + a_{p}}{p}.
        \end{align*}
    \end{block}
    Wir verwenden die Ungleichung für den Fall \( p = 5 \):
\end{frame}



\begin{frame}{Hilfsabschätzung}
    Für den Beweis verwenden wir die Ungleichung zwischen dem geometrischen und arithmetischen Mittel:
    \begin{block}{Hilfsabschätzung}
        Für alle \( a_{1}, \ldots, a_{p} \geq 0 \), \( p \in \bN \), gilt die Abschätzung:
        \begin{align*}
            \sqrt[p]{a_{1} \cdot \ldots \cdot a_{p}} 
            \leq \frac{a_{1} + \ldots + a_{p}}{p}.
        \end{align*}
    \end{block}
    Wir verwenden die Ungleichung für den Fall \( p = 5 \):
    \begin{align*}
        \sqrt[5]{a_{1} a_{2} a_{3} a_{4} a_{5}} 
        \leq \frac{a_{1} + a_{2} + a_{3} + a_{4} + a_{5}}{5}.
    \end{align*}
\end{frame}



\begin{frame}{Beweis}
    
\end{frame}



\begin{frame}{Beweis}
    Seien \( a, b, c, d, e \geq 0 \) beliebige nicht-negative reelle Zahlen. 
\end{frame}



\begin{frame}{Beweis}
    Seien \( a, b, c, d, e \geq 0 \) beliebige nicht-negative reelle Zahlen. Wir schätzen wie folgt ab: 
\end{frame}



\begin{frame}{Beweis}
    Seien \( a, b, c, d, e \geq 0 \) beliebige nicht-negative reelle Zahlen. Wir schätzen wie folgt ab:
    \begin{align*}
        \left( a^{2} + b^{2} + c^{2} + d^{2} + e^{2} \right) \cdot \left( a^{3} + b^{3} + c^{3} + d^{3} + e^{3} \right)
    \end{align*}
\end{frame}



\begin{frame}{Beweis}
    Seien \( a, b, c, d, e \geq 0 \) beliebige nicht-negative reelle Zahlen. Wir schätzen wie folgt ab:
    \begin{align*}
        & \left( a^{2} + b^{2} + c^{2} + d^{2} + e^{2} \right) \cdot \left( a^{3} + b^{3} + c^{3} + d^{3} + e^{3} \right) \\
        & = 25 \cdot \frac{\left( a^{2} + b^{2} + c^{2} + d^{2} + e^{2} \right)}{5} \cdot \frac{\left( a^{3} + b^{3} + c^{3} + d^{3} + e^{3} \right)}{5}
    \end{align*}
\end{frame}



\begin{frame}{Beweis}
    Seien \( a, b, c, d, e \geq 0 \) beliebige nicht-negative reelle Zahlen. Wir schätzen wie folgt ab:
    \begin{align*}
        & \left( a^{2} + b^{2} + c^{2} + d^{2} + e^{2} \right) \cdot \left( a^{3} + b^{3} + c^{3} + d^{3} + e^{3} \right) \\
        & = 25 \cdot \frac{\left( a^{2} + b^{2} + c^{2} + d^{2} + e^{2} \right)}{5} \cdot \frac{\left( a^{3} + b^{3} + c^{3} + d^{3} + e^{3} \right)}{5} \\
        & \geq 25 \sqrt[5]{a^{2} b^{2} c^{2} d^{2} e^{2}} \cdot \frac{\left( a^{3} + b^{3} + c^{3} + d^{3} + e^{3} \right)}{5}
    \end{align*}
\end{frame}



\begin{frame}{Beweis}
    Seien \( a, b, c, d, e \geq 0 \) beliebige nicht-negative reelle Zahlen. Wir schätzen wie folgt ab:
    \begin{align*}
        & \left( a^{2} + b^{2} + c^{2} + d^{2} + e^{2} \right) \cdot \left( a^{3} + b^{3} + c^{3} + d^{3} + e^{3} \right) \\
        & = 25 \cdot \frac{\left( a^{2} + b^{2} + c^{2} + d^{2} + e^{2} \right)}{5} \cdot \frac{\left( a^{3} + b^{3} + c^{3} + d^{3} + e^{3} \right)}{5} \\
        & \geq 25 \sqrt[5]{a^{2} b^{2} c^{2} d^{2} e^{2}} \cdot \frac{\left( a^{3} + b^{3} + c^{3} + d^{3} + e^{3} \right)}{5} \\
        & \geq 25 \sqrt[5]{a^{2} b^{2} c^{2} d^{2} e^{2}} \cdot \sqrt[5]{a^{3} b^{3} c^{3} d^{3} e^{3}}
    \end{align*}
\end{frame}



\begin{frame}{Beweis}
    Seien \( a, b, c, d, e \geq 0 \) beliebige nicht-negative reelle Zahlen. Wir schätzen wie folgt ab:
    \begin{align*}
        & \left( a^{2} + b^{2} + c^{2} + d^{2} + e^{2} \right) \cdot \left( a^{3} + b^{3} + c^{3} + d^{3} + e^{3} \right) \\
        & = 25 \cdot \frac{\left( a^{2} + b^{2} + c^{2} + d^{2} + e^{2} \right)}{5} \cdot \frac{\left( a^{3} + b^{3} + c^{3} + d^{3} + e^{3} \right)}{5} \\
        & \geq 25 \sqrt[5]{a^{2} b^{2} c^{2} d^{2} e^{2}} \cdot \frac{\left( a^{3} + b^{3} + c^{3} + d^{3} + e^{3} \right)}{5} \\
        & \geq 25 \sqrt[5]{a^{2} b^{2} c^{2} d^{2} e^{2}} \cdot \sqrt[5]{a^{3} b^{3} c^{3} d^{3} e^{3}} \\
        & \geq 25 \sqrt[5]{\left( a b c d e \right)^{2}} \cdot \sqrt[5]{\left( a b c d e \right)^{3}}
    \end{align*}
\end{frame}



\begin{frame}{Beweis}
    Seien \( a, b, c, d, e \geq 0 \) beliebige nicht-negative reelle Zahlen. Wir schätzen wie folgt ab:
    \begin{align*}
        & \left( a^{2} + b^{2} + c^{2} + d^{2} + e^{2} \right) \cdot \left( a^{3} + b^{3} + c^{3} + d^{3} + e^{3} \right) \\
        & = 25 \cdot \frac{\left( a^{2} + b^{2} + c^{2} + d^{2} + e^{2} \right)}{5} \cdot \frac{\left( a^{3} + b^{3} + c^{3} + d^{3} + e^{3} \right)}{5} \\
        & \geq 25 \sqrt[5]{a^{2} b^{2} c^{2} d^{2} e^{2}} \cdot \frac{\left( a^{3} + b^{3} + c^{3} + d^{3} + e^{3} \right)}{5} \\
        & \geq 25 \sqrt[5]{a^{2} b^{2} c^{2} d^{2} e^{2}} \cdot \sqrt[5]{a^{3} b^{3} c^{3} d^{3} e^{3}} \\
        & \geq 25 \sqrt[5]{\left( a b c d e \right)^{2}} \cdot \sqrt[5]{\left( a b c d e \right)^{3}} \\
        & = 25 \sqrt[5]{\left( a b c d e \right)^{2} \cdot \left( a b c d e \right)^{3}}
    \end{align*}
\end{frame}



\begin{frame}{Beweis}
    Seien \( a, b, c, d, e \geq 0 \) beliebige nicht-negative reelle Zahlen. Wir schätzen wie folgt ab:
    \begin{align*}
        & \left( a^{2} + b^{2} + c^{2} + d^{2} + e^{2} \right) \cdot \left( a^{3} + b^{3} + c^{3} + d^{3} + e^{3} \right) \\
        & = 25 \cdot \frac{\left( a^{2} + b^{2} + c^{2} + d^{2} + e^{2} \right)}{5} \cdot \frac{\left( a^{3} + b^{3} + c^{3} + d^{3} + e^{3} \right)}{5} \\
        & \geq 25 \sqrt[5]{a^{2} b^{2} c^{2} d^{2} e^{2}} \cdot \frac{\left( a^{3} + b^{3} + c^{3} + d^{3} + e^{3} \right)}{5} \\
        & \geq 25 \sqrt[5]{a^{2} b^{2} c^{2} d^{2} e^{2}} \cdot \sqrt[5]{a^{3} b^{3} c^{3} d^{3} e^{3}} \\
        & \geq 25 \sqrt[5]{\left( a b c d e \right)^{2}} \cdot \sqrt[5]{\left( a b c d e \right)^{3}} \\
        & = 25 \sqrt[5]{\left( a b c d e \right)^{2} \cdot \left( a b c d e \right)^{3}} \\
        & = 25 \sqrt[5]{\left( a b c d e \right)^{5}}
    \end{align*}
\end{frame}



\begin{frame}{Beweis}
    Seien \( a, b, c, d, e \geq 0 \) beliebige nicht-negative reelle Zahlen. Wir schätzen wie folgt ab:
    \begin{align*}
        & \left( a^{2} + b^{2} + c^{2} + d^{2} + e^{2} \right) \cdot \left( a^{3} + b^{3} + c^{3} + d^{3} + e^{3} \right) \\
        & = 25 \cdot \frac{\left( a^{2} + b^{2} + c^{2} + d^{2} + e^{2} \right)}{5} \cdot \frac{\left( a^{3} + b^{3} + c^{3} + d^{3} + e^{3} \right)}{5} \\
        & \geq 25 \sqrt[5]{a^{2} b^{2} c^{2} d^{2} e^{2}} \cdot \frac{\left( a^{3} + b^{3} + c^{3} + d^{3} + e^{3} \right)}{5} \\
        & \geq 25 \sqrt[5]{a^{2} b^{2} c^{2} d^{2} e^{2}} \cdot \sqrt[5]{a^{3} b^{3} c^{3} d^{3} e^{3}} \\
        & \geq 25 \sqrt[5]{\left( a b c d e \right)^{2}} \cdot \sqrt[5]{\left( a b c d e \right)^{3}} \\
        & = 25 \sqrt[5]{\left( a b c d e \right)^{2} \cdot \left( a b c d e \right)^{3}} \\
        & = 25 \sqrt[5]{\left( a b c d e \right)^{5}} \\
        & = 25 a b c d e.
    \end{align*}
\end{frame}
% ============================================================

\end{document}