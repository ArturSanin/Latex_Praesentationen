\documentclass[10pt]{beamer}

\title{}
\author{Artur's Mathematikstübchen}
\date{}


% ===== Packages =========
\usepackage[utf8]{inputenc}

\usepackage[natbibapa]{apacite}
\bibliographystyle{apacite}
\usepackage[ngerman]{babel}
\usepackage{graphicx}
\usepackage{fancyhdr}
\usepackage{amsmath}
\usepackage{amssymb}
\usepackage{graphicx}
\usepackage{MnSymbol}
\usepackage{enumitem}
\usepackage{amsthm}
\usepackage{mleftright}
\usepackage{dsfont}
\usepackage{tikz-cd}


\def\bD{\mathbb{D}}               
\def\bE{\mathbb{E}}
\def\bG{\mathbb{G}}
\def\bN{\mathbb{N}}
\def\bP{\mathbb{P}}
\def\bQ{\mathbb{Q}}
\def\bR{\mathbb{R}}
\def\bBarR{\bar{\mathbb{R}}}
\def\bY{\mathbb{Y}}



\def\mA{\mathcal{A}}
\def\mB{\mathcal{B}}
\def\mD{\mathcal{D}}
\def\mE{\mathcal{E}}
\def\mF{\mathcal{F}}
\def\mG{\mathcal{G}}
\def\mH{\mathcal{H}}
\def\mL{\mathcal{L}}
\def\mN{\mathcal{N}}
\def\mP{\mathcal{P}}
\def\mS{\mathcal{S}}
\def\mT{\mathcal{T}}
\def\mX{\mathcal{X}}
\def\mY{\mathcal{Y}}



\usetheme{Madrid}



% ======================== Beginn Document ========================

\begin{document}





% ======================== Begrüßung ==================

\begin{frame}
    \begin{center}
        \textbf{\huge Willkommen in der guten Stube \newline \newline :D}
    \end{center}
\end{frame}
% =====================================================



% ======================== Präsentation ==================

\begin{frame}
    \begin{alertblock}{Aufgabe}
        Seien \( x_{1}, x_{2}, \ldots, x_{n} > 0 \), \( n \in \bN \), positive reelle Zahlen. Weiter sei \( s \coloneq x_{1} + x_{2} + \ldots + x_{n} \) die Summe der Zahlen \( x_{1}, x_{2}, \ldots, x_{n} \). Man zeige die Abschätzung:
        \begin{align*}
            \left( 1 + x_{1} \right) \cdot \left( 1 + x_{2} \right) \cdot \ldots \cdot \left( 1 + x_{n} \right)
            \leq 1 + s + \frac{s^{2}}{2!} + + \frac{s^{3}}{3!} + \ldots + \frac{s^{n}}{n!}.
        \end{align*}
    \end{alertblock}
\end{frame}



\begin{frame}{Hilfsabschätzung}
    
\end{frame}



\begin{frame}{Hilfsabschätzung}
    Für den Beweis verwenden wir die Ungleichung zwischen dem geometrischen und arithmetischen Mittel:
\end{frame}



\begin{frame}{Hilfsabschätzung}
    Für den Beweis verwenden wir die Ungleichung zwischen dem geometrischen und arithmetischen Mittel:
    \begin{block}{Hilfsabschätzung}
        Für alle \( a_{1}, \ldots, a_{p} \geq 0 \), \( p \in \bN \), gilt die Abschätzung:
        \begin{align*}
            \sqrt[p]{a_{1} \cdot \ldots \cdot a_{p}} 
            \leq \frac{a_{1} + \ldots + a_{p}}{p}.
        \end{align*}
    \end{block}
\end{frame}



\begin{frame}{Hilfsabschätzung}
    Für den Beweis verwenden wir die Ungleichung zwischen dem geometrischen und arithmetischen Mittel:
    \begin{block}{Hilfsabschätzung}
        Für alle \( a_{1}, \ldots, a_{p} \geq 0 \), \( p \in \bN \), gilt die Abschätzung:
        \begin{align*}
            \sqrt[p]{a_{1} \cdot \ldots \cdot a_{p}} 
            \leq \frac{a_{1} + \ldots + a_{p}}{p}.
        \end{align*}
    \end{block}
    Insbesondere folgt aus dieser Ungleichung die Abschätzung:
\end{frame}



\begin{frame}{Hilfsabschätzung}
    Für den Beweis verwenden wir die Ungleichung zwischen dem geometrischen und arithmetischen Mittel:
    \begin{block}{Hilfsabschätzung}
        Für alle \( a_{1}, \ldots, a_{p} \geq 0 \), \( p \in \bN \), gilt die Abschätzung:
        \begin{align*}
            \sqrt[p]{a_{1} \cdot \ldots \cdot a_{p}} 
            \leq \frac{a_{1} + \ldots + a_{p}}{p}.
        \end{align*}
    \end{block}
    Insbesondere folgt aus dieser Ungleichung die Abschätzung:
    \begin{align*}
        a_{1} \cdot \ldots \cdot a_{p} 
        \leq \frac{\left( a_{1} + \ldots + a_{p} \right)^{p}}{p^{p}}.
    \end{align*}
\end{frame}



\begin{frame}{Hilfsabschätzung}
    Für den Beweis verwenden wir die Ungleichung zwischen dem geometrischen und arithmetischen Mittel:
    \begin{block}{Hilfsabschätzung}
        Für alle \( a_{1}, \ldots, a_{p} \geq 0 \), \( p \in \bN \), gilt die Abschätzung:
        \begin{align*}
            \sqrt[p]{a_{1} \cdot \ldots \cdot a_{p}} 
            \leq \frac{a_{1} + \ldots + a_{p}}{p}.
        \end{align*}
    \end{block}
    Insbesondere folgt aus dieser Ungleichung die Abschätzung:
    \begin{align*}
        a_{1} \cdot \ldots \cdot a_{p} 
        \leq \frac{\left( a_{1} + \ldots + a_{p} \right)^{p}}{p^{p}}.
    \end{align*}
    Zusätzlich benötigen wir den binomischen Lehrsatz:
\end{frame}



\begin{frame}{Hilfsabschätzung}
    Für den Beweis verwenden wir die Ungleichung zwischen dem geometrischen und arithmetischen Mittel:
    \begin{block}{Hilfsabschätzung}
        Für alle \( a_{1}, \ldots, a_{p} \geq 0 \), \( p \in \bN \), gilt die Abschätzung:
        \begin{align*}
            \sqrt[p]{a_{1} \cdot \ldots \cdot a_{p}} 
            \leq \frac{a_{1} + \ldots + a_{p}}{p}.
        \end{align*}
    \end{block}
    Insbesondere folgt aus dieser Ungleichung die Abschätzung:
    \begin{align*}
        a_{1} \cdot \ldots \cdot a_{p} 
        \leq \frac{\left( a_{1} + \ldots + a_{p} \right)^{p}}{p^{p}}.
    \end{align*}
    Zusätzlich benötigen wir den binomischen Lehrsatz:
    \begin{align*}
        \left( x + y \right)^{m}
        = \sum_{k = 0}^{m} \binom{m}{k} x^{n - k} y^{k}.
    \end{align*}
\end{frame}



\begin{frame}{Beweis}
    
\end{frame}



\begin{frame}{Beweis}
    Seien \( x_{1}, x_{2}, \ldots, x_{n} > 0 \) positive reelle Zahlen und \( s = x_{1} + x_{2} + \ldots + x_{n} \). 
\end{frame}



\begin{frame}{Beweis}
    Seien \( x_{1}, x_{2}, \ldots, x_{n} > 0 \) positive reelle Zahlen und \( s = x_{1} + x_{2} + \ldots + x_{n} \). Dann schätzen wir wie folgt ab: 
\end{frame}



\begin{frame}{Beweis}
    Seien \( x_{1}, x_{2}, \ldots, x_{n} > 0 \) positive reelle Zahlen und \( s = x_{1} + x_{2} + \ldots + x_{n} \). Dann schätzen wir wie folgt ab:
    \begin{align*}
         \left( 1 + x_{1} \right) \cdot \left( 1 + x_{2} \right) \cdot \ldots \cdot \left( 1 + x_{n} \right)
    \end{align*}
\end{frame}



\begin{frame}{Beweis}
    Seien \( x_{1}, x_{2}, \ldots, x_{n} > 0 \) positive reelle Zahlen und \( s = x_{1} + x_{2} + \ldots + x_{n} \). Dann schätzen wir wie folgt ab:
    \begin{align*}
         \left( 1 + x_{1} \right) \cdot \left( 1 + x_{2} \right) \cdot \ldots \cdot \left( 1 + x_{n} \right)
         & \leq \frac{\left( 1 + x_{1} + 1 + x_{2} + \ldots + 1 + x_{n} \right)^{n}}{n^{n}}
    \end{align*}
\end{frame}



\begin{frame}{Beweis}
    Seien \( x_{1}, x_{2}, \ldots, x_{n} > 0 \) positive reelle Zahlen und \( s = x_{1} + x_{2} + \ldots + x_{n} \). Dann schätzen wir wie folgt ab:
    \begin{align*}
         \left( 1 + x_{1} \right) \cdot \left( 1 + x_{2} \right) \cdot \ldots \cdot \left( 1 + x_{n} \right)
         & \leq \frac{\left( 1 + x_{1} + 1 + x_{2} + \ldots + 1 + x_{n} \right)^{n}}{n^{n}} \\
         & = \frac{\left( n + s \right)^{n}}{n^{n}}
    \end{align*}
\end{frame}



\begin{frame}{Beweis}
    Seien \( x_{1}, x_{2}, \ldots, x_{n} > 0 \) positive reelle Zahlen und \( s = x_{1} + x_{2} + \ldots + x_{n} \). Dann schätzen wir wie folgt ab:
    \begin{align*}
         \left( 1 + x_{1} \right) \cdot \left( 1 + x_{2} \right) \cdot \ldots \cdot \left( 1 + x_{n} \right)
         & \leq \frac{\left( 1 + x_{1} + 1 + x_{2} + \ldots + 1 + x_{n} \right)^{n}}{n^{n}} \\
         & = \frac{\left( n + s \right)^{n}}{n^{n}} \\
         & = \left( 1 + \frac{s}{n} \right)^{n}
    \end{align*}
\end{frame}



\begin{frame}{Beweis}
    Seien \( x_{1}, x_{2}, \ldots, x_{n} > 0 \) positive reelle Zahlen und \( s = x_{1} + x_{2} + \ldots + x_{n} \). Dann schätzen wir wie folgt ab:
    \begin{align*}
         \left( 1 + x_{1} \right) \cdot \left( 1 + x_{2} \right) \cdot \ldots \cdot \left( 1 + x_{n} \right)
         & \leq \frac{\left( 1 + x_{1} + 1 + x_{2} + \ldots + 1 + x_{n} \right)^{n}}{n^{n}} \\
         & = \frac{\left( n + s \right)^{n}}{n^{n}} \\
         & = \left( 1 + \frac{s}{n} \right)^{n} \\
         & = \sum_{k = 0}^{n} \binom{n}{k}\left( \frac{s}{n} \right)^{k}
    \end{align*}
\end{frame}



\begin{frame}{Beweis}
    Seien \( x_{1}, x_{2}, \ldots, x_{n} > 0 \) positive reelle Zahlen und \( s = x_{1} + x_{2} + \ldots + x_{n} \). Dann schätzen wir wie folgt ab:
    \begin{align*}
         \left( 1 + x_{1} \right) \cdot \left( 1 + x_{2} \right) \cdot \ldots \cdot \left( 1 + x_{n} \right)
         & \leq \frac{\left( 1 + x_{1} + 1 + x_{2} + \ldots + 1 + x_{n} \right)^{n}}{n^{n}} \\
         & = \frac{\left( n + s \right)^{n}}{n^{n}} \\
         & = \left( 1 + \frac{s}{n} \right)^{n} \\
         & = \sum_{k = 0}^{n} \binom{n}{k}\left( \frac{s}{n} \right)^{k} \\
         & = \sum_{k = 0}^{n} \frac{n!}{k! \left( n - k \right)!} \frac{s^{k}}{n^{k}}
    \end{align*}
\end{frame}



\begin{frame}{Beweis}
    \begin{align*}
        \sum_{k = 0}^{n} \frac{n!}{k! \left( n - k \right)!} \frac{s^{k}}{n^{k}}
    \end{align*}
\end{frame}



\begin{frame}{Beweis}
    \begin{align*}
        \sum_{k = 0}^{n} \frac{n!}{k! \left( n - k \right)!} \frac{s^{k}}{n^{k}}
        & = \sum_{k = 0}^{n} \frac{n!}{n^{k} \left( n - k \right)!} \frac{s^{k}}{k!}
    \end{align*}
\end{frame}



\begin{frame}{Beweis}
    \begin{align*}
        \sum_{k = 0}^{n} \frac{n!}{k! \left( n - k \right)!} \frac{s^{k}}{n^{k}}
        & = \sum_{k = 0}^{n} \frac{n!}{n^{k} \left( n - k \right)!} \frac{s^{k}}{k!} \\
        & = \sum_{k = 0}^{n} \frac{n \left( n - 1 \right) \ldots \left( n - k + 1 \right)}{n^{k}} \frac{s^{k}}{k!}
    \end{align*}
\end{frame}



\begin{frame}{Beweis}
    \begin{align*}
        \sum_{k = 0}^{n} \frac{n!}{k! \left( n - k \right)!} \frac{s^{k}}{n^{k}}
        & = \sum_{k = 0}^{n} \frac{n!}{n^{k} \left( n - k \right)!} \frac{s^{k}}{k!} \\
        & = \sum_{k = 0}^{n} \frac{n \left( n - 1 \right) \ldots \left( n - k + 1 \right)}{n^{k}} \frac{s^{k}}{k!} \\
        & = \sum_{k = 0}^{n} \frac{n}{n} \frac{\left( n - 1 \right)}{n} \ldots \frac{\left( n - k + 1 \right)}{n} \frac{s^{k}}{k!}
    \end{align*}
\end{frame}



\begin{frame}{Beweis}
    \begin{align*}
        \sum_{k = 0}^{n} \frac{n!}{k! \left( n - k \right)!} \frac{s^{k}}{n^{k}}
        & = \sum_{k = 0}^{n} \frac{n!}{n^{k} \left( n - k \right)!} \frac{s^{k}}{k!} \\
        & = \sum_{k = 0}^{n} \frac{n \left( n - 1 \right) \ldots \left( n - k + 1 \right)}{n^{k}} \frac{s^{k}}{k!} \\
        & = \sum_{k = 0}^{n} \frac{n}{n} \frac{\left( n - 1 \right)}{n} \ldots \frac{\left( n - k + 1 \right)}{n} \frac{s^{k}}{k!} \\
        & = \sum_{k = 0}^{n} \left( 1 - \frac{1}{n} \right) \ldots \left( 1 - \frac{k - 1}{n} \right) \frac{s^{k}}{k!}
    \end{align*}
\end{frame}



\begin{frame}{Beweis}
    \begin{align*}
        \sum_{k = 0}^{n} \frac{n!}{k! \left( n - k \right)!} \frac{s^{k}}{n^{k}}
        & = \sum_{k = 0}^{n} \frac{n!}{n^{k} \left( n - k \right)!} \frac{s^{k}}{k!} \\
        & = \sum_{k = 0}^{n} \frac{n \left( n - 1 \right) \ldots \left( n - k + 1 \right)}{n^{k}} \frac{s^{k}}{k!} \\
        & = \sum_{k = 0}^{n} \frac{n}{n} \frac{\left( n - 1 \right)}{n} \ldots \frac{\left( n - k + 1 \right)}{n} \frac{s^{k}}{k!} \\
        & = \sum_{k = 0}^{n} \left( 1 - \frac{1}{n} \right) \ldots \left( 1 - \frac{k - 1}{n} \right) \frac{s^{k}}{k!} \\
        & \leq \sum_{k = 0}^{n} \frac{s^{k}}{k!}
    \end{align*}
\end{frame}



\begin{frame}{Beweis}
    \begin{align*}
        \sum_{k = 0}^{n} \frac{n!}{k! \left( n - k \right)!} \frac{s^{k}}{n^{k}}
        & = \sum_{k = 0}^{n} \frac{n!}{n^{k} \left( n - k \right)!} \frac{s^{k}}{k!} \\
        & = \sum_{k = 0}^{n} \frac{n \left( n - 1 \right) \ldots \left( n - k + 1 \right)}{n^{k}} \frac{s^{k}}{k!} \\
        & = \sum_{k = 0}^{n} \frac{n}{n} \frac{\left( n - 1 \right)}{n} \ldots \frac{\left( n - k + 1 \right)}{n} \frac{s^{k}}{k!} \\
        & = \sum_{k = 0}^{n} \left( 1 - \frac{1}{n} \right) \ldots \left( 1 - \frac{k - 1}{n} \right) \frac{s^{k}}{k!} \\
        & \leq \sum_{k = 0}^{n} \frac{s^{k}}{k!} \\
        & = 1 + s + \frac{s^{2}}{2!} + + \frac{s^{3}}{3!} + \ldots + \frac{s^{n}}{n!}.
    \end{align*}
\end{frame}
% ============================================================

\end{document}