\documentclass[10pt]{beamer}

\title{}
\author{Artur's Mathematikstübchen}
\date{}


% ===== Packages =========
\usepackage[utf8]{inputenc}

\usepackage[natbibapa]{apacite}
\bibliographystyle{apacite}
\usepackage[ngerman]{babel}
\usepackage{graphicx}
\usepackage{fancyhdr}
\usepackage{amsmath}
\usepackage{amssymb}
\usepackage{graphicx}
\usepackage{MnSymbol}
\usepackage{enumitem}
\usepackage{amsthm}
\usepackage{mleftright}
\usepackage{dsfont}
\usepackage{tikz-cd}


\def\bD{\mathbb{D}}               
\def\bE{\mathbb{E}}
\def\bG{\mathbb{G}}
\def\bN{\mathbb{N}}
\def\bP{\mathbb{P}}
\def\bQ{\mathbb{Q}}
\def\bR{\mathbb{R}}
\def\bBarR{\bar{\mathbb{R}}}
\def\bY{\mathbb{Y}}



\def\mA{\mathcal{A}}
\def\mB{\mathcal{B}}
\def\mD{\mathcal{D}}
\def\mE{\mathcal{E}}
\def\mF{\mathcal{F}}
\def\mG{\mathcal{G}}
\def\mH{\mathcal{H}}
\def\mL{\mathcal{L}}
\def\mN{\mathcal{N}}
\def\mP{\mathcal{P}}
\def\mS{\mathcal{S}}
\def\mT{\mathcal{T}}
\def\mX{\mathcal{X}}
\def\mY{\mathcal{Y}}



\usetheme{Madrid}



% ======================== Beginn Document ========================

\begin{document}





% ======================== Begrüßung ==================

\begin{frame}
    \begin{center}
        \textbf{\huge Willkommen in der guten Stube \newline \newline :D}
    \end{center}
\end{frame}
% =====================================================



% ======================== Präsentation ==================

\begin{frame}
    \begin{alertblock}{Aufgabe}
        Seien \( x, y, z > 0 \) drei positive reelle Zahlen. Man zeige die Abschätzung:
        \begin{align*}
            \frac{x^{4}}{y z^{3}} + \frac{y^{4}}{z x^{3}} + \frac{z^{4}}{x y^{3}}
            \geq 3.
        \end{align*}
    \end{alertblock}
\end{frame}



\begin{frame}{Hilfsabschätzung}
    
\end{frame}



\begin{frame}{Hilfsabschätzung}
    Für den Beweis verwenden wir die Ungleichung zwischen dem geometrischen und arithmetischen Mittel:
\end{frame}



\begin{frame}{Hilfsabschätzung}
    Für den Beweis verwenden wir die Ungleichung zwischen dem geometrischen und arithmetischen Mittel:
    \begin{block}{Hilfsabschätzung}
        Für alle \( a_{1}, \ldots, a_{p} \geq 0 \), \( p \in \bN \), gilt die Abschätzung:
        \begin{align*}
            \sqrt[p]{a_{1} \cdot \ldots \cdot a_{p}} 
            \leq \frac{a_{1} + \ldots + a_{p}}{p}.
        \end{align*}
    \end{block}
\end{frame}



\begin{frame}{Hilfsabschätzung}
    Für den Beweis verwenden wir die Ungleichung zwischen dem geometrischen und arithmetischen Mittel:
    \begin{block}{Hilfsabschätzung}
        Für alle \( a_{1}, \ldots, a_{p} \geq 0 \), \( p \in \bN \), gilt die Abschätzung:
        \begin{align*}
            \sqrt[p]{a_{1} \cdot \ldots \cdot a_{p}} 
            \leq \frac{a_{1} + \ldots + a_{p}}{p}.
        \end{align*}
    \end{block}
    Wir verwenden die Ungleichung für den Fall \( p = 3 \):
\end{frame}



\begin{frame}{Hilfsabschätzung}
    Für den Beweis verwenden wir die Ungleichung zwischen dem geometrischen und arithmetischen Mittel:
    \begin{block}{Hilfsabschätzung}
        Für alle \( a_{1}, \ldots, a_{p} \geq 0 \), \( p \in \bN \), gilt die Abschätzung:
        \begin{align*}
            \sqrt[p]{a_{1} \cdot \ldots \cdot a_{p}} 
            \leq \frac{a_{1} + \ldots + a_{p}}{p}.
        \end{align*}
    \end{block}
    Wir verwenden die Ungleichung für den Fall \( p = 3 \):
    \begin{align*}
        \sqrt[3]{a_{1} \cdot a_{2} \cdot a_{3}} 
        \leq \frac{a_{1} + a_{2} + a_{3}}{3}.
    \end{align*}
\end{frame}



\begin{frame}{Beweis}
    
\end{frame}



\begin{frame}{Beweis}
    Seien \( x, y, z > 0 \) drei positive rellee Zahlen.
\end{frame}



\begin{frame}{Beweis}
    Seien \( x, y, z > 0 \) drei positive rellee Zahlen. Wir schätzen wie folgt ab:
\end{frame}



\begin{frame}{Beweis}
    Seien \( x, y, z > 0 \) drei positive rellee Zahlen. Wir schätzen wie folgt ab:
    \begin{align*}
        \frac{x^{4}}{y z^{3}} + \frac{y^{4}}{z x^{3}} + \frac{z^{4}}{x y^{3}}
    \end{align*}
\end{frame}



\begin{frame}{Beweis}
    Seien \( x, y, z > 0 \) drei positive rellee Zahlen. Wir schätzen wie folgt ab:
    \begin{align*}
        \frac{x^{4}}{y z^{3}} + \frac{y^{4}}{z x^{3}} + \frac{z^{4}}{x y^{3}}
        & = 3 \cdot \frac{\frac{x^{4}}{y z^{3}} + \frac{y^{4}}{z x^{3}} + \frac{z^{4}}{x y^{3}}}{3}
    \end{align*}
\end{frame}



\begin{frame}{Beweis}
    Seien \( x, y, z > 0 \) drei positive rellee Zahlen. Wir schätzen wie folgt ab:
    \begin{align*}
        \frac{x^{4}}{y z^{3}} + \frac{y^{4}}{z x^{3}} + \frac{z^{4}}{x y^{3}}
        & = 3 \cdot \frac{\frac{x^{4}}{y z^{3}} + \frac{y^{4}}{z x^{3}} + \frac{z^{4}}{x y^{3}}}{3} \\
        & \geq 3 \cdot \sqrt[3]{\frac{x^{4}}{y z^{3}} \cdot \frac{y^{4}}{z x^{3}} \cdot \frac{z^{4}}{x y^{3}}}
    \end{align*}
\end{frame}



\begin{frame}{Beweis}
    Seien \( x, y, z > 0 \) drei positive rellee Zahlen. Wir schätzen wie folgt ab:
    \begin{align*}
        \frac{x^{4}}{y z^{3}} + \frac{y^{4}}{z x^{3}} + \frac{z^{4}}{x y^{3}}
        & = 3 \cdot \frac{\frac{x^{4}}{y z^{3}} + \frac{y^{4}}{z x^{3}} + \frac{z^{4}}{x y^{3}}}{3} \\
        & \geq 3 \cdot \sqrt[3]{\frac{x^{4}}{y z^{3}} \cdot \frac{y^{4}}{z x^{3}} \cdot \frac{z^{4}}{x y^{3}}} \\
        & = 3 \cdot \sqrt[3]{\frac{x^{4} \cdot y^{4} \cdot z^{4}}{y z^{3} \cdot z x^{3} \cdot x y^{3}}}
    \end{align*}
\end{frame}



\begin{frame}{Beweis}
    Seien \( x, y, z > 0 \) drei positive rellee Zahlen. Wir schätzen wie folgt ab:
    \begin{align*}
        \frac{x^{4}}{y z^{3}} + \frac{y^{4}}{z x^{3}} + \frac{z^{4}}{x y^{3}}
        & = 3 \cdot \frac{\frac{x^{4}}{y z^{3}} + \frac{y^{4}}{z x^{3}} + \frac{z^{4}}{x y^{3}}}{3} \\
        & \geq 3 \cdot \sqrt[3]{\frac{x^{4}}{y z^{3}} \cdot \frac{y^{4}}{z x^{3}} \cdot \frac{z^{4}}{x y^{3}}} \\
        & = 3 \cdot \sqrt[3]{\frac{x^{4} \cdot y^{4} \cdot z^{4}}{y z^{3} \cdot z x^{3} \cdot x y^{3}}} \\
        & = 3 \cdot \sqrt[3]{\frac{x^{4} \cdot y^{4} \cdot z^{4}}{x^{4} \cdot y^{4} \cdot z^{4}}}
    \end{align*}
\end{frame}



\begin{frame}{Beweis}
    Seien \( x, y, z > 0 \) drei positive rellee Zahlen. Wir schätzen wie folgt ab:
    \begin{align*}
        \frac{x^{4}}{y z^{3}} + \frac{y^{4}}{z x^{3}} + \frac{z^{4}}{x y^{3}}
        & = 3 \cdot \frac{\frac{x^{4}}{y z^{3}} + \frac{y^{4}}{z x^{3}} + \frac{z^{4}}{x y^{3}}}{3} \\
        & \geq 3 \cdot \sqrt[3]{\frac{x^{4}}{y z^{3}} \cdot \frac{y^{4}}{z x^{3}} \cdot \frac{z^{4}}{x y^{3}}} \\
        & = 3 \cdot \sqrt[3]{\frac{x^{4} \cdot y^{4} \cdot z^{4}}{y z^{3} \cdot z x^{3} \cdot x y^{3}}} \\
        & = 3 \cdot \sqrt[3]{\frac{x^{4} \cdot y^{4} \cdot z^{4}}{x^{4} \cdot y^{4} \cdot z^{4}}} \\
        & = 3 \cdot \sqrt[3]{1}
    \end{align*}
\end{frame}



\begin{frame}{Beweis}
    Seien \( x, y, z > 0 \) drei positive rellee Zahlen. Wir schätzen wie folgt ab:
    \begin{align*}
        \frac{x^{4}}{y z^{3}} + \frac{y^{4}}{z x^{3}} + \frac{z^{4}}{x y^{3}}
        & = 3 \cdot \frac{\frac{x^{4}}{y z^{3}} + \frac{y^{4}}{z x^{3}} + \frac{z^{4}}{x y^{3}}}{3} \\
        & \geq 3 \cdot \sqrt[3]{\frac{x^{4}}{y z^{3}} \cdot \frac{y^{4}}{z x^{3}} \cdot \frac{z^{4}}{x y^{3}}} \\
        & = 3 \cdot \sqrt[3]{\frac{x^{4} \cdot y^{4} \cdot z^{4}}{y z^{3} \cdot z x^{3} \cdot x y^{3}}} \\
        & = 3 \cdot \sqrt[3]{\frac{x^{4} \cdot y^{4} \cdot z^{4}}{x^{4} \cdot y^{4} \cdot z^{4}}} \\
        & = 3 \cdot \sqrt[3]{1} \\
        & = 3.
    \end{align*}
\end{frame}
% ============================================================

\end{document}