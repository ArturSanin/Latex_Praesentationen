\documentclass[10pt]{beamer}

\title{}
\author{Artur's Mathematikstübchen}
\date{}


% ===== Packages =========
\usepackage[utf8]{inputenc}

\usepackage[natbibapa]{apacite}
\bibliographystyle{apacite}
\usepackage[ngerman]{babel}
\usepackage{graphicx}
\usepackage{fancyhdr}
\usepackage{amsmath}
\usepackage{amssymb}
\usepackage{graphicx}
\usepackage{MnSymbol}
\usepackage{enumitem}
\usepackage{amsthm}
\usepackage{mleftright}
\usepackage{dsfont}
\usepackage{tikz-cd}


\def\bD{\mathbb{D}}               
\def\bE{\mathbb{E}}
\def\bG{\mathbb{G}}
\def\bN{\mathbb{N}}
\def\bP{\mathbb{P}}
\def\bQ{\mathbb{Q}}
\def\bR{\mathbb{R}}
\def\bBarR{\bar{\mathbb{R}}}
\def\bY{\mathbb{Y}}



\def\mA{\mathcal{A}}
\def\mB{\mathcal{B}}
\def\mD{\mathcal{D}}
\def\mE{\mathcal{E}}
\def\mF{\mathcal{F}}
\def\mG{\mathcal{G}}
\def\mH{\mathcal{H}}
\def\mL{\mathcal{L}}
\def\mN{\mathcal{N}}
\def\mP{\mathcal{P}}
\def\mS{\mathcal{S}}
\def\mT{\mathcal{T}}
\def\mX{\mathcal{X}}
\def\mY{\mathcal{Y}}



\usetheme{Madrid}



% ======================== Beginn Document ========================

\begin{document}





% ======================== Begrüßung ==================

\begin{frame}
    \begin{center}
        \textbf{\huge Willkommen in der guten Stube \newline \newline :D}
    \end{center}
\end{frame}
% =====================================================



% ======================== Präsentation ==================

\begin{frame}
    \begin{alertblock}{Aufgabe}
        Sei \( \left\vert x \right\vert < 1 \) eine reelle Zahl. Für \( n \in \bN \) zeige man die Abschätzung:
        \begin{align*}
            \sum_{k = 0}^{n} \sqrt{\binom{n}{k}} \cdot x^{k}
            \leq \frac{2^{\frac{n}{2}}}{\sqrt{1 - x^{2}}}.
        \end{align*}
    \end{alertblock}
\end{frame}



\begin{frame}{Hilfsabschätzung}
    
\end{frame}



\begin{frame}{Hilfsabschätzung}
    \begin{block}{Cauchy-Schwarzsche Ungleichung}
        Für alle \( x_{1}, x_{2}, \ldots, x_{m}, y_{1}, y_{2}, \ldots, y_{m} \in \bR \), \( m \in \bN \), gilt die Abschätzung:
        \begin{align*}
            \sum_{k = 1}^{m} x_{k} y_{k}
            \leq \sqrt{\sum_{k = 1}^{m} x_{k}^{2}} \cdot \sqrt{\sum_{k = 1}^{m} y_{k}^{2}}.
        \end{align*}
    \end{block}
\end{frame}



\begin{frame}{Beweis}
    
\end{frame}



\begin{frame}{Beweis}
    Sei \( \left\vert x \right\vert < 1 \) eine reelle Zahl und \( n \in \bN \) eine beliebige natürliche Zahl.
\end{frame}



\begin{frame}{Beweis}
    Sei \( \left\vert x \right\vert < 1 \) eine reelle Zahl und \( n \in \bN \) eine beliebige natürliche Zahl. Dann schätzen wir ab:
\end{frame}



\begin{frame}{Beweis}
    Sei \( \left\vert x \right\vert < 1 \) eine reelle Zahl und \( n \in \bN \) eine beliebige natürliche Zahl. Dann schätzen wir ab:
    \begin{align*}
        \sum_{k = 0}^{n} \sqrt{\binom{n}{k}} \cdot x^{k}
    \end{align*}
\end{frame}



\begin{frame}{Beweis}
    Sei \( \left\vert x \right\vert < 1 \) eine reelle Zahl und \( n \in \bN \) eine beliebige natürliche Zahl. Dann schätzen wir ab:
    \begin{align*}
        \sum_{k = 0}^{n} \sqrt{\binom{n}{k}} \cdot x^{k}
        & \leq \sqrt{\sum_{k = 0}^{n} \left( \sqrt{\binom{n}{k}} \right)^{2}} \cdot \sqrt{\sum_{k = 0}^{n} x^{2k}}
    \end{align*}
\end{frame}



\begin{frame}{Beweis}
    Sei \( \left\vert x \right\vert < 1 \) eine reelle Zahl und \( n \in \bN \) eine beliebige natürliche Zahl. Dann schätzen wir ab:
    \begin{align*}
        \sum_{k = 0}^{n} \sqrt{\binom{n}{k}} \cdot x^{k}
        & \leq \sqrt{\sum_{k = 0}^{n} \left( \sqrt{\binom{n}{k}} \right)^{2}} \cdot \sqrt{\sum_{k = 0}^{n} x^{2k}} \\
        & = \sqrt{\sum_{k = 0}^{n} \binom{n}{k}} \cdot \sqrt{\sum_{k = 0}^{n} x^{2k}}
    \end{align*}
\end{frame}



\begin{frame}{Beweis}
    Sei \( \left\vert x \right\vert < 1 \) eine reelle Zahl und \( n \in \bN \) eine beliebige natürliche Zahl. Dann schätzen wir ab:
    \begin{align*}
        \sum_{k = 0}^{n} \sqrt{\binom{n}{k}} \cdot x^{k}
        & \leq \sqrt{\sum_{k = 0}^{n} \left( \sqrt{\binom{n}{k}} \right)^{2}} \cdot \sqrt{\sum_{k = 0}^{n} x^{2k}} \\
        & = \sqrt{\sum_{k = 0}^{n} \binom{n}{k}} \cdot \sqrt{\sum_{k = 0}^{n} x^{2k}} \\
        & = \sqrt{\left( 1 + 1 \right)^{n}} \cdot \sqrt{\sum_{k = 0}^{n} x^{2k}} 
    \end{align*}
\end{frame}



\begin{frame}{Beweis}
    Sei \( \left\vert x \right\vert < 1 \) eine reelle Zahl und \( n \in \bN \) eine beliebige natürliche Zahl. Dann schätzen wir ab:
    \begin{align*}
        \sum_{k = 0}^{n} \sqrt{\binom{n}{k}} \cdot x^{k}
        & \leq \sqrt{\sum_{k = 0}^{n} \left( \sqrt{\binom{n}{k}} \right)^{2}} \cdot \sqrt{\sum_{k = 0}^{n} x^{2k}} \\
        & = \sqrt{\sum_{k = 0}^{n} \binom{n}{k}} \cdot \sqrt{\sum_{k = 0}^{n} x^{2k}} \\
        & = \sqrt{\left( 1 + 1 \right)^{n}} \cdot \sqrt{\sum_{k = 0}^{n} x^{2k}} \\
        & = \sqrt{2^{n}} \cdot \sqrt{\sum_{k = 0}^{n} x^{2k}}
    \end{align*}
\end{frame}



\begin{frame}{Beweis}
    Sei \( \left\vert x \right\vert < 1 \) eine reelle Zahl und \( n \in \bN \) eine beliebige natürliche Zahl. Dann schätzen wir ab:
    \begin{align*}
        \sum_{k = 0}^{n} \sqrt{\binom{n}{k}} \cdot x^{k}
        & \leq \sqrt{\sum_{k = 0}^{n} \left( \sqrt{\binom{n}{k}} \right)^{2}} \cdot \sqrt{\sum_{k = 0}^{n} x^{2k}} \\
        & = \sqrt{\sum_{k = 0}^{n} \binom{n}{k}} \cdot \sqrt{\sum_{k = 0}^{n} x^{2k}} \\
        & = \sqrt{\left( 1 + 1 \right)^{n}} \cdot \sqrt{\sum_{k = 0}^{n} x^{2k}} \\
        & = \sqrt{2^{n}} \cdot \sqrt{\sum_{k = 0}^{n} x^{2k}} \\
        & \leq \sqrt{2^{n}} \cdot \sqrt{\sum_{k = 0}^{\infty} x^{2k}}
    \end{align*}
\end{frame}



\begin{frame}{Beweis}
    Sei \( \left\vert x \right\vert < 1 \) eine reelle Zahl und \( n \in \bN \) eine beliebige natürliche Zahl. Dann schätzen wir ab:
    {\small
    \begin{align*}
        \sum_{k = 0}^{n} \sqrt{\binom{n}{k}} \cdot x^{k}
        & \leq \sqrt{\sum_{k = 0}^{n} \left( \sqrt{\binom{n}{k}} \right)^{2}} \cdot \sqrt{\sum_{k = 0}^{n} x^{2k}} \\
        & = \sqrt{\sum_{k = 0}^{n} \binom{n}{k}} \cdot \sqrt{\sum_{k = 0}^{n} x^{2k}} \\
        & = \sqrt{\left( 1 + 1 \right)^{n}} \cdot \sqrt{\sum_{k = 0}^{n} x^{2k}} \\
        & = \sqrt{2^{n}} \cdot \sqrt{\sum_{k = 0}^{n} x^{2k}} \\
        & \leq \sqrt{2^{n}} \cdot \sqrt{\sum_{k = 0}^{\infty} x^{2k}} \\
        & = \frac{2^{\frac{n}{2}}}{\sqrt{1 - x^{2}}}.
    \end{align*}
    }
\end{frame}
% ============================================================

\end{document}