\documentclass[10pt]{beamer}

\title{}
\author{Artur's Mathematikstübchen}
\date{}


% ===== Packages =========
\usepackage[utf8]{inputenc}

\usepackage[natbibapa]{apacite}
\bibliographystyle{apacite}
\usepackage[ngerman]{babel}
\usepackage{graphicx}
\usepackage{fancyhdr}
\usepackage{amsmath}
\usepackage{amssymb}
\usepackage{graphicx}
\usepackage{MnSymbol}
\usepackage{enumitem}
\usepackage{amsthm}
\usepackage{mleftright}
\usepackage{dsfont}
\usepackage{tikz-cd}


\def\bD{\mathbb{D}}               
\def\bE{\mathbb{E}}
\def\bG{\mathbb{G}}
\def\bN{\mathbb{N}}
\def\bP{\mathbb{P}}
\def\bQ{\mathbb{Q}}
\def\bR{\mathbb{R}}
\def\bBarR{\bar{\mathbb{R}}}
\def\bY{\mathbb{Y}}



\def\mA{\mathcal{A}}
\def\mB{\mathcal{B}}
\def\mD{\mathcal{D}}
\def\mE{\mathcal{E}}
\def\mF{\mathcal{F}}
\def\mG{\mathcal{G}}
\def\mH{\mathcal{H}}
\def\mL{\mathcal{L}}
\def\mN{\mathcal{N}}
\def\mP{\mathcal{P}}
\def\mS{\mathcal{S}}
\def\mT{\mathcal{T}}
\def\mX{\mathcal{X}}
\def\mY{\mathcal{Y}}



\usetheme{Madrid}



% ======================== Beginn Document ========================

\begin{document}





% ======================== Begrüßung ==================

\begin{frame}
    \begin{center}
        \textbf{\huge Willkommen in der guten Stube \newline \newline :D}
    \end{center}
\end{frame}
% =====================================================



% ======================== Präsentation ==================

\begin{frame}
    \begin{alertblock}{Aufgabe}
        Seien \( x_{1}, x_{2}, \ldots, x_{n} > 0 \), \( n \in \bN \), positive reelle Zahlen mit 
        \begin{align*}
            \sum_{k = 1}^{n} x_{k}^{2} 
            \leq 1.
        \end{align*}
        Man zeige die Abschätzung:
        \begin{align*}
            \sum_{k = 1}^{n} \sqrt{x_{k}}
            \leq \sqrt{\sum_{k = 1}^{n} \frac{1}{x_{k}}}.
        \end{align*}
    \end{alertblock}
\end{frame}



\begin{frame}{Hilfsabschätzung}
    
\end{frame}



\begin{frame}{Hilfsabschätzung}
    \begin{block}{Cauchy-Schwarzsche Ungleichung}
        Für alle \( x_{1}, x_{2}, \ldots, x_{m}, y_{1}, y_{2}, \ldots, y_{m} \in \bR \), \( m \in \bN \), gilt die Abschätzung:
        \begin{align*}
            \sum_{k = 1}^{m} x_{k} y_{k}
            \leq \sqrt{\sum_{k = 1}^{m} x_{k}^{2}} \cdot \sqrt{\sum_{k = 1}^{m} y_{k}^{2}}.
        \end{align*}
    \end{block}
\end{frame}



\begin{frame}{Beweis}
    
\end{frame}



\begin{frame}{Beweis}
    Seien \( x_{1}, x_{2}, \ldots, x_{n} > 0 \) positive reelle Zahlen mit 
    \begin{align*}
        \sum_{k = 1}^{n} x_{k}^{2} 
        \leq 1.
    \end{align*}
\end{frame}



\begin{frame}{Beweis}
    Seien \( x_{1}, x_{2}, \ldots, x_{n} > 0 \) positive reelle Zahlen mit 
    \begin{align*}
        \sum_{k = 1}^{n} x_{k}^{2} 
        \leq 1.
    \end{align*}
    Dann folgt mit der Cauchy-Schwarzschen Ungleichung:
\end{frame}



\begin{frame}{Beweis}
    Seien \( x_{1}, x_{2}, \ldots, x_{n} > 0 \) positive reelle Zahlen mit 
    \begin{align*}
        \sum_{k = 1}^{n} x_{k}^{2} 
        \leq 1.
    \end{align*}
    Dann folgt mit der Cauchy-Schwarzschen Ungleichung:
    \begin{align*}
        \sum_{k = 1}^{n} \sqrt{x_{k}}
    \end{align*}
\end{frame}



\begin{frame}{Beweis}
    Seien \( x_{1}, x_{2}, \ldots, x_{n} > 0 \) positive reelle Zahlen mit 
    \begin{align*}
        \sum_{k = 1}^{n} x_{k}^{2} 
        \leq 1.
    \end{align*}
    Dann folgt mit der Cauchy-Schwarzschen Ungleichung:
    \begin{align*}
        \sum_{k = 1}^{n} \sqrt{x_{k}}
        & = \sum_{k = 1}^{n} \sqrt{x_{k}} \cdot \frac{\sqrt{x_{k}}}{\sqrt{x_{k}}}
    \end{align*}
\end{frame}



\begin{frame}{Beweis}
    Seien \( x_{1}, x_{2}, \ldots, x_{n} > 0 \) positive reelle Zahlen mit 
    \begin{align*}
        \sum_{k = 1}^{n} x_{k}^{2} 
        \leq 1.
    \end{align*}
    Dann folgt mit der Cauchy-Schwarzschen Ungleichung:
    \begin{align*}
        \sum_{k = 1}^{n} \sqrt{x_{k}}
        & = \sum_{k = 1}^{n} \sqrt{x_{k}} \cdot \frac{\sqrt{x_{k}}}{\sqrt{x_{k}}} \\
        & = \sum_{k = 1}^{n} x_{k} \cdot \frac{1}{\sqrt{x_{k}}}
    \end{align*}
\end{frame}



\begin{frame}{Beweis}
    Seien \( x_{1}, x_{2}, \ldots, x_{n} > 0 \) positive reelle Zahlen mit 
    \begin{align*}
        \sum_{k = 1}^{n} x_{k}^{2} 
        \leq 1.
    \end{align*}
    Dann folgt mit der Cauchy-Schwarzschen Ungleichung:
    \begin{align*}
        \sum_{k = 1}^{n} \sqrt{x_{k}}
        & = \sum_{k = 1}^{n} \sqrt{x_{k}} \cdot \frac{\sqrt{x_{k}}}{\sqrt{x_{k}}} \\
        & = \sum_{k = 1}^{n} x_{k} \cdot \frac{1}{\sqrt{x_{k}}} \\
        & \leq \sqrt{\sum_{k = 1}^{n} x_{k}^{2}} \cdot \sqrt{\sum_{k = 1}^{n} \left( \frac{1}{\sqrt{x_{k}}} \right)^{2}}
    \end{align*}
\end{frame}



\begin{frame}{Beweis}
    Seien \( x_{1}, x_{2}, \ldots, x_{n} > 0 \) positive reelle Zahlen mit 
    \begin{align*}
        \sum_{k = 1}^{n} x_{k}^{2} 
        \leq 1.
    \end{align*}
    Dann folgt mit der Cauchy-Schwarzschen Ungleichung:
    \begin{align*}
        \sum_{k = 1}^{n} \sqrt{x_{k}}
        & = \sum_{k = 1}^{n} \sqrt{x_{k}} \cdot \frac{\sqrt{x_{k}}}{\sqrt{x_{k}}} \\
        & = \sum_{k = 1}^{n} x_{k} \cdot \frac{1}{\sqrt{x_{k}}} \\
        & \leq \sqrt{\sum_{k = 1}^{n} x_{k}^{2}} \cdot \sqrt{\sum_{k = 1}^{n} \left( \frac{1}{\sqrt{x_{k}}} \right)^{2}} \\
        & = \sqrt{\sum_{k = 1}^{n} x_{k}^{2}} \cdot \sqrt{\sum_{k = 1}^{n} \frac{1}{x_{k}}}
    \end{align*}
\end{frame}



\begin{frame}{Beweis}
    Seien \( x_{1}, x_{2}, \ldots, x_{n} > 0 \) positive reelle Zahlen mit 
    \begin{align*}
        \sum_{k = 1}^{n} x_{k}^{2} 
        \leq 1.
    \end{align*}
    Dann folgt mit der Cauchy-Schwarzschen Ungleichung:
    \begin{align*}
        \sum_{k = 1}^{n} \sqrt{x_{k}}
        & = \sum_{k = 1}^{n} \sqrt{x_{k}} \cdot \frac{\sqrt{x_{k}}}{\sqrt{x_{k}}} \\
        & = \sum_{k = 1}^{n} x_{k} \cdot \frac{1}{\sqrt{x_{k}}} \\
        & \leq \sqrt{\sum_{k = 1}^{n} x_{k}^{2}} \cdot \sqrt{\sum_{k = 1}^{n} \left( \frac{1}{\sqrt{x_{k}}} \right)^{2}} \\
        & = \sqrt{\sum_{k = 1}^{n} x_{k}^{2}} \cdot \sqrt{\sum_{k = 1}^{n} \frac{1}{x_{k}}} \\
        & \leq \sqrt{\sum_{k = 1}^{n} \frac{1}{x_{k}}}.
    \end{align*}
\end{frame}
% ============================================================

\end{document}