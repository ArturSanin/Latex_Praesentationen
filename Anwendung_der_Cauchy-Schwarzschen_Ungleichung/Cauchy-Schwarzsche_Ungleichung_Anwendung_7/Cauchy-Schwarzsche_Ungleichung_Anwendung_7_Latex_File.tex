\documentclass[10pt]{beamer}

\title{}
\author{Artur's Mathematikstübchen}
\date{}


% ===== Packages =========
\usepackage[utf8]{inputenc}

\usepackage[natbibapa]{apacite}
\bibliographystyle{apacite}
\usepackage[ngerman]{babel}
\usepackage{graphicx}
\usepackage{fancyhdr}
\usepackage{amsmath}
\usepackage{amssymb}
\usepackage{graphicx}
\usepackage{MnSymbol}
\usepackage{enumitem}
\usepackage{amsthm}
\usepackage{mleftright}
\usepackage{dsfont}
\usepackage{tikz-cd}


\def\bD{\mathbb{D}}               
\def\bE{\mathbb{E}}
\def\bG{\mathbb{G}}
\def\bN{\mathbb{N}}
\def\bP{\mathbb{P}}
\def\bQ{\mathbb{Q}}
\def\bR{\mathbb{R}}
\def\bBarR{\bar{\mathbb{R}}}
\def\bY{\mathbb{Y}}



\def\mA{\mathcal{A}}
\def\mB{\mathcal{B}}
\def\mD{\mathcal{D}}
\def\mE{\mathcal{E}}
\def\mF{\mathcal{F}}
\def\mG{\mathcal{G}}
\def\mH{\mathcal{H}}
\def\mL{\mathcal{L}}
\def\mN{\mathcal{N}}
\def\mP{\mathcal{P}}
\def\mS{\mathcal{S}}
\def\mT{\mathcal{T}}
\def\mX{\mathcal{X}}
\def\mY{\mathcal{Y}}



\usetheme{Madrid}



% ======================== Beginn Document ========================

\begin{document}





% ======================== Begrüßung ==================

\begin{frame}
    \begin{center}
        \textbf{\huge Willkommen in der guten Stube \newline \newline :D}
    \end{center}
\end{frame}
% =====================================================



% ======================== Präsentation ==================

\begin{frame}
    \begin{alertblock}{Aufgabe}
        Seien \( x, y, z > 0 \) drei positive reelle Zahlen. Man zeige die Abschätzung:
        \begin{align*}
            x + y + z
            \leq 2 \cdot \left( \frac{x^{2}}{y + z} + \frac{y^{2}}{x + z} + \frac{z^{2}}{x + y} \right).
        \end{align*}
    \end{alertblock}
\end{frame}



\begin{frame}{Hilfsabschätzung}
    
\end{frame}



\begin{frame}{Hilfsabschätzung}
    \begin{block}{Cauchy-Schwarzsche Ungleichung}
        Für alle \( x_{1}, x_{2}, \ldots, x_{m}, y_{1}, y_{2}, \ldots, y_{m} \in \bR \), \( m \in \bN \), gilt die Abschätzung:
        \begin{align*}
            \sum_{k = 1}^{m} x_{k} y_{k}
            \leq \sqrt{\sum_{k = 1}^{m} x_{k}^{2}} \cdot \sqrt{\sum_{k = 1}^{m} y_{k}^{2}}.
        \end{align*}
    \end{block}
\end{frame}



\begin{frame}{Beweis}
    
\end{frame}



\begin{frame}{Beweis}
    Seien \( x, y, z > 0 \) drei positive reelle Zahlen.
\end{frame}



\begin{frame}{Beweis}
    Seien \( x, y, z > 0 \) drei positive reelle Zahlen. Wir schätzen zunächst ab:
\end{frame}



\begin{frame}{Beweis}
    Seien \( x, y, z > 0 \) drei positive reelle Zahlen. Wir schätzen zunächst ab:
    \begin{align*}
        x + y + z
    \end{align*}
\end{frame}



\begin{frame}{Beweis}
    Seien \( x, y, z > 0 \) drei positive reelle Zahlen. Wir schätzen zunächst ab:
    \begin{align*}
        x + y + z
        & = \sqrt{y + z} \cdot \frac{x}{\sqrt{y + z}} + \sqrt{x + z} \cdot \frac{y}{\sqrt{x + z}} + \sqrt{x + y} \cdot \frac{z}{\sqrt{x + y}}
    \end{align*}
\end{frame}



\begin{frame}{Beweis}
    Seien \( x, y, z > 0 \) drei positive reelle Zahlen. Wir schätzen zunächst ab:
    {\footnotesize
    \begin{align*}
        x + y + z
        & = \sqrt{y + z} \cdot \frac{x}{\sqrt{y + z}} + \sqrt{x + z} \cdot \frac{y}{\sqrt{x + z}} + \sqrt{x + y} \cdot \frac{z}{\sqrt{x + y}} \\
        & \leq \sqrt{\left( \sqrt{y + z} \right)^{2} + \left( \sqrt{x + z} \right)^{2} + \left( \sqrt{x + y} \right)^{2}} \cdot \sqrt{\left( \frac{x}{\sqrt{y + z}}\right)^{2} + \left( \frac{y}{\sqrt{x + z}}\right)^{2} + \left( \frac{z}{\sqrt{x + y}}\right)^{2}}
    \end{align*}
    }
\end{frame}



\begin{frame}{Beweis}
    Seien \( x, y, z > 0 \) drei positive reelle Zahlen. Wir schätzen zunächst ab:
    {\footnotesize
    \begin{align*}
        x + y + z
        & = \sqrt{y + z} \cdot \frac{x}{\sqrt{y + z}} + \sqrt{x + z} \cdot \frac{y}{\sqrt{x + z}} + \sqrt{x + y} \cdot \frac{z}{\sqrt{x + y}} \\
        & \leq \sqrt{\left( \sqrt{y + z} \right)^{2} + \left( \sqrt{x + z} \right)^{2} + \left( \sqrt{x + y} \right)^{2}} \cdot \sqrt{\left( \frac{x}{\sqrt{y + z}}\right)^{2} + \left( \frac{y}{\sqrt{x + z}}\right)^{2} + \left( \frac{z}{\sqrt{x + y}}\right)^{2}} \\
        & = \sqrt{y + z + x + z + x + y} \cdot \sqrt{\frac{x^{2}}{y + z} + \frac{y^{2}}{x + z} + \frac{z^{2}}{x + y}}
    \end{align*}
    }
\end{frame}



\begin{frame}{Beweis}
    Seien \( x, y, z > 0 \) drei positive reelle Zahlen. Wir schätzen zunächst ab:
    {\footnotesize
    \begin{align*}
        x + y + z
        & = \sqrt{y + z} \cdot \frac{x}{\sqrt{y + z}} + \sqrt{x + z} \cdot \frac{y}{\sqrt{x + z}} + \sqrt{x + y} \cdot \frac{z}{\sqrt{x + y}} \\
        & \leq \sqrt{\left( \sqrt{y + z} \right)^{2} + \left( \sqrt{x + z} \right)^{2} + \left( \sqrt{x + y} \right)^{2}} \cdot \sqrt{\left( \frac{x}{\sqrt{y + z}}\right)^{2} + \left( \frac{y}{\sqrt{x + z}}\right)^{2} + \left( \frac{z}{\sqrt{x + y}}\right)^{2}} \\
        & = \sqrt{y + z + x + z + x + y} \cdot \sqrt{\frac{x^{2}}{y + z} + \frac{y^{2}}{x + z} + \frac{z^{2}}{x + y}} \\
        & = \sqrt{2} \cdot \sqrt{x + y + z} \cdot \sqrt{\frac{x^{2}}{y + z} + \frac{y^{2}}{x + z} + \frac{z^{2}}{x + y}}.
    \end{align*}
    }
\end{frame}



\begin{frame}{Beweis}
    Seien \( x, y, z > 0 \) drei positive reelle Zahlen. Wir schätzen zunächst ab:
    {\footnotesize
    \begin{align*}
        x + y + z
        & = \sqrt{y + z} \cdot \frac{x}{\sqrt{y + z}} + \sqrt{x + z} \cdot \frac{y}{\sqrt{x + z}} + \sqrt{x + y} \cdot \frac{z}{\sqrt{x + y}} \\
        & \leq \sqrt{\left( \sqrt{y + z} \right)^{2} + \left( \sqrt{x + z} \right)^{2} + \left( \sqrt{x + y} \right)^{2}} \cdot \sqrt{\left( \frac{x}{\sqrt{y + z}}\right)^{2} + \left( \frac{y}{\sqrt{x + z}}\right)^{2} + \left( \frac{z}{\sqrt{x + y}}\right)^{2}} \\
        & = \sqrt{y + z + x + z + x + y} \cdot \sqrt{\frac{x^{2}}{y + z} + \frac{y^{2}}{x + z} + \frac{z^{2}}{x + y}} \\
        & = \sqrt{2} \cdot \sqrt{x + y + z} \cdot \sqrt{\frac{x^{2}}{y + z} + \frac{y^{2}}{x + z} + \frac{z^{2}}{x + y}}.
    \end{align*}
    }
    Teilt man nun beide Seiten der Ungleichung durch \( \sqrt{x + y + z} \) so folgt die Abschätzung:
\end{frame}



\begin{frame}{Beweis}
    \begin{align*}
        \sqrt{x + y + z}
        \leq \sqrt{2} \cdot \sqrt{\frac{x^{2}}{y + z} + \frac{y^{2}}{x + z} + \frac{z^{2}}{x + y}}
    \end{align*}
\end{frame}



\begin{frame}{Beweis}
    \begin{align*}
        \sqrt{x + y + z}
        \leq \sqrt{2} \cdot \sqrt{\frac{x^{2}}{y + z} + \frac{y^{2}}{x + z} + \frac{z^{2}}{x + y}}.
    \end{align*}
    Die quadratische Funktion streng monoton wachsend für positive reelle Zahlen.
\end{frame}



\begin{frame}{Beweis}
    \begin{align*}
        \sqrt{x + y + z}
        \leq \sqrt{2} \cdot \sqrt{\frac{x^{2}}{y + z} + \frac{y^{2}}{x + z} + \frac{z^{2}}{x + y}}.
    \end{align*}
    Die quadratische Funktion streng monoton wachsend für positive reelle Zahlen. Damit folgt schließlich:
\end{frame}



\begin{frame}{Beweis}
    \begin{align*}
        \sqrt{x + y + z}
        \leq \sqrt{2} \cdot \sqrt{\frac{x^{2}}{y + z} + \frac{y^{2}}{x + z} + \frac{z^{2}}{x + y}}.
    \end{align*}
    Die quadratische Funktion streng monoton wachsend für positive reelle Zahlen. Damit folgt schließlich:
    \begin{align*}
        x + y + z
    \end{align*}
\end{frame}



\begin{frame}{Beweis}
    \begin{align*}
        \sqrt{x + y + z}
        \leq \sqrt{2} \cdot \sqrt{\frac{x^{2}}{y + z} + \frac{y^{2}}{x + z} + \frac{z^{2}}{x + y}}.
    \end{align*}
    Die quadratische Funktion streng monoton wachsend für positive reelle Zahlen. Damit folgt schließlich:
    \begin{align*}
        x + y + z
        & = \left( \sqrt{x + y + z} \right)^{2}
    \end{align*}
\end{frame}



\begin{frame}{Beweis}
    \begin{align*}
        \sqrt{x + y + z}
        \leq \sqrt{2} \cdot \sqrt{\frac{x^{2}}{y + z} + \frac{y^{2}}{x + z} + \frac{z^{2}}{x + y}}.
    \end{align*}
    Die quadratische Funktion streng monoton wachsend für positive reelle Zahlen. Damit folgt schließlich:
    \begin{align*}
        x + y + z
        & = \left( \sqrt{x + y + z} \right)^{2} \\
        & \leq \left( \sqrt{2} \cdot \sqrt{\frac{x^{2}}{y + z} + \frac{y^{2}}{x + z} + \frac{z^{2}}{x + y}} \right)^{2}
    \end{align*}
\end{frame}



\begin{frame}{Beweis}
    \begin{align*}
        \sqrt{x + y + z}
        \leq \sqrt{2} \cdot \sqrt{\frac{x^{2}}{y + z} + \frac{y^{2}}{x + z} + \frac{z^{2}}{x + y}}.
    \end{align*}
    Die quadratische Funktion streng monoton wachsend für positive reelle Zahlen. Damit folgt schließlich:
    \begin{align*}
        x + y + z
        & = \left( \sqrt{x + y + z} \right)^{2} \\
        & \leq \left( \sqrt{2} \cdot \sqrt{\frac{x^{2}}{y + z} + \frac{y^{2}}{x + z} + \frac{z^{2}}{x + y}} \right)^{2} \\
         & = 2 \cdot \left( \frac{x^{2}}{y + z} + \frac{y^{2}}{x + z} + \frac{z^{2}}{x + y} \right).
    \end{align*}
\end{frame}
% ============================================================

\end{document}