\documentclass[10pt]{beamer}

\title{}
\author{Artur's Mathematikstübchen}
\date{}


% ===== Packages =========
\usepackage[utf8]{inputenc}

\usepackage[natbibapa]{apacite}
\bibliographystyle{apacite}
\usepackage[ngerman]{babel}
\usepackage{graphicx}
\usepackage{fancyhdr}
\usepackage{amsmath}
\usepackage{amssymb}
\usepackage{graphicx}
\usepackage{MnSymbol}
\usepackage{tikz-cd}
\usepackage{enumitem}
\usepackage{amsthm}
\usepackage{mleftright}
\usepackage{dsfont}



\def\bD{\mathbb{D}}               
\def\bE{\mathbb{E}}
\def\bG{\mathbb{G}}
\def\bN{\mathbb{N}}
\def\bP{\mathbb{P}}
\def\bQ{\mathbb{Q}}
\def\bR{\mathbb{R}}
\def\bBarR{\bar{\mathbb{R}}}
\def\bY{\mathbb{Y}}



\def\mA{\mathcal{A}}
\def\mB{\mathcal{B}}
\def\mD{\mathcal{D}}
\def\mE{\mathcal{E}}
\def\mF{\mathcal{F}}
\def\mG{\mathcal{G}}
\def\mH{\mathcal{H}}
\def\mL{\mathcal{L}}
\def\mN{\mathcal{N}}
\def\mP{\mathcal{P}}
\def\mS{\mathcal{S}}
\def\mT{\mathcal{T}}
\def\mX{\mathcal{X}}
\def\mY{\mathcal{Y}}



\usetheme{Madrid}



\begin{document}



% ======================== Begrüßung ==================

\begin{frame}
    \begin{center}
        \textbf{\huge Willkommen in der guten Stube \newline \newline :D}
    \end{center}
\end{frame}
% =====================================================



% ======================== Präsentation ==================

\begin{frame}{}
    \begin{alertblock}{Aufgabe}
        Sei \( m \in \mathbb{N} \) eine beliebige natürliche Zahl. Man zeige für alle \( n \in \bN \) mit \( n \geq m \) die Abschätzung:
        \begin{align*}
            n
        	\geq \sum_{k = 0}^{m} \binom{n}{k} \cdot \left( \sqrt[n]{n} - 1 \right)^{k}.
        \end{align*}
    \end{alertblock}
\end{frame}



\begin{frame}{Hilfsmittel}
    \begin{itemize}
        \item<1-> Für alle \( x, y \in \bR \) und alle \( p \in \bN \) gilt:  
         \begin{align*}
             \left( x + y \right)^{p}
             = \sum_{k = 0}^{p} \binom{p}{k} \cdot x^{p - k} \cdot y^{k}.
         \end{align*} 
    \end{itemize}
\end{frame}



\begin{frame}{Beweis}
    Sei \( n \in \bN \) eine natürliche Zahl mit \( n \geq m \).
\end{frame}



\begin{frame}{Beweis}
    Sei \( n \in \bN \) eine natürliche Zahl mit \( n \geq m \). Zusammen mit dem binomischen Lehrsatz folgt:
\end{frame}



\begin{frame}{Beweis}
    Sei \( n \in \bN \) eine natürliche Zahl mit \( n \geq m \). Zusammen mit dem binomischen Lehrsatz folgt:
    \begin{align*}
        n
    \end{align*}
\end{frame}



\begin{frame}{Beweis}
    Sei \( n \in \bN \) eine natürliche Zahl mit \( n \geq m \). Zusammen mit dem binomischen Lehrsatz folgt:
    \begin{align*}
        n
        & = \left( \sqrt[n]{n} \right)^{n} \\
    \end{align*}
\end{frame}



\begin{frame}{Beweis}
    Sei \( n \in \bN \) eine natürliche Zahl mit \( n \geq m \). Zusammen mit dem binomischen Lehrsatz folgt:
    \begin{align*}
        n
        & = \left( \sqrt[n]{n} \right)^{n} \\
        & = \left( 1 + \sqrt[n]{n} - 1 \right)^{n} \\
    \end{align*}
\end{frame}



\begin{frame}{Beweis}
    Sei \( n \in \bN \) eine natürliche Zahl mit \( n \geq m \). Zusammen mit dem binomischen Lehrsatz folgt:
    \begin{align*}
        n
        & = \left( \sqrt[n]{n} \right)^{n} \\
        & = \left( 1 + \sqrt[n]{n} - 1 \right)^{n} \\
        & = \sum_{k = 0}^{n}\binom{n}{k} \cdot \left( \sqrt[n]{n} - 1 \right)^{k} \\
    \end{align*}
\end{frame}



\begin{frame}{Beweis}
    Sei \( n \in \bN \) eine natürliche Zahl mit \( n \geq m \). Zusammen mit dem binomischen Lehrsatz folgt:
    \begin{align*}
        n
        & = \left( \sqrt[n]{n} \right)^{n} \\
        & = \left( 1 + \sqrt[n]{n} - 1 \right)^{n} \\
        & = \sum_{k = 0}^{n} \binom{n}{k} \cdot \left( \sqrt[n]{n} - 1 \right)^{k} \\
        & = \sum_{k = 0}^{m} \binom{n}{k} \cdot \left( \sqrt[n]{n} - 1 \right)^{k} + \underbrace{\sum_{k = m + 1}^{n} \binom{n}{k} \cdot \left( \sqrt[n]{n} - 1 \right)^{k}}_{\text{\( \geq 0 \)}} \\
    \end{align*}
\end{frame}



\begin{frame}{Beweis}
    Sei \( n \in \bN \) eine natürliche Zahl mit \( n \geq m \). Zusammen mit dem binomischen Lehrsatz folgt:
    \begin{align*}
        n
        & = \left( \sqrt[n]{n} \right)^{n} \\
        & = \left( 1 + \sqrt[n]{n} - 1 \right)^{n} \\
        & = \sum_{k = 0}^{n} \binom{n}{k} \cdot \left( \sqrt[n]{n} - 1 \right)^{k} \\
        & = \sum_{k = 0}^{m} \binom{n}{k} \cdot \left( \sqrt[n]{n} - 1 \right)^{k} + \underbrace{\sum_{k = m + 1}^{n} \binom{n}{k} \cdot \left( \sqrt[n]{n} - 1 \right)^{k}}_{\text{\( \geq 0 \)}} \\
        & \geq \sum_{k = 0}^{m} \binom{n}{k} \cdot \left( \sqrt[n]{n} - 1 \right)^{k}.
    \end{align*}
\end{frame}



\begin{frame}{Folgerung 1}
    Für \( m = 1 \) folgt die Abschätzung:
\end{frame}



\begin{frame}{Folgerung 1}
    Für \( m = 1 \) folgt die Abschätzung:
    \begin{align*}
        n 
    \end{align*}
\end{frame}



\begin{frame}{Folgerung 1}
    Für \( m = 1 \) folgt die Abschätzung:
    \begin{align*}
        n 
        & \geq \sum_{k = 0}^{1} \binom{n}{k} \cdot \left( \sqrt[n]{n} - 1 \right)^{k} \\
    \end{align*}
\end{frame}



\begin{frame}{Folgerung 1}
    Für \( m = 1 \) folgt die Abschätzung:
    \begin{align*}
        n 
        & \geq \sum_{k = 0}^{1} \binom{n}{k} \cdot \left( \sqrt[n]{n} - 1 \right)^{k} \\
        & = \binom{n}{0} \cdot \left( \sqrt[n]{n} - 1 \right)^{0} + \binom{n}{1} \cdot \left( \sqrt[n]{n} - 1 \right)^{1} \\
    \end{align*}
\end{frame}



\begin{frame}{Folgerung 1}
    Für \( m = 1 \) folgt die Abschätzung:
    \begin{align*}
        n 
        & \geq \sum_{k = 0}^{1} \binom{n}{k} \cdot \left( \sqrt[n]{n} - 1 \right)^{k} \\
        & = \binom{n}{0} \cdot \left( \sqrt[n]{n} - 1 \right)^{0} + \binom{n}{1} \cdot \left( \sqrt[n]{n} - 1 \right)^{1} \\
        & = 1 + n \cdot \left( \sqrt[n]{n} - 1 \right).
    \end{align*}
\end{frame}



\begin{frame}{Folgerung 2}
    Für \( m = 2 \) folgt die Abschätzung:
\end{frame}



\begin{frame}{Folgerung 2}
    Für \( m = 2 \) folgt die Abschätzung:
    \begin{align*}
        n 
    \end{align*}
\end{frame}



\begin{frame}{Folgerung 2}
    Für \( m = 2 \) folgt die Abschätzung:
    \begin{align*}
        n 
        & \geq \sum_{k = 0}^{2} \binom{n}{k} \cdot \left( \sqrt[n]{n} - 1 \right)^{k} \\
    \end{align*}
\end{frame}



\begin{frame}{Folgerung 2}
    Für \( m = 2 \) folgt die Abschätzung:
    \begin{align*}
        n 
        & \geq \sum_{k = 0}^{2} \binom{n}{k} \cdot \left( \sqrt[n]{n} - 1 \right)^{k} \\
        & = \binom{n}{0} \cdot \left( \sqrt[n]{n} - 1 \right)^{0} + \binom{n}{1} \cdot \left( \sqrt[n]{n} - 1 \right)^{1} + \binom{n}{2} \cdot \left( \sqrt[n]{n} - 1 \right)^{2} \\
    \end{align*}
\end{frame}



\begin{frame}{Folgerung 2}
    Für \( m = 2 \) folgt die Abschätzung:
    \begin{align*}
        n 
        & \geq \sum_{k = 0}^{2} \binom{n}{k} \cdot \left( \sqrt[n]{n} - 1 \right)^{k} \\
        & = \binom{n}{0} \cdot \left( \sqrt[n]{n} - 1 \right)^{0} + \binom{n}{1} \cdot \left( \sqrt[n]{n} - 1 \right)^{1} + \binom{n}{2} \cdot \left( \sqrt[n]{n} - 1 \right)^{2} \\
        & > 1 + \binom{n}{2} \cdot \left( \sqrt[n]{n} - 1 \right)^{2}.
    \end{align*}
\end{frame}
% ============================================================

\end{document}