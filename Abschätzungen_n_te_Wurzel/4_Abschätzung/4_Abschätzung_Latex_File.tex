\documentclass[10pt]{beamer}

\title{}
\author{Artur's Mathematikstübchen}
\date{}


% ===== Packages =========
\usepackage[utf8]{inputenc}

\usepackage[natbibapa]{apacite}
\bibliographystyle{apacite}
\usepackage[ngerman]{babel}
\usepackage{graphicx}
\usepackage{fancyhdr}
\usepackage{amsmath}
\usepackage{amssymb}
\usepackage{graphicx}
\usepackage{MnSymbol}
\usepackage{tikz-cd}
\usepackage{enumitem}
\usepackage{amsthm}
\usepackage{mleftright}
\usepackage{dsfont}



\def\bD{\mathbb{D}}               
\def\bE{\mathbb{E}}
\def\bG{\mathbb{G}}
\def\bN{\mathbb{N}}
\def\bP{\mathbb{P}}
\def\bQ{\mathbb{Q}}
\def\bR{\mathbb{R}}
\def\bBarR{\bar{\mathbb{R}}}
\def\bY{\mathbb{Y}}



\def\mA{\mathcal{A}}
\def\mB{\mathcal{B}}
\def\mD{\mathcal{D}}
\def\mE{\mathcal{E}}
\def\mF{\mathcal{F}}
\def\mG{\mathcal{G}}
\def\mH{\mathcal{H}}
\def\mL{\mathcal{L}}
\def\mN{\mathcal{N}}
\def\mP{\mathcal{P}}
\def\mS{\mathcal{S}}
\def\mT{\mathcal{T}}
\def\mX{\mathcal{X}}
\def\mY{\mathcal{Y}}



\usetheme{Madrid}



\begin{document}



% ======================== Begrüßung ==================

\begin{frame}
    \begin{center}
        \textbf{\huge Willkommen in der guten Stube \newline \newline :D}
    \end{center}
\end{frame}
% =====================================================



% ======================== Präsentation ==================

\begin{frame}{}
    \begin{alertblock}{Aufgabe}
        Man zeige für alle \( n \in \bN \) mit \( n \geq 2 \) die Abschätzung:
        \begin{equation}
            \sqrt[n]{n}
        	\leq \sqrt{3 - \frac{2}{n}}.
        \end{equation}
    \end{alertblock}
\end{frame}



\begin{frame}{Hilfsabschätzung}
    \begin{itemize}
        \item<1->  Für den Beweis verwenden wir die Ungleichung zwischen dem geometrischen und quadratischen Mittel:   
        \item<2-> 
        \begin{block}{Hilfsabschätzung}
             Für alle \( x_{1}, \ldots, x_{m} > 0 \), \( m \in \bN \), gilt die Abschätzung:
             \begin{align*}
                \sqrt[m]{x_{1} \cdot \ldots \cdot x_{m}} 
                \leq \sqrt{\frac{x_{1}^{2} + \ldots + x_{m}^{2}}{m}}.
             \end{align*}
        \end{block} 
    \end{itemize}
\end{frame}



\begin{frame}{Beweis}
    Sei \( n \in \bN \) eine natürliche Zahl mit \( n \geq 2 \).
\end{frame}



\begin{frame}{Beweis}
    Sei \( n \in \bN \) eine natürliche Zahl mit \( n \geq m \). Wir schreiben \( n = \sqrt{n} \cdot \sqrt{n} \cdot \underbrace{ 1 \cdot \ldots \cdot 1}_{\text{ \( \left( n - 2 \right) \)-mal}} \) und schätzen mit der Ungleichung zwischen dem geometrischen und quadratischen Mittel ab:
\end{frame}


\begin{frame}{Beweis}
    Sei \( n \in \bN \) eine natürliche Zahl mit \( n \geq m \). Wir schreiben \( n = \sqrt{n} \cdot \sqrt{n} \cdot \underbrace{ 1 \cdot \ldots \cdot 1}_{\text{ \( \left( n - 2 \right) \)-mal}} \) und schätzen mit der Ungleichung zwischen dem geometrischen und quadratischen Mittel ab:
    \begin{align*}
        \sqrt[n]{n}
    \end{align*}
\end{frame}



\begin{frame}{Beweis}
    Sei \( n \in \bN \) eine natürliche Zahl mit \( n \geq m \). Wir schreiben \( n = \sqrt{n} \cdot \sqrt{n} \cdot \underbrace{ 1 \cdot \ldots \cdot 1}_{\text{ \( \left( n - 2 \right) \)-mal}} \) und schätzen mit der Ungleichung zwischen dem geometrischen und quadratischen Mittel ab:
    \begin{align*}
        \sqrt[n]{n}
        & = \sqrt[n]{\sqrt{n} \cdot \sqrt{n} \cdot 1 \cdot \ldots \cdot 1} \\
    \end{align*}
\end{frame}



\begin{frame}{Beweis}
    Sei \( n \in \bN \) eine natürliche Zahl mit \( n \geq m \). Wir schreiben \( n = \sqrt{n} \cdot \sqrt{n} \cdot \underbrace{ 1 \cdot \ldots \cdot 1}_{\text{ \( \left( n - 2 \right) \)-mal}} \) und schätzen mit der Ungleichung zwischen dem geometrischen und quadratischen Mittel ab:
    \begin{align*}
        \sqrt[n]{n}
        & = \sqrt[n]{\sqrt{n} \cdot \sqrt{n} \cdot 1 \cdot \ldots \cdot 1} \\
        & \leq \sqrt{\frac{\left( \sqrt{n} \right)^{2} + \left( \sqrt{n} \right)^{2} + 1^{2} + \ldots + 1^{2}}{n}} \\
    \end{align*}
\end{frame}



\begin{frame}{Beweis}
    Sei \( n \in \bN \) eine natürliche Zahl mit \( n \geq m \). Wir schreiben \( n = \sqrt{n} \cdot \sqrt{n} \cdot \underbrace{ 1 \cdot \ldots \cdot 1}_{\text{ \( \left( n - 2 \right) \)-mal}} \) und schätzen mit der Ungleichung zwischen dem geometrischen und quadratischen Mittel ab:
    \begin{align*}
        \sqrt[n]{n}
        & = \sqrt[n]{\sqrt{n} \cdot \sqrt{n} \cdot 1 \cdot \ldots \cdot 1} \\
        & \leq \sqrt{\frac{\left( \sqrt{n} \right)^{2} + \left( \sqrt{n} \right)^{2} + 1^{2} + \ldots + 1^{2}}{n}} \\
        & = \sqrt{\frac{n + n + 1 + \ldots + 1}{n}} \\
    \end{align*}
\end{frame}



\begin{frame}{Beweis}
    Sei \( n \in \bN \) eine natürliche Zahl mit \( n \geq m \). Wir schreiben \( n = \sqrt{n} \cdot \sqrt{n} \cdot \underbrace{ 1 \cdot \ldots \cdot 1}_{\text{ \( \left( n - 2 \right) \)-mal}} \) und schätzen mit der Ungleichung zwischen dem geometrischen und quadratischen Mittel ab:
    \begin{align*}
        \sqrt[n]{n}
        & = \sqrt[n]{\sqrt{n} \cdot \sqrt{n} \cdot 1 \cdot \ldots \cdot 1} \\
        & \leq \sqrt{\frac{\left( \sqrt{n} \right)^{2} + \left( \sqrt{n} \right)^{2} + 1^{2} + \ldots + 1^{2}}{n}} \\
        & = \sqrt{\frac{n + n + 1 + \ldots + 1}{n}} \\
        & = \sqrt{\frac{2n + n -2}{n}} \\
    \end{align*}
\end{frame}



\begin{frame}{Beweis}
    Sei \( n \in \bN \) eine natürliche Zahl mit \( n \geq m \). Wir schreiben \( n = \sqrt{n} \cdot \sqrt{n} \cdot \underbrace{ 1 \cdot \ldots \cdot 1}_{\text{ \( \left( n - 2 \right) \)-mal}} \) und schätzen mit der Ungleichung zwischen dem geometrischen und quadratischen Mittel ab:
    \begin{align*}
        \sqrt[n]{n}
        & = \sqrt[n]{\sqrt{n} \cdot \sqrt{n} \cdot 1 \cdot \ldots \cdot 1} \\
        & \leq \sqrt{\frac{\left( \sqrt{n} \right)^{2} + \left( \sqrt{n} \right)^{2} + 1^{2} + \ldots + 1^{2}}{n}} \\
        & = \sqrt{\frac{n + n + 1 + \ldots + 1}{n}} \\
        & = \sqrt{\frac{2n + n -2}{n}} \\
        & = \sqrt{\frac{3n -2}{n}} \\
    \end{align*}
\end{frame}



\begin{frame}{Beweis}
    Sei \( n \in \bN \) eine natürliche Zahl mit \( n \geq m \). Wir schreiben \( n = \sqrt{n} \cdot \sqrt{n} \cdot \underbrace{ 1 \cdot \ldots \cdot 1}_{\text{ \( \left( n - 2 \right) \)-mal}} \) und schätzen mit der Ungleichung zwischen dem geometrischen und quadratischen Mittel ab:
    \begin{align*}
        \sqrt[n]{n}
        & = \sqrt[n]{\sqrt{n} \cdot \sqrt{n} \cdot 1 \cdot \ldots \cdot 1} \\
        & \leq \sqrt{\frac{\left( \sqrt{n} \right)^{2} + \left( \sqrt{n} \right)^{2} + 1^{2} + \ldots + 1^{2}}{n}} \\
        & = \sqrt{\frac{n + n + 1 + \ldots + 1}{n}} \\
        & = \sqrt{\frac{2n + n -2}{n}} \\
        & = \sqrt{\frac{3n -2}{n}} \\
        & = \sqrt{3 - \frac{2}{n}}.
    \end{align*}
\end{frame}



\begin{frame}{Bemerkung 1}
    Die Abschätzung die wir gezeigt haben, gilt für alle \( n \in \bN \). 
\end{frame}



\begin{frame}{Bemerkung 1}
    Die Abschätzung die wir gezeigt haben, gilt für alle \( n \in \bN \). Den es ist: 
\end{frame}



\begin{frame}{Bemerkung 1}
    Die Abschätzung die wir gezeigt haben, gilt für alle \( n \in \bN \). Den es ist: 
    \begin{align*}
        \sqrt[1]{1}
    \end{align*}
\end{frame}



\begin{frame}{Bemerkung 1}
    Die Abschätzung die wir gezeigt haben, gilt für alle \( n \in \bN \). Den es ist: 
    \begin{align*}
        \sqrt[1]{1}
        = 1
    \end{align*}
\end{frame}



\begin{frame}{Bemerkung 1}
    Die Abschätzung die wir gezeigt haben, gilt für alle \( n \in \bN \). Den es ist: 
    \begin{align*}
        \sqrt[1]{1}
        = 1
        = \sqrt{3 - \frac{2}{1}}.
    \end{align*}
\end{frame}



\begin{frame}{Bemerkung 2}
    Es gilt zusätzlich die Abschätzung:
\end{frame}



\begin{frame}{Bemerkung 2}
    Es gilt zusätzlich die Abschätzung:
    \begin{align*}
        AM( x_{1}, \ldots, x_{m} )
        = \frac{x_{1} + \ldots + x_{m}}{m}
        \leq \sqrt{\frac{x_{1}^{2} + \ldots + x_{n}^{2}}{m}} 
      	= QM( x_{1}, \ldots, x_{m} ).
    \end{align*}
\end{frame}



\begin{frame}{Bemerkung 2}
    Es gilt zusätzlich die Abschätzung:
    \begin{align*}
        AM( x_{1}, \ldots, x_{m} )
        = \frac{x_{1} + \ldots + x_{m}}{m}
        \leq \sqrt{\frac{x_{1}^{2} + \ldots + x_{n}^{2}}{m}} 
      	= QM( x_{1}, \ldots, x_{m} ).
    \end{align*}
    Daraus folgt:
\end{frame}



\begin{frame}{Bemerkung 2}
    Es gilt zusätzlich die Abschätzung:
    \begin{align*}
        AM( x_{1}, \ldots, x_{m} )
        = \frac{x_{1} + \ldots + x_{m}}{m}
        \leq \sqrt{\frac{x_{1}^{2} + \ldots + x_{n}^{2}}{m}} 
      	= QM( x_{1}, \ldots, x_{m} ).
    \end{align*}
    Daraus folgt:
    \begin{align*}
        1 - \frac{2}{n} + \frac{2}{\sqrt{n}}
    \end{align*}
\end{frame}



\begin{frame}{Bemerkung 2}
    Es gilt zusätzlich die Abschätzung:
    \begin{align*}
        AM( x_{1}, \ldots, x_{m} )
        = \frac{x_{1} + \ldots + x_{m}}{m}
        \leq \sqrt{\frac{x_{1}^{2} + \ldots + x_{n}^{2}}{m}} 
      	= QM( x_{1}, \ldots, x_{m} ).
    \end{align*}
    Daraus folgt:
    \begin{align*}
        1 - \frac{2}{n} + \frac{2}{\sqrt{n}}
        & = \frac{\sqrt{n} + \sqrt{n} + 1 \ldots + 1}{n} \\
    \end{align*}
\end{frame}



\begin{frame}{Bemerkung 2}
    Es gilt zusätzlich die Abschätzung:
    \begin{align*}
        AM( x_{1}, \ldots, x_{m} )
        = \frac{x_{1} + \ldots + x_{m}}{m}
        \leq \sqrt{\frac{x_{1}^{2} + \ldots + x_{n}^{2}}{m}} 
      	= QM( x_{1}, \ldots, x_{m} ).
    \end{align*}
    Daraus folgt:
    \begin{align*}
        1 - \frac{2}{n} + \frac{2}{\sqrt{n}}
        & = \frac{\sqrt{n} + \sqrt{n} + 1 \ldots + 1}{n} \\
        & = AM(\sqrt{n}, \sqrt{n}, 1, \ldots, 1) \\
    \end{align*}
\end{frame}



\begin{frame}{Bemerkung 2}
    Es gilt zusätzlich die Abschätzung:
    \begin{align*}
        AM( x_{1}, \ldots, x_{m} )
        = \frac{x_{1} + \ldots + x_{m}}{m}
        \leq \sqrt{\frac{x_{1}^{2} + \ldots + x_{n}^{2}}{m}} 
      	= QM( x_{1}, \ldots, x_{m} ).
    \end{align*}
    Daraus folgt:
    \begin{align*}
        1 - \frac{2}{n} + \frac{2}{\sqrt{n}}
        & = \frac{\sqrt{n} + \sqrt{n} + 1 \ldots + 1}{n} \\
        & = AM(\sqrt{n}, \sqrt{n}, 1, \ldots, 1) \\
        & \leq QM(\sqrt{n}, \sqrt{n}, 1, \ldots, 1) \\
    \end{align*}
\end{frame}



\begin{frame}{Bemerkung 2}
    Es gilt zusätzlich die Abschätzung:
    \begin{align*}
        AM( x_{1}, \ldots, x_{m} )
        = \frac{x_{1} + \ldots + x_{m}}{m}
        \leq \sqrt{\frac{x_{1}^{2} + \ldots + x_{n}^{2}}{m}} 
      	= QM( x_{1}, \ldots, x_{m} ).
    \end{align*}
    Daraus folgt:
    \begin{align*}
        1 - \frac{2}{n} + \frac{2}{\sqrt{n}}
        & = \frac{\sqrt{n} + \sqrt{n} + 1 \ldots + 1}{n} \\
        & = AM(\sqrt{n}, \sqrt{n}, 1, \ldots, 1) \\
        & \leq QM(\sqrt{n}, \sqrt{n}, 1, \ldots, 1) \\
        & = \sqrt{\frac{\left( \sqrt{n} \right)^{2} + \left( \sqrt{n} \right)^{2} + 1^{2} + \ldots + 1^{2}}{n}} \\
    \end{align*}
\end{frame}



\begin{frame}{Bemerkung 2}
    Es gilt zusätzlich die Abschätzung:
    \begin{align*}
        AM( x_{1}, \ldots, x_{m} )
        = \frac{x_{1} + \ldots + x_{m}}{m}
        \leq \sqrt{\frac{x_{1}^{2} + \ldots + x_{n}^{2}}{m}} 
      	= QM( x_{1}, \ldots, x_{m} ).
    \end{align*}
    Daraus folgt:
    \begin{align*}
        1 - \frac{2}{n} + \frac{2}{\sqrt{n}}
        & = \frac{\sqrt{n} + \sqrt{n} + 1 \ldots + 1}{n} \\
        & = AM(\sqrt{n}, \sqrt{n}, 1, \ldots, 1) \\
        & \leq QM(\sqrt{n}, \sqrt{n}, 1, \ldots, 1) \\
        & = \sqrt{\frac{\left( \sqrt{n} \right)^{2} + \left( \sqrt{n} \right)^{2} + 1^{2} + \ldots + 1^{2}}{n}} \\
        & = \sqrt{3 - \frac{2}{n}}.
    \end{align*}
\end{frame}



\begin{frame}{Folgerung 1}
    Die Wurzelfunktion \( x \mapsto \sqrt{x} \) ist streng monoton steigend auf dem Intervall \( \left[0, \infty \right) \). 
\end{frame}



\begin{frame}{Folgerung 1}
    Die Wurzelfunktion \( x \mapsto \sqrt{x} \) ist streng monoton steigend auf dem Intervall \( \left[0, \infty \right) \). Mit der eben gezeigten Abschätzung erhalten wir für alle \( n \in \bN \) die Abschätzung: 
\end{frame}



\begin{frame}{Folgerung 1}
    Die Wurzelfunktion \( x \mapsto \sqrt{x} \) ist streng monoton steigend auf dem Intervall \( \left[0, \infty \right) \). Mit der eben gezeigten Abschätzung erhalten wir für alle \( n \in \bN \) die Abschätzung:
    \begin{align*}
        \sqrt[n]{n}
    \end{align*}
\end{frame}



\begin{frame}{Folgerung 1}
    Die Wurzelfunktion \( x \mapsto \sqrt{x} \) ist streng monoton steigend auf dem Intervall \( \left[0, \infty \right) \). Mit der eben gezeigten Abschätzung erhalten wir für alle \( n \in \bN \) die Abschätzung:
    \begin{align*}
        \sqrt[n]{n}
        \leq \sqrt{3 - \frac{2}{n}}
    \end{align*}
\end{frame}



\begin{frame}{Folgerung 1}
    Die Wurzelfunktion \( x \mapsto \sqrt{x} \) ist streng monoton steigend auf dem Intervall \( \left[0, \infty \right) \). Mit der eben gezeigten Abschätzung erhalten wir für alle \( n \in \bN \) die Abschätzung:
    \begin{align*}
        \sqrt[n]{n}
        \leq \sqrt{3 - \frac{2}{n}}
        < \sqrt{3}.
    \end{align*}
\end{frame}



\begin{frame}{Folgerung 2}
    Weiter gilt für den Grenzwert der Folge \( w_{n} = \sqrt[n]{n} \):
\end{frame}



\begin{frame}{Folgerung 2}
    Weiter gilt für den Grenzwert der Folge \( w_{n} = \sqrt[n]{n} \):
    \begin{align*}
        \lim_{n \to \infty} \sqrt[n]{n}
    \end{align*}
\end{frame}



\begin{frame}{Folgerung 2}
    Weiter gilt für den Grenzwert der Folge \( w_{n} = \sqrt[n]{n} \):
    \begin{align*}
        \lim_{n \to \infty} \sqrt[n]{n}
        \leq \lim_{n \to \infty} \sqrt{3 - \frac{2}{n}}
    \end{align*}
\end{frame}



\begin{frame}{Folgerung 2}
    Weiter gilt für den Grenzwert der Folge \( w_{n} = \sqrt[n]{n} \):
    \begin{align*}
        \lim_{n \to \infty} \sqrt[n]{n}
        \leq \lim_{n \to \infty} \sqrt{3 - \frac{2}{n}}
        = \sqrt{3}.
    \end{align*}
\end{frame}
% ============================================================

\end{document}