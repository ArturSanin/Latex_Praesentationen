\documentclass[10pt]{beamer}

\title{}
\author{Artur's Mathematikstübchen}
\date{}


% ===== Packages =========
\usepackage[utf8]{inputenc}

\usepackage[natbibapa]{apacite}
\bibliographystyle{apacite}
\usepackage[ngerman]{babel}
\usepackage{graphicx}
\usepackage{fancyhdr}
\usepackage{amsmath}
\usepackage{amssymb}
\usepackage{graphicx}
\usepackage{MnSymbol}
\usepackage{tikz-cd}
\usepackage{enumitem}
\usepackage{amsthm}
\usepackage{mleftright}
\usepackage{dsfont}



\def\bD{\mathbb{D}}               
\def\bE{\mathbb{E}}
\def\bG{\mathbb{G}}
\def\bN{\mathbb{N}}
\def\bP{\mathbb{P}}
\def\bQ{\mathbb{Q}}
\def\bR{\mathbb{R}}
\def\bBarR{\bar{\mathbb{R}}}
\def\bY{\mathbb{Y}}



\def\mA{\mathcal{A}}
\def\mB{\mathcal{B}}
\def\mD{\mathcal{D}}
\def\mE{\mathcal{E}}
\def\mF{\mathcal{F}}
\def\mG{\mathcal{G}}
\def\mH{\mathcal{H}}
\def\mL{\mathcal{L}}
\def\mN{\mathcal{N}}
\def\mP{\mathcal{P}}
\def\mS{\mathcal{S}}
\def\mT{\mathcal{T}}
\def\mX{\mathcal{X}}
\def\mY{\mathcal{Y}}



\usetheme{Madrid}



\begin{document}



% ======================== Begrüßung ==================

\begin{frame}
    \begin{center}
        \textbf{\huge Willkommen in der guten Stube \newline \newline :D}
    \end{center}
\end{frame}
% =====================================================



% ======================== Präsentation ==================

\begin{frame}{}
    \begin{alertblock}{Aufgabe}
        Man zeige für alle \( n \in \bN \) die Abschätzung:
        \begin{align*}
            \sqrt[n]{n}
        	\geq \frac{n^{2}}{n^{2} - n + 1}.
        \end{align*}
    \end{alertblock}
\end{frame}



\begin{frame}{Hilfsabschätzung}
    \begin{itemize}
        \item<1-> Für den Beweis verwenden wir die Ungleichung zwischen dem harmonischen und geometrischen Mittel: 
        \item<2->
        \begin{block}{Hilfsabschätzung}
            Für alle \( x_{1}, \ldots, x_{m} > 0 \), \( m \in \bN \), gilt die Abschätzung:
            \begin{align*}
                \sqrt[m]{x_{1} \cdot \ldots \cdot x_{m}}
                \geq \frac{m}{\frac{1}{x_{1}} + \ldots + \frac{1}{x_{m}}}.
            \end{align*}
        \end{block}
    \end{itemize}
\end{frame}



\begin{frame}{Beweis}
    Sei \( n \in \bN \) eine beliebige natürliche Zahl.
\end{frame}



\begin{frame}{Beweis}
    Sei \( n \in \bN \) eine beliebige natürliche Zahl. Wir schreiben 
    \[ 
        n = n \cdot \underbrace{1 \cdot \ldots \cdot 1}_{\text{\( \left( n - 1 \right) \)-mal}} 
    \] 
    und schätzen mit der Ungleichung zwischen dem harmonischen und geometrischen Mittel ab:
\end{frame}



\begin{frame}{Beweis}
    Sei \( n \in \bN \) eine beliebige natürliche Zahl. Wir schreiben
    \[ 
        n = n \cdot \underbrace{1 \cdot \ldots \cdot 1}_{\text{\( \left( n - 1 \right) \)-mal}} 
    \] 
    und schätzen mit der Ungleichung zwischen dem harmonischen und geometrischen Mittel ab:
    \begin{align*}
        \sqrt[n]{n}
    \end{align*}
\end{frame}



\begin{frame}{Beweis}
    Sei \( n \in \bN \) eine beliebige natürliche Zahl. Wir schreiben
    \[ 
        n = n \cdot \underbrace{1 \cdot \ldots \cdot 1}_{\text{\( \left( n - 1 \right) \)-mal}} 
    \] 
    und schätzen mit der Ungleichung zwischen dem harmonischen und geometrischen Mittel ab:
    \begin{align*}
        \sqrt[n]{n}
        & = \sqrt[n]{n \cdot 1 \cdot \ldots \cdot 1} \\
    \end{align*}
\end{frame}



\begin{frame}{Beweis}
    Sei \( n \in \bN \) eine beliebige natürliche Zahl. Wir schreiben
    \[ 
        n = n \cdot \underbrace{1 \cdot \ldots \cdot 1}_{\text{\( \left( n - 1 \right) \)-mal}} 
    \] 
    und schätzen mit der Ungleichung zwischen dem harmonischen und geometrischen Mittel ab:
    \begin{align*}
        \sqrt[n]{n}
        & = \sqrt[n]{n \cdot 1 \cdot \ldots \cdot 1} \\
        & \geq \frac{n}{\frac{1}{n} + \frac{1}{1} + \frac{1}{1} + \ldots + \frac{1}{1}} \\
    \end{align*}
\end{frame}



\begin{frame}{Beweis}
    Sei \( n \in \bN \) eine beliebige natürliche Zahl. Wir schreiben
    \[ 
        n = n \cdot \underbrace{1 \cdot \ldots \cdot 1}_{\text{\( \left( n - 1 \right) \)-mal}} 
    \] 
    und schätzen mit der Ungleichung zwischen dem harmonischen und geometrischen Mittel ab:
    \begin{align*}
        \sqrt[n]{n}
        & = \sqrt[n]{n \cdot 1 \cdot \ldots \cdot 1} \\
        & \geq \frac{n}{\frac{1}{n} + \frac{1}{1} + \frac{1}{1} + \ldots + \frac{1}{1}} \\
        & =	\frac{n}{\frac{1}{n} + 1 + 1 + \ldots + 1} \\
    \end{align*}
\end{frame}



\begin{frame}{Beweis}
    Sei \( n \in \bN \) eine beliebige natürliche Zahl. Wir schreiben
    \[ 
        n = n \cdot \underbrace{1 \cdot \ldots \cdot 1}_{\text{\( \left( n - 1 \right) \)-mal}} 
    \] 
    und schätzen mit der Ungleichung zwischen dem harmonischen und geometrischen Mittel ab:
    \begin{align*}
        \sqrt[n]{n}
        & = \sqrt[n]{n \cdot 1 \cdot \ldots \cdot 1} \\
        & \geq \frac{n}{\frac{1}{n} + \frac{1}{1} + \frac{1}{1} + \ldots + \frac{1}{1}} \\
        & =	\frac{n}{\frac{1}{n} + 1 + 1 + \ldots + 1} \\
        & = \frac{n}{\frac{1}{n} + n - 1} \\
    \end{align*}
\end{frame}



\begin{frame}{Beweis}
    Sei \( n \in \bN \) eine beliebige natürliche Zahl. Wir schreiben
    \[ 
        n = n \cdot \underbrace{1 \cdot \ldots \cdot 1}_{\text{\( \left( n - 1 \right) \)-mal}} 
    \] 
    und schätzen mit der Ungleichung zwischen dem harmonischen und geometrischen Mittel ab:
    \begin{align*}
        \sqrt[n]{n}
        & = \sqrt[n]{n \cdot 1 \cdot \ldots \cdot 1} \\
        & \geq \frac{n}{\frac{1}{n} + \frac{1}{1} + \frac{1}{1} + \ldots + \frac{1}{1}} \\
        & =	\frac{n}{\frac{1}{n} + 1 + 1 + \ldots + 1} \\
        & = \frac{n}{\frac{1}{n} + n - 1} \\
		& = \frac{n^{2}}{n^{2} - n + 1}.
    \end{align*}
\end{frame}
% ============================================================

\end{document}