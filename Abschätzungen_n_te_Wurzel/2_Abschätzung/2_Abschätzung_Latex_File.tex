\documentclass[10pt]{beamer}

\title{}
\author{Artur's Mathematikstübchen}
\date{}


% ===== Packages =========
\usepackage[utf8]{inputenc}

\usepackage[natbibapa]{apacite}
\bibliographystyle{apacite}
\usepackage[ngerman]{babel}
\usepackage{graphicx}
\usepackage{fancyhdr}
\usepackage{amsmath}
\usepackage{amssymb}
\usepackage{graphicx}
\usepackage{MnSymbol}
\usepackage{tikz-cd}
\usepackage{enumitem}
\usepackage{amsthm}
\usepackage{mleftright}
\usepackage{dsfont}



\def\bD{\mathbb{D}}               
\def\bE{\mathbb{E}}
\def\bG{\mathbb{G}}
\def\bN{\mathbb{N}}
\def\bP{\mathbb{P}}
\def\bQ{\mathbb{Q}}
\def\bR{\mathbb{R}}
\def\bBarR{\bar{\mathbb{R}}}
\def\bY{\mathbb{Y}}



\def\mA{\mathcal{A}}
\def\mB{\mathcal{B}}
\def\mD{\mathcal{D}}
\def\mE{\mathcal{E}}
\def\mF{\mathcal{F}}
\def\mG{\mathcal{G}}
\def\mH{\mathcal{H}}
\def\mL{\mathcal{L}}
\def\mN{\mathcal{N}}
\def\mP{\mathcal{P}}
\def\mS{\mathcal{S}}
\def\mT{\mathcal{T}}
\def\mX{\mathcal{X}}
\def\mY{\mathcal{Y}}



\usetheme{Madrid}



\begin{document}



% ======================== Begrüßung ==================

\begin{frame}
    \begin{center}
        \textbf{\huge Willkommen in der guten Stube \newline \newline :D}
    \end{center}
\end{frame}
% =====================================================



% ======================== Präsentation ==================

\begin{frame}{}
    \begin{alertblock}{Aufgabe}
        Man zeige für alle \( n \in \bN \) die Gültigkeit der Abschätzung:
        \begin{align*}
            \sqrt[n]{n}
        	\leq 1 - \frac{2}{n} + \frac{2}{\sqrt{n}}.
        \end{align*}
    \end{alertblock}
\end{frame}



\begin{frame}{Hilfsabschätzung}
    \begin{itemize}
        \item<1->  Für den Beweis verwenden wir als Hilfsmittel die Ungleichung zwischen dem geometrischen und arithmetischen Mittel:   
        \item<2-> 
        \begin{block}{Hilfsabschätzung}
             Für alle \( x_{1}, \ldots, x_{m} > 0 \), \( m \in \bN \), gilt die Abschätzung:
             \[
                \sqrt[m]{x_{1} \cdot \ldots \cdot x_{m}} 
                \leq \frac{x_{1} + \ldots + x_{m}}{m}.
             \]
        \end{block} 
    \end{itemize}
\end{frame}



\begin{frame}{Beweis}
    Sei \( n \in \bN \) eine beliebige natürliche Zahl.
\end{frame}



\begin{frame}{Beweis}
    Sei \( n \in \bN \) eine beliebige natürliche Zahl. Wir schreiben \( n = \sqrt{n} \cdot \sqrt{n} \cdot \underbrace{1 \cdot \ldots \cdot 1}_{\text{\( \left( n - 2 \right) \)-mal}} \) und schätzen mit der Ungleichung zwischen dem geometrischen und arithmetischen Mittel ab:
\end{frame}


\begin{frame}{Beweis}
    Sei \( n \in \bN \) eine beliebige natürliche Zahl. Wir schreiben \( n = \sqrt{n} \cdot \sqrt{n} \cdot \underbrace{1 \cdot \ldots \cdot 1}_{\text{\( \left( n - 2 \right) \)-mal}} \) und schätzen mit der Ungleichung zwischen dem geometrischen und arithmetischen Mittel ab:
    \begin{align*}
        \sqrt[n]{n}
    \end{align*}
\end{frame}



\begin{frame}{Beweis}
    Sei \( n \in \bN \) eine beliebige natürliche Zahl. Wir schreiben \( n = \sqrt{n} \cdot \sqrt{n} \cdot \underbrace{1 \cdot \ldots \cdot 1}_{\text{\( \left( n - 2 \right) \)-mal}} \) und schätzen mit der Ungleichung zwischen dem geometrischen und arithmetischen Mittel ab:
    \begin{align*}
        \sqrt[n]{n}
        & = \sqrt[n]{\sqrt{n} \cdot \sqrt{n} \cdot 1 \ldots \cdot 1} \\
    \end{align*}
\end{frame}



\begin{frame}{Beweis}
    Sei \( n \in \bN \) eine beliebige natürliche Zahl. Wir schreiben \( n = \sqrt{n} \cdot \sqrt{n} \cdot \underbrace{1 \cdot \ldots \cdot 1}_{\text{\( \left( n - 2 \right) \)-mal}} \) und schätzen mit der Ungleichung zwischen dem geometrischen und arithmetischen Mittel ab:
    \begin{align*}
        \sqrt[n]{n}
        & = \sqrt[n]{\sqrt{n} \cdot \sqrt{n} \cdot 1 \ldots \cdot 1} \\
        & \leq \frac{\sqrt{n} + \sqrt{n} + 1 + \ldots + 1}{n} \\
    \end{align*}
\end{frame}




\begin{frame}{Beweis}
    Sei \( n \in \bN \) eine beliebige natürliche Zahl. Wir schreiben \( n = \sqrt{n} \cdot \sqrt{n} \cdot \underbrace{1 \cdot \ldots \cdot 1}_{\text{\( \left( n - 2 \right) \)-mal}} \) und schätzen mit der Ungleichung zwischen dem geometrischen und arithmetischen Mittel ab:
    \begin{align*}
        \sqrt[n]{n}
        & = \sqrt[n]{\sqrt{n} \cdot \sqrt{n} \cdot 1 \ldots \cdot 1} \\
        & \leq \frac{\sqrt{n} + \sqrt{n} + 1 + \ldots + 1}{n} \\
        & = \frac{2 \cdot \sqrt{n} + n - 2}{n} \\
    \end{align*}
\end{frame}



\begin{frame}{Beweis}
    Sei \( n \in \bN \) eine beliebige natürliche Zahl. Wir schreiben \( n = \sqrt{n} \cdot \sqrt{n} \cdot \underbrace{1 \cdot \ldots \cdot 1}_{\text{\( \left( n - 2 \right) \)-mal}} \) und schätzen mit der Ungleichung zwischen dem geometrischen und arithmetischen Mittel ab:
    \begin{align*}
        \sqrt[n]{n}
        & = \sqrt[n]{\sqrt{n} \cdot \sqrt{n} \cdot 1 \ldots \cdot 1} \\
        & \leq \frac{\sqrt{n} + \sqrt{n} + 1 + \ldots + 1}{n} \\
        & = \frac{2 \cdot \sqrt{n} + n - 2}{n} \\
        & = 1 - \frac{2}{n} + \frac{2 \cdot \sqrt{n}}{n} \\
    \end{align*}
\end{frame}




\begin{frame}{Beweis}
    Sei \( n \in \bN \) eine beliebige natürliche Zahl. Wir schreiben \( n = \sqrt{n} \cdot \sqrt{n} \cdot \underbrace{1 \cdot \ldots \cdot 1}_{\text{\( \left( n - 2 \right) \)-mal}} \) und schätzen mit der Ungleichung zwischen dem geometrischen und arithmetischen Mittel ab:
    \begin{align*}
        \sqrt[n]{n}
        & = \sqrt[n]{\sqrt{n} \cdot \sqrt{n} \cdot 1 \ldots \cdot 1} \\
        & \leq \frac{\sqrt{n} + \sqrt{n} + 1 + \ldots + 1}{n} \\
        & = \frac{2 \cdot \sqrt{n} + n - 2}{n} \\
        & = 1 - \frac{2}{n} + \frac{2 \cdot \sqrt{n}}{n} \\
        & = 1 - \frac{2}{n} + \frac{2}{\sqrt{n}}.
    \end{align*}
\end{frame}



\begin{frame}{Folgerung 1}
    Aus der eben gezeigten Abschätzung erhalten wir insbesondere die Abschätzung:
\end{frame}



\begin{frame}{Folgerung 1}
    Aus der eben gezeigten Abschätzung erhalten wir insbesondere die Abschätzung:
    \[
        \sqrt[n]{n}
    \]
\end{frame}



\begin{frame}{Folgerung 1}
    Aus der eben gezeigten Abschätzung erhalten wir insbesondere die Abschätzung:
    \[
        \sqrt[n]{n}
        \leq 1 - \frac{2}{n} + \frac{2}{\sqrt{n}}
    \]
\end{frame}



\begin{frame}{Folgerung 1}
    Aus der eben gezeigten Abschätzung erhalten wir insbesondere die Abschätzung:
    \[
        \sqrt[n]{n}
        \leq 1 - \frac{2}{n} + \frac{2}{\sqrt{n}}
        < 1 + \frac{2}{\sqrt{n}}.
    \]
\end{frame}



\begin{frame}{Folgerung 2}
    Wir setzen \( w_{n} \coloneq \sqrt[n]{n} \).
\end{frame}



\begin{frame}{Folgerung 2}
    Wir setzen \( w_{n} \coloneq \sqrt[n]{n} \). Für den Grenzwert der Folge \( \left( w_{n} \right)_{n \in \bN} \) gilt:
\end{frame}



\begin{frame}{Folgerung 2}
    Wir setzen \( w_{n} \coloneq \sqrt[n]{n} \). Für den Grenzwert der Folge \( \left( w_{n} \right)_{n \in \bN} \) gilt:
    \[
        \lim_{n \to \infty} \sqrt[n]{n}
    \]
\end{frame}



\begin{frame}{Folgerung 2}
    Wir setzen \( w_{n} \coloneq \sqrt[n]{n} \). Für den Grenzwert der Folge \( \left( w_{n} \right)_{n \in \bN} \) gilt:
    \[
        \lim_{n \to \infty} \sqrt[n]{n} 
        \leq \lim_{n \to \infty} \left( 1 - \frac{2}{n} + \frac{2}{\sqrt{n}} \right)
    \]
\end{frame}



\begin{frame}{Folgerung 2}
    Wir setzen \( w_{n} \coloneq \sqrt[n]{n} \). Für den Grenzwert der Folge \( \left( w_{n} \right)_{n \in \bN} \) gilt:
    \[
        \lim_{n \to \infty} \sqrt[n]{n} 
        \leq \lim_{n \to \infty} \left( 1 - \frac{2}{n} + \frac{2}{\sqrt{n}} \right)
        = 1.
    \]
\end{frame}



\begin{frame}{Folgerung 2}
    Wir setzen \( w_{n} \coloneq \sqrt[n]{n} \). Für den Grenzwert der Folge \( \left( w_{n} \right)_{n \in \bN} \) gilt:
    \[
        \lim_{n \to \infty} \sqrt[n]{n} 
        \leq \lim_{n \to \infty} \left( 1 - \frac{2}{n} + \frac{2}{\sqrt{n}} \right)
        = 1.
    \]
    Wegen \( 1 \leq \sqrt[n]{n} \), für alle \( n \in \bN \), erhalten wir:
\end{frame}



\begin{frame}{Folgerung 2}
    Wir setzen \( w_{n} \coloneq \sqrt[n]{n} \). Für den Grenzwert der Folge \( \left( w_{n} \right)_{n \in \bN} \) gilt:
    \[
        \lim_{n \to \infty} \sqrt[n]{n} 
        \leq \lim_{n \to \infty} \left( 1 - \frac{2}{n} + \frac{2}{\sqrt{n}} \right)
        = 1.
    \]
    Wegen \( 1 \leq \sqrt[n]{n} \), für alle \( n \in \bN \), erhalten wir:
    \[
        \lim_{n \to \infty} \sqrt[n]{n}
        = 1.
    \]
\end{frame}
% =====================================================

\end{document}